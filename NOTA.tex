\chapter*{Nota preliminar}
\addcontentsline{toc}{chapter}{Nota preliminar
\bigskip}
\hedramarkboth{Nota preliminar}{}

Este livro compõe-se de quatro ensaios sobre uma das questões mais espinhosas debatidas pela teoria literária do último século: a autoria da obra literária. A sua proposta consiste em esclarecer certos pontos nebulosos desse debate a partir de sua revisão crítica e lidar com alguns de seus impasses fundamentais por meio de deslocamentos de perspectiva.

Embora se trate de textos veiculados em circunstâncias distintas, sua concepção previu uma posterior reunião em livro. Três deles tiveram sua primeira versão nos anos de 2007 e 2008, quando, como pesquisador de pós-doutorado na \textsc{usp}, recebi uma bolsa da fapesp para que pudesse me dedicar com exclusividade a esse tema, e foram publicados, mediante revisão, em revistas especializadas. O ensaio restante teve a sua primeira versão escrita como desfecho da tese de doutorado que defendi na unicamp, em 2005.

“Autoria e Heteronímia na moderna teoria da literatura” foi publicado, com outro título, em 2010, na revista Estudos Avançados, do Instituto de Estudos Avançados da usp. O segundo ensaio, “Depõe-se um rei, a paternidade do poema: ‘Chuva Oblíqua’ e a forma heteronímica em Fernando Pessoa”, é inédito. “Autor, Autoria e Autoridade: argumentação e ideologia em Roland Barthes” foi publicado, em 2012, na Revista Magma, do Departamento de Teoria Literária e Literatura Comparada da usp. Por fim, “Disfarce e Fraude Autoral: memória, testemunho e ficção” foi publicado, em 2014, na revista Criação e Crítica, da área de Estudos Linguísticos, Literários e Tradutológicos em Francês, do Departamento de Letras Modernas da usp. 

Ainda na fase de escrita dos ensaios, contei com importantes sugestões de Vagner Camilo, professor da área de Literatura Brasileira, da usp, para o texto que viria a ser o capítulo 1, e de Fábio Rigatto de Souza Andrade, professor do Departamento de Teoria Literária e Literatura Comparada, da usp, e então supervisor de minha pesquisa de pós-doutorado, para aquele que viria a ser o capítulo 3. Tive a sorte de encontrar aspectos levantados nestes ensaios sendo discutidos em alguns artigos, dissertações e teses de cujas bancas participei nos últimos anos, seja como convidado, seja como orientador. Contei ainda com a leitura estimulante do escritor Marco Catalão, bem como de alguns dos pesquisadores do Grupo Estudos Pessoanos, Daiane Walker Araujo, Mateus Ramos Lourenço e, especialmente, Flávio Rodrigo Penteado, leitor atento dos artigos em revista e, depois, revisor atencioso do volume já reunido. A eles devo o meu agradecimento e a minha amizade. Agradeço, igualmente, ao Programa de Pós-Graduação em Literatura Portuguesa, nomeadamente à sua coordenadora, Profa. Annie Gisele Fernandes, que me ofereceu as condições de publicação. Com escrúpulo redobrado, apaguei algumas repetições dos textos, deixei outras, por julgá-las significativas, cortei passagens que o tempo me revelou serem desnecessárias, procurei sanar os lapsos e as imprecisões que vim encontrando e acrescentei-lhes algumas linhas. 

Nos capítulos 1 e 3, procurei partir das reflexões mais acuradas a respeito da autoria literária, formuladas do início do século xx para cá, de modo a apresentá-la segundo suas diferentes acepções, desde a sua rejeição drástica até sua retomada menos diretiva. No capítulo 2, empenhei-me em analisar a obra literária que considero ter levado ao ponto mais alto o jogo autoral e refletir não apenas sobre ela, mas com ela. Por fim, no capítulo 4, creio ter formulado e assumido mais diretamente a proposta que vim amadurecendo ao longo do livro. 

Com este volume, espero não ter passado a pretensão de lançar a expressão “renascimento do autor” como algo original, muito menos panfletário. Em muitos sentidos, o texto literário não se ressentiu – na verdade, se beneficiou – do distanciamento da figura biográfica do autor no ato da leitura. Em outros, no entanto, quando extremado, esse apagamento levou a uma visão insípida da literatura. Essa polaridade está presente em todos os capítulos deste volume, e espero ter contribuído para esclarecer algumas posições ainda nebulosas a esse respeito e desmistificar outras. Vivemos um tempo em que o autor se destaca não apenas como ethos, como imagem autoral derivada do texto, mas como sujeito empírico, indivíduo que precede o texto. Que este livro seja lido à luz de seu tempo, é uma condição inevitável. Que ele possa contribuir para uma consciência crítica a respeito da figura do autor em sua época, é sua condição ideal. 