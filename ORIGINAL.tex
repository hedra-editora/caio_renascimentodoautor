%\textsc{o renascimento do autor}

%\textsc{autoria, heteronímia e \emph{fake memoirs}}

%\textsc{Caio Gagliardi}

%O RENASCIMENTO DO AUTOR

%AUTORIA, HETERONÍMIA E \emph{FAKE MEMOIRS}

%\textsc{sumário} 

%\section{AUTORIA E HETERONÍMIA }\label{autoria-e-heteronuxedmia}

\chapter*{Autoria e heteronímia\\ na moderna teoria da literatura}
\addcontentsline{toc}{chapter}{\large\versal{AUTORIA E HETERONÍMIA\\ NA MODERNA TEORIA DA LITERATURA}}
\hedramarkboth{Autoria e heteronímia}{}

\section*{I. Sacrificar o sujeito -- emancipar~a~linguagem}
\addcontentsline{toc}{section}{Sacrificar o sujeito -- emancipar a linguagem}


Na moderna teoria da literatura, as formas de rejeição que se vão
acumulando em torno da noção de \emph{autoria} caracterizam"-na,
sumariamente, como símbolo de um humanismo que os discursos críticos
formulados a partir dos anos 1910, na Rússia, dos anos 1930, nos
\textsc{eua} e Inglaterra, e já preponderantes na França dos anos 1960,
procuraram eliminar dos estudos estéticos. A contrapelo de uma crítica
do fenômeno literário que procura na psicologia, na biografia ou na
sociologia do indivíduo fatores determinantes do texto, a maior parte
das correntes críticas surgidas no século passado relega ao autor um
papel contingente ao fazer literário.~Esses diferentes modos de
desvalorizar a ação atribuída à intenção premeditada, quando analisados
sistematicamente, tornam possível acompanhar alguns dos passos decisivos
que fizeram avançar o pensamento crítico"-teórico no século \textsc{xx}.

A limitação ou mesmo a eliminação do autor dos estudos críticos sobre
literatura se realiza segundo a formulação de refutações teóricas que se
mostram decisivas como ferramentas de análise e como novos horizontes
conceituais de atuação crítica. Antes disso, seu momento de
cristalização poética data dos finais do século \textsc{xix}, por meio
de Rimbaud e \index{Mallarmé, Stéphane}Mallarmé, sobretudo. Acompanhemos, no entanto, suas
formulações especulativas mais sistemáticas.

Em ``A nova poesia russa'' (1919), \index{Jakobson, Roman}Jakobson define
a \emph{literariedade}, isto é, aquilo que torna um texto efetivamente
literário, como algo independente da intenção do autor, uma vez que ela
reclama atenção para o discurso, em detrimento da possível premeditação
que o terá guiado. Sumariamente, entende"-se por literariedade um ou mais
procedimentos linguísticos que conferem traços distintivos ao objeto
literário. Não se trata, pois, de um conteúdo qualquer, uma ideia, uma
imagem, uma emoção; não há, portanto, temas literários, segundo
\index{Jakobson, Roman}Jakobson. Os temas serão literários uma vez que sejam
processados \emph{literariamente}. Dessa perspectiva, o traço distintivo
da poesia reside no fato de que, nela, as palavras reagem ao seu uso
rotineiro e estereotipado. Elas não são apenas veículos de significados
ou expressão de emoções; elas e seus arranjos adquirem um peso e um
valor próprios.

Nessa defesa do discurso, a noção de \emph{autoria} não figura como
objeto de interesse do crítico formalista.~Tal apagamento é tanto ou
mais significativo quando se verifica que, em seu detrimento, até mesmo
o leitor é contemplado, por \index{Chklovski, Victor}Chklovski: ``chamaremos objeto estético, no
sentido próprio da palavra, os objetos criados através de procedimentos
particulares, cujo objetivo é assegurar para estes objetos uma percepção
estética''.\footnote{\index{Chklovski, Victor}Vitor Chklovski, ``A arte como procedimento'', in
  \emph{Teoria da literatura: formalistas russos} (organização de
  Dionísio de Oliveira Toledo; tradução de Ana Mariza Ribeiro, Maria
  Aparecida Pereira, Regina L. Zilberman e Antônio Carlos Hohlfeld),
  Porto Alegre, Globo, 1971, p. 41.} Segundo tal ponto de vista, o
caráter estético está associado à nossa maneira de perceber o objeto,
visto que um texto pode ser ``criado para ser prosaico, e ser percebido
como poético, ou então criado para ser poético e percebido como
prosaico''.\footnote{Ibidem.}~Assim, a despeito de uma possível intenção
autoral, o modo de perceber, conduzido pelo discurso, é que determina o
efeito estético.

O método morfológico e essencialmente descritivo dos formalistas russos
pode ser considerado como marco zero, na moderna teoria da literatura,
da reflexão teórica a respeito da linguagem literária e da obra como
organização artística. Para \index{Aguiar e Silva, Vítor Manuel de}Vítor Manuel de Aguiar e Silva, autor de um
dos mais influentes (e agradáveis) manuais de teoria da literatura em
língua portuguesa, ``não é fácil encontrar, em tempos anteriores, uma
reflexão tão profunda e tão rigorosa como a dos formalistas russos
acerca da natureza da linguagem literária''.\footnote{Vítor Manuel de
  Aguiar e Silva, ``O formalismo russo''. In \emph{Teoria da
  literatura}. 3ª. ed. revista e aumentada. Coimbra: Livraria Almedina,
  1973, p. 560.}

Em \emph{Contre Sainte"-Beuve}\index{Sainte-Beuve, Charles Augustin}, escrito em 1900 e publicado pela primeira
vez apenas em 1954,\footnote{Marcel Proust, \emph{Contre Sainte"-Beuve} -- notas sobre crítica e literatura (tradução de Haroldo Ramanzini), São Paulo, Iluminuras, 1988.} \index{Proust, Marcel}Proust denuncia em tom
especialmente combativo o método de projeção dos dados biográficos sobre
o perfil autoral como um retrato de superfície que passa ao largo da
obra. A denúncia que faz daquele que era então considerado o maior
crítico de seu tempo em língua francesa sistematiza uma posição já
defendida por \index{Valéry, Paul}Valéry e \index{Mallarmé, Stéphane}Mallarmé, responsáveis por conferir à palavra uma
autonomia quase mística. O escritor ironiza o ``guia inegável da crítica
no século \textsc{xix}'', preocupado em ``munir"-se de todas as
informações possíveis sobre um dado escritor, em colecionar
correspondência, em interrogar os homens que o conheceram, conversando
com eles se ainda estiverem vivos, lendo aquilo que puderam escrever,
caso estejam mortos''.~Segundo \index{Proust, Marcel}Proust, ``esse método desprezava aquilo
que uma convivência um tanto profunda com nós mesmos pode ensinar: que
um livro é o produto de um outro eu e não daquele que manifestamos nos
costumes, na sociedade, nos vícios''.\footnote{Ibid., p. 51-52.}

Especialmente sugestiva é a concepção de \emph{um outro eu}, adotada
pelo autor francês.~Um eu não biográfico, que, em \emph{Teoria da
literatura} (1948), \index{Wellek, René}Wellek e \index{Warren, Austin}Warren definiriam como ``eu fictício'', mas
sem distingui"-lo do eu lírico. Semelhante adesão ao antibiografismo
levou \index{Hamburger, Käte}Käte Hamburger a afirmar que ``não existe critério exato, nem
lógico, nem estético, nem interior, nem exterior, que nos permita a
identificação ou não do sujeito"-de"-enunciação lírico com o
poeta''.\footnote{Käte Hamburger, \emph{A lógica da criação literária}, 2ª ed (tradução de Margot Petry Malnic), São Paulo, Perspectiva, 1986, p. 196.}~Isso para dizer que a enunciação lírica, por mais que seja
uma forma de aproximação ao caráter vivencial do enunciador, não
funciona numa conexão real; não é, em suma, informação sobre alguém ou
sobre a realidade não literária.

Em outra séria contestação do método crítico de \index{Sainte-Beuve, Charles Augustin}Sainte"-Beuve, \index{Eliot, Thomas Stearns}T. S. Eliot contrapõe à investigação dos testemunhos do poeta, colhidos
sistematicamente pelo crítico francês antes mesmo do contato com o
texto, uma concepção oposta do fazer poético:~``O que acontece é uma contínua entrega de si mesmo [o poeta] tal como ele é no momento a algo que é mais valioso. O progresso de um artista é um contínuo autossacrifício, uma
contínua extinção da personalidade''.\footnote{T. S. Eliot, ``Tradition
  and the individual talent'', in \emph{Selected prose}, Londres,
  Penguin Books/ Faber and Faber, 1955, p. 26. {[}Tradução minha.{]}.}
Para \index{Eliot, Thomas Stearns}Eliot, a ``crítica honesta'' e a ``apreciação sensível'' são
direcionadas à poesia, não ao poeta. Não são as emoções pessoais,
provocadas por eventos específicos de sua vida, que interessam à poesia.
Sua complexidade é outra, as emoções reais não são formas de expressão,
mas manifestações naturais do próprio ser. Na poesia, o que conta é o
trabalho intelectual sobre essas emoções, a fim de fazê-las dizer algo
quando transpostas para outro plano e ali transformadas.~Daí a obtenção
do que \index{Eliot, Thomas Stearns}Eliot chama de ``prazer estético'', que é, segundo ele, de
natureza diferente do prazer na vida: ``Quanto mais perfeito o artista,
mais completamente separados estarão nele o homem que sofre e a mente
que cria; e mais perfeitamente a mente irá digerir e transmutar as
paixões que são o seu material''.\footnote{Ibid., p. 27. {[}Tradução
  minha.{]}.}

Em resumo, \index{Eliot, Thomas Stearns}Eliot considera a poesia não como um simples ``verter de
emoções'', mas como uma fuga delas.~Não é a expressão da personalidade,
mas o distanciamento dela que faz o poeta. Por esse motivo, o ato de
criação inconsciente torna, na visão do crítico, a poesia pessoal e, em
decorrência disso, ruim. Segundo essa concepção, há, por evidente, uma
recusa ao \emph{modus operandi} biografista, que confere ao texto o
papel de espelho de seu autor.

Do ponto de vista do método crítico, não difere desse procedimento o que
\index{Croce, Benedetto}Croce sugere, em \emph{A poesia}, ao afirmar a especificidade dos
estudos sobre o gênero requerendo que se coloque de lado todo e qualquer
dado biográfico a respeito do autor.~Para o pensador italiano, o poeta é
nada além do que sua poesia:

\begin{quote}
Que deve fazer o crítico e historiador da poesia quando se encontra ante
um amontoado de documentos e notícias sobre o poeta? Ele deve fazer o
que sempre faz quando realmente conhece o seu ofício: afastar os
documentos e notícias que se referem exclusivamente à vida privada do
poeta (\ldots{}), os que se referem exclusivamente à sua vida pública (\ldots{}),
e também tudo aquilo que concerne aos seus estudos de botânica,
anatomia, filosofia ou história (\ldots{}). O crítico e historiador deve
reter somente os documentos que se referem à poesia.\footnote{Benedetto
  Croce, \emph{A poesia} -- introdução à crítica e história da poesia e
  da literatura (tradução de Flávio Loureiro Chaves; supervisão e
  revisão de Angelo Ricci), Porto Alegre, Ed. \textsc{ufrgs}, 1965, p.
  173.}
\end{quote}

Essa perspectiva é reforçada nos anos 1940, marcados nos \textsc{eua}
pela noção de ``falácia intencional'', expressão por meio da qual
\index{Beardsley, Monroe}Beardsley e \index{Wimsatt, William K.}Wimsatt (1946) asseveraram que a explicação do texto pela
intenção do autor inutilizaria a crítica literária.~De seu ponto de
vista, encontrar o sentido do texto na intenção do autor significa
reduzir a tarefa do crítico a uma entrevista ou mera coleta de
testemunhos -- a uma investigação distinta do contato mais detido com o
próprio texto. No \emph{new criticism}, torna"-se possível entrever que a
ascensão da assim chamada ``crítica profissional'' na Inglaterra e nos
\textsc{eua} se faz sobre uma sólida baliza formalista, que exclui da
tarefa investigativa a psicologia, a biografia e a sociologia do autor
como métodos ``extrínsecos'' ao texto. É essa a denominação que lhe dão
\index{Wellek, René}Wellek, formado no Círculo Linguístico de Praga, e \index{Warren, Austin}Warren, \emph{new
critic} americano.

\begin{quote}
Não precisamos, por certo, ter uma intenção depreciativa ao afirmarmos
serem os estudos biográficos distintos dos poéticos, dentro da
especialização literária. Há, entretanto, o risco de se confundirem os
estudos biográficos e os poéticos, havendo ainda o perigo de tomar"-se o
biográfico pelo poético.\footnote{Apud Luiz Costa Lima
  (org.), \emph{Teoria da literatura em suas fontes}, 3ª ed, Rio de
  Janeiro, Civilização Brasileira, 2002, p. 647.}
\end{quote}

Nos anos 1960, assiste"-se, na França, a uma série de ataques ao
biografismo (não como gênero, mas como método crítico), cujo ``perigo''
é evitado, sobretudo, por três trabalhos fundamentais, responsáveis por
deitar por terra a tradicional imagem do autor.~Em seu estudo sobre
\index{Husserl, Edmund}Husserl, ``A voz e o fenômeno'' (1967), \index{Derrida, Jacques}Derrida\footnote{Jacques
  Derrida, \emph{A voz e o fenómeno} (tradução de Maria José Semião e
  Carlos Aboim de Brito), Lisboa, Edições 70, 1996.} é o primeiro a
combater a primazia do significado, isto é, o ``querer"-dizer'' vinculado
à figura do autor. \index{Barthes, Roland}Barthes, em seguida, lança mão daquele que seria o
seu mais radical bordão. Em ``A morte do autor'' (1968),\footnote{Roland
  Barthes, ``A morte do autor'', in \emph{O rumor da língua} (tradução
  de Mário Laranjeira), São Paulo, Brasiliense, 1988.} essa figura é
tratada como uma construção histórica e ideológica vinculada à burguesia
e ao individualismo, e que deve ser preterida em prol da autonomia do
discurso. Em sua esteira, numa conferência mais desenvolvida, \emph{O
que é um autor?} (1969), \index{Foucault, Michel}Foucault\footnote{Michel Foucault. \emph{O que
  é um autor?} (tradução de António Fernando Cascais), Lisboa, Vega,
  2002.} reflete sobre a noção de ``função autor'', num momento em que
fica claramente marcada a passagem do estruturalismo para o
pós"-estruturalismo, ou seja, para um conjunto de reflexões de caráter
crítico"-teórico em que a recusa do autor é alargada para a recusa do
significado, e, no limite, do próprio texto enquanto realidade estável.

\begin{quote}
Uma vez afastado o Autor, a pretensão de ``decifrar'' um texto se torna
totalmente inútil. Dar ao texto um Autor é impor"-lhe um travão, é
provê-lo de um significado último, é fechar a escritura. Essa concepção
convém muito à crítica, que quer dar"-se então como tarefa importante
descobrir o Autor (ou as suas hipóteses: a sociedade, a história, a
psique, a liberdade) sob a obra: encontrado o Autor, o texto está
``explicado'', o crítico venceu; não é de se admirar, portanto, que,
historicamente, o reinado do Autor tenha sido também o do Crítico, nem
tampouco que a crítica (mesmo a nova) esteja hoje abalada ao mesmo tempo
que o Autor.\footnote{Roland Barthes, op. cit., p. 69.}
\end{quote}

O argumento de \index{Barthes, Roland}Barthes põe em xeque dois reinados, para ele
indissociáveis: os reinados do autor e do crítico. Se encontrar o
significado é o mesmo que desvendar a autoria, então devemos recusar a
significação.

Nesse percurso que abrange cerca de seis décadas do século \textsc{xx},
a discussão que se estabelece sobre a \emph{autoria} permanece, com
algumas variações, a mesma. O autor é, \emph{grosso modo}, considerado
como uma necessidade típica da cultura humanista anterior à segunda
metade do século \textsc{xix}, que legava ao \emph{homem de gênio} o
mérito e o sentido de seu texto.

Se, por um lado, a explicação pela intenção, ao reduzir a crítica à
busca de uma única resposta para o texto, desautoriza a liberdade
interpretativa, por outro, essa mesma ``velha'' crítica, que pretende
``explicar'', isto é, resolver, encontrar a chave do texto, não permite
particularizar a teoria da literatura em relação a outras formas de
investigação que tomam o texto como seu objeto, como a Filologia e a
História, por exemplo.

Para \index{Derrida, Jacques}Derrida, ao proteger o terreno da teoria \emph{literária} desses
outros métodos de especulação, ao tratá-los como prolegômenos e,
portanto, como considerações dispensáveis a respeito da circunvizinhança
do texto, a teoria deve se preocupar com preservar em seu horizonte de
interesses e atuação a história da própria obra:

\begin{quote}
Obedecendo à intenção legítima de proteger a verdade e o
sentido \emph{internos} da obra contra um historicismo, um biografismo
ou um psicologismo, arriscamo"-nos a não mais prestar atenção à
historicidade interna da própria obra, na sua relação com uma origem
subjetiva que não é simplesmente psicológica ou mental.\footnote{Jacques
  Derrida, \emph{A escritura e a diferença}, 2ª ed. (tradução de Maria
  Beatriz Marques Nizza da Silva), São Paulo, Perspectiva, 1995.}
\end{quote}

Essa breve consideração do crítico permite modalizar tanto o sentido de
um texto (a partir de seu histórico de recepção, e não mais de sua
hipotética premeditação) como sua suposta intenção. A discussão dessas
possibilidades não se sobreleva, contudo, ao apagamento do autor. Para
\index{Derrida, Jacques}Derrida, ``escrever é retirar"-se''. A escrita seria um procedimento de
emancipação da linguagem: ``ser poeta é saber abandonar a palavra'',
``deixá-la falar sozinha''.\footnote{Ibid., p. 61.}

Discutir as diferentes objeções ao papel de relevo conferido à autoria
de um texto significa aqui, e numa primeira etapa, portanto, identificar
uma inclinação decisiva do pensamento crítico"-teórico no século
\textsc{xx}.

\pagebreak

\section*{II. A luta com o sentido}
\addcontentsline{toc}{section}{A luta com o sentido}

Um aprofundamento dessa discussão se desencadeia pela constatação de que
o autor que é recusado por essas diferentes correntes críticas é ainda,
de certa forma, o sujeito psicológico e biográfico presente na filologia
e no positivismo causalista da \emph{explication de texte}. É aquela
imagem autoral que se verifica nas \emph{Lundis}, de \index{Sainte-Beuve, Charles Augustin}Sainte"-Beuve, nas
deduções de caráter determinista de \index{Taine, Hippolyte}Taine e naquelas outras
universalizantes a respeito da psicologia da ``natureza humana'', que
\index{Freud, Sigmund}Freud radicalizou em textos como ``A Gradiva de Jensen'', ``Escritores
criativos e devaneios'' e ``Dostoiévski e o parricídio''.~O autor que se
procura apagar da moderna teoria da literatura não é muito diferente,
afinal, daquele concebido no romantismo, que, das formas mais variadas,
é tomado como alguém que se confessa na obra.

Para a crítica surgida junto às vanguardas modernistas e imediatamente
depois delas, críticos como \index{Pater, Walter}Walter Pater, \index{Sainte-Beuve, Charles Augustin}Sainte"-Beuve e \index{Taine, Hippolyte}Taine identificam"-se entre si por enfatizarem a visão pessoal do autor,
considerada como agente do sentido da obra.~É preciso, por esse motivo,
ponderar a respeito do alvo e do alcance das objeções formuladas pela
crítica com relação à \emph{autoria}.

Em muitos críticos, a rejeição ao eu biográfico como princípio da
criação estética não se estende à intencionalidade. Na contestação de
\index{Proust, Marcel}Proust a \index{Sainte-Beuve, Charles Augustin}Sainte"-Beuve, referida no início desta reflexão, o que está em
jogo é substituir a intenção superficial, confirmada pela vida ou pelos
testemunhos do autor, por outra, mais profunda: aquilo que o autor quis
dizer por meio dos enunciados do texto.

Por seu turno, \index{Eliot, Thomas Stearns}Eliot, em sua oposição ao método causalista, não nega a
influência de uma visão particular sobre o texto, mas afirma que ela é
fruto de uma ``experiência pessoal'' que resulta da fusão de sentimentos
e sensações de natureza diversa e inquantificável.~Por esse motivo, o
que um autor nos diz a respeito daquilo que pretendeu com seu poema é
entendido por \index{Eliot, Thomas Stearns}Eliot como uma consideração \emph{a posteriori}, que
provavelmente engloba ideias levadas em conta no ato da escrita, mas que
terão recebido relevo especial apenas quando o trabalho já estava
finalizado.

De modo análogo, \index{Wimsatt, William K.}Wimsatt e \index{Beardsley, Monroe}Beardsley não negam, em seu célebre estudo, a presença do elemento intencional na estrutura de um poema; o que recusam
é a aplicabilidade de qualquer análise genética do conceito de
intencionalidade.~Seu argumento é o de que a linguagem que é
matéria"-prima das estruturas verbais de um poema é um sistema público,
não um código privado, isto é, um sistema regido por convenções sociais
e não a consubstanciação do que se passa com um indivíduo.

Para usar uma expressão de \index{Compagnon, Antoine}Compagnon, as ``teses
anti"-intencionalistas''\footnote{Com a referência, procuro deixar claro
  o diálogo que proponho, nas duas primeiras partes deste capítulo, com
  uma obra"-chave de \index{Compagnon, Antoine}Compagnon para os estudos sobre a autoria literária:
  ``O autor'', \emph{O demônio da teoria} -- literatura e senso comum
  (tradução de Cleonice Paes Barreto Mourão e Consuelo Fortes Santiago),
  Belo Horizonte, Ed. \textsc{ufmg}, 2003.} anteriores a \index{Barthes, Roland}Barthes de
muitas maneiras não abalaram a noção de \emph{autoria}, e sim os métodos
explicativos do texto. Com radicalidade, ``A morte do autor'' procura
confiscar a \emph{autoridade} da investigação do sentido, mas, após um
período relativamente longo de apoteose do discurso, a autoria volta a
ser reivindicada pela crítica contemporânea, dessa vez sobre bases
bastante diversas daquelas rejeitadas.

Entre os caminhos traçados em direção ao \emph{autor}, talvez o mais
ortodoxo e combatido seja o percorrido por Harold Bloom. O crítico
americano faz pouco caso das objeções de \index{Barthes, Roland}Barthes e \index{Foucault, Michel}Foucault, e se mostra
empenhado em desautorizar as leituras ``multiculturalistas'' que tendem
a se caracterizar pela identificação de grupos sociais minoritários
(raciais, sexuais, étnicos, religiosos etc.) como geradores culturais.
Bloom defende, com certa antipatia, que a experiência estética é
necessariamente individual e que as possíveis formas de atuação da
superestrutura social sobre o texto são infinitamente menos importantes
do que o \emph{gênio }individual:

\begin{quote}
William Shakespeare escreveu trinta e oito peças, vinte e quatro delas
obras"-primas, enquanto a energia social jamais escreveu uma única cena.
A morte do autor é um tropo, e um tanto pernicioso; a vida do autor é
uma entidade quantificável. (\ldots{}) A morte do autor, proclamada por
\index{Foucault, Michel}Foucault, \index{Barthes, Roland}Barthes e muitos clones depois deles, é outro mito
anticanônico, semelhante ao grito de guerra do ressentimento, que
gostaria de descartar ``todos os homens brancos europeus
mortos''.\footnote{Harold Bloom, \emph{O cânone ocidental} -- os livros
  e a escola do tempo (tradução de Marcos Santarrita), Rio de Janeiro,
  Objetiva, 1994, pp. 43-45.}
\end{quote}

A despeito dos apagamentos dessa perspectiva, é importante considerá-la
como protagonista de uma inversão importante de papéis no que diz
respeito à noção contemporânea de \emph{autor}.~Para Bloom, o autor é
antes de mais nada um \emph{leitor criativo}, ou um \emph{desleitor}. É
desse ponto de vista que o papel da tradição é visto como fundamental
para a escrita: os demais autores convertem"-se na matéria"-prima daquele
que os sucede. Essa não é uma dimensão nova para a noção
de \emph{autoria}.

\begin{quote}
Qualquer poema é um interpoema e qualquer leitura de um poema é uma
interleitura. Um poema não é escritura, mas \emph{re"-escritura}, e,
apesar de um poema forte ser um novo ponto de partida, esse início é
sempre um reinício.\footnote{Harold Bloom, \emph{Poesia e repressão} --
  o revisionismo de Blake a Stevens (tradução de Cillu Maia), Rio de
  Janeiro, Imago, 1994, p. 15.}
\end{quote}

O que Bloom faz é levar ao extremo a visão de \index{Eliot, Thomas Stearns}Eliot (e, antes dele, de
Vico) a respeito da \emph{tradição}, mas numa clave psicologizante.

Dessa inversão, no que diz respeito ao lugar da autoridade de um texto,
um escritor tem especial relevo. Em ``Kafka e seus precursores'' (1951),
\index{Borges, Jorge Luis}Borges\footnote{Jorge Luis Borges, \emph{Outras inquisições} (tradução
  de Sérgio Molina), São Paulo, Globo, 1989.} perturba as noções de
\emph{originalidade} e de \emph{influência} ao inverter o ângulo das
observações sobre a tradição: para ele, é \index{Kafka, Franz}Kafka quem provoca uma leitura
criativa de seus precursores, e, mais que isso, é \index{Kafka, Franz}Kafka quem cria seus
precursores. Essa inversão da imagem autoral é fundamentalmente uma
inversão cronológica:~\index{Borges, Jorge Luis}Borges rompe com o senso comum a respeito do
passado e do futuro.~Num conto seu, ``Pierre Menard, autor do Quixote''
(1939),\footnote{Jorge Luis Borges, \emph{Ficções} (tradução de Davi
  Arrigucci Jr.), São Paulo, Companhia das Letras, 2007.} Menard teria
reescrito os capítulos 9 e 38 da obra de \index{Cervantes, Miguel de}Cervantes, e, ao reescrevê-los,
o autor o teria feito de forma idêntica ao original.~Apesar disso, ao
confrontar dois fragmentos perfeitamente iguais, o narrador borgeano os
considera totalmente diferentes.~Nessa confrontação aparentemente
absurda, o fato de \index{Cervantes, Miguel de}Cervantes reaparecer idêntico três séculos depois, ou
seja, o deslocamento temporal dos textos, modifica inteiramente seu
significado.~De modo análogo a esse, \index{Santiago, Silviano}Silviano Santiago lê \index{Queirós, Eça de}Eça de Queirós
como \emph{autor} de \emph{Madame Bovary}.\footnote{Silviano Santiago,
  ``Eça, autor de Madame Bovary'', in \emph{Uma literatura nos
  trópicos}, São Paulo, Perspectiva, 1978.}

Estamos diante de outra perspectiva de leitura, baseada no famigerado
\emph{anacronismo}: deslocar um texto de seu momento de produção
mobiliza sua imagem autoral, redefinindo seus possíveis sentidos. À luz
dessa consideração, a \emph{Estética da recepção} e
o \emph{Reader"-response} não são simplesmente um desvio de atenção da
autoria, mas sua reformulação, seu deslocamento para a outra ponta do
sistema literário (autor"-obra"-público): o leitor como legitimador do
sentido.~O autor está vivo.~O significado continua sob a tutela de
alguém, que agora deixa de ser aquele que arranja palavras no papel e
passa a ser o que as percorre com os olhos.

O leitor torna"-se autor. Eis uma hipótese interpretativa para as
diferentes fenomenologias do leitor individual (\index{Ingarden, Roman}R. Ingarden e \index{Iser, Wolfgang}W. Iser) e
coletivo (\index{Jauss, Hans Robert}H. R. Jauss e \index{Eco, Umberto}U. Eco). Possivelmente, uma de suas bases esteja
na objeção que \index{Booth, Wayne}Wayne Booth (1961) formulou ao já referido texto de
\index{Beardsley, Monroe}Beardsley e Wimsatt, e que já era uma maneira de recusar o futuro clichê
da \emph{morte do autor}. Segundo o crítico, o autor nunca se retira
totalmente de sua obra. Ele deixa nela sempre um substituto que a
controla em sua ausência: o \emph{autor implícito}. \index{Booth, Wayne}Booth afirmava que o
autor constrói seu leitor da mesma maneira como constrói o
seu \emph{segundo eu} (lembre"-se do \emph{outro eu}, de \index{Proust, Marcel}Proust), e que a
leitura mais bem"-sucedida é aquela para a qual os ``eus'' construídos
(autor e leitor) podem entrar em acordo.~O autor implícito se dirige ao
leitor implícito (ou o narrador ao narratário). Quando isso acontece, o
autor define as condições de entrada do leitor real no livro: o leitor
implícito é uma construção textual, prevista, portanto, pelo autor.

Dessa aproximação entre autor e leitor, \index{Eco, Umberto}Umberto Eco se vale com especial
atenção. Sem assumir uma posição específica na discussão, o semiólogo
italiano defende um aparente meio"-termo entre a intenção do
autor e a intenção do leitor, a que chama de \emph{intentio
operis}.~É curioso reparar no mecanismo retórico de \index{Eco, Umberto}Eco. Nele, é
constante a recorrência a exemplos pessoais e retirados de \emph{O nome
da rosa}. Note"-se o \emph{mea culpa}, em tom de apelo, do autor italiano
com seu interlocutor: ``Espero que meus ouvintes concordem que introduzi
o autor empírico neste jogo para enfatizar sua irrelevância e reafirmar
os direitos do texto''.\footnote{Umberto Eco, \emph{Interpretação e
  superinterpretação} (tradução de Monica Stahel), São Paulo, Martins
  Fontes, 1993, p. 100.}~Ao que parece, \index{Eco, Umberto}Eco tenta encobrir, não sem
engenhosidade e apelo, uma postura conservadora: diz concordar com
os \emph{new critics}, que rejeitam a intenção pré-textual do autor como
pedra de toque interpretativa, mas em seguida afirma que o autor
empírico deve ter ao menos a permissão de rejeitar certas
interpretações.~É justamente o autor de \emph{Obra aberta} quem afirma:
``Tenho a impressão de que, no decorrer das últimas décadas, os direitos
dos intérpretes foram exagerados''.\footnote{Ibid., p. 27.} A maneira
como \index{Eco, Umberto}Eco arremata sua última conferência a respeito da
``interpretação'', ou do papel e do lugar do leitor no processo de
construção do sentido, é uma reclamação da presença do \emph{autor},
referido pela palavra \emph{texto}: ``Entre a história misteriosa de uma
produção textual e o curso incontrolável de suas interpretações futuras,
o texto enquanto tal representa uma presença confortável, o ponto ao
qual nos agarramos''.\footnote{Ibid., p. 104.}~O ``texto enquanto tal''
é, afinal, uma maneira de dizer que o sentido está seguro e determinado
pelo autor.

Esses postulados são estranhos à crítica anti"-intencio- nalista.~O texto, encarado como um código autônomo com relação ao autor, há muito deixou de ser considerado um porto seguro para o leitor. Para muitos
teóricos da literatura (o \index{Barthes, Roland}Barthes de \emph{S/Z}, o \index{Derrida, Jacques}Derrida
pós"-estruturalista, \index{Iser, Wolfgang}Iser e, é claro, \index{Fish, Stanley}Fish), o texto é apenas o ponto de
partida, o estímulo inicial, a partitura com base na qual as
expectativas de leitura de certas comunidades interpretativas atuarão.
Desse ângulo, a defesa da determinação do sentido com base em leituras
``textuais'' (\index{Eco, Umberto}Eco) soa como um refugo teórico, por ser um retorno ao
porto seguro da \emph{autoria}, na discussão a respeito da liberdade do
leitor. O argumento de \index{Eco, Umberto}Eco converte"-se, como identificou \index{Compagnon, Antoine}Antoine Compagnon, numa maneira dissimulada de defender, em última análise, a supremacia do autor.

O retorno ao \emph{autor} é, assim, caracterizado por posições tanto
heterodoxas quanto tradicionalistas. No contexto das revisões a seu respeito, uma das interpretações mais bem embasadas é
uma análise retrospectiva que \index{de Man, Paul}Paul de Man faz a respeito do \emph{new
criticism}. Para \index{de Man, Paul}de Man, os críticos formalistas americanos buscam, em
comum, defender a poesia de instrumentos deterministas simplificadores
da relação complexa entre tema e estilo. No entanto, essa defesa
careceria de uma reflexão mais apurada a respeito da noção de
``intencionalidade''. Esse é o núcleo argumentativo de ``Forma e
intencionalidade no \emph{New Criticism} americano'' (1966).\footnote{Cf.
  Paul de Man, \emph{O ponto de vista da cegueira} -- ensaios sobre a
  retórica da crítica contemporânea (tradução de Miguel Tamen), Braga;
  Coimbra; Lisboa, Angelus Novus/ Cotovia, 1999.}

Em linhas gerais, \index{de Man, Paul}de Man figura entre os críticos mais empenhados em
refletir a respeito das limitações de alcance das correntes formalistas
de análise, ou, dizendo de outro modo, das correntes críticas que
relegam ao autor uma função acessória no processo de interpretação.~Ao
invés disso, ele defende a concepção de uma ``estrutura intencional da
forma literária'':

\begin{quote}
Um estudo verdadeiramente sistemático dos principais críticos
formalistas de língua inglesa dos últimos trinta anos revelaria sempre
uma rejeição mais ou menos deliberada do princípio da intencionalidade.
O resultado seria um endurecimento do texto numa mera superfície que
impede a análise estilística de penetrar para além das aparências
sensoriais e chegar até a percepção da ``luta com o sentido'', cuja
descrição deveria ser o objeto de toda a crítica, incluindo a crítica
das formas.~Com efeito, as superfícies, ao serem artificialmente
separadas do fundo que as suporta, permanecem também ocultas. O malogro
parcial do formalismo americano, que não produziu obras de primeira
grandeza, deve"-se a uma falta de consciência da estrutura intencional da
forma literária.\footnote{Ibid., p. 59.}
\end{quote}

Entre seus trabalhos relevantes para esse tema, incluem"-se, além do
estudo mencionado, as reflexões a respeito das noções de ``eu
literário'', ``impessoalidade'' e ``sublimação do eu'', aplicadas
respectivamente às críticas de \index{Poulet, Georges}Georges Poulet, \index{Blanchot, Maurice}Maurice Blanchot e \index{Binswanger, Ludwig}Ludwig Binswanger, as análises do estruturalismo e do formalismo, e, por fim, a
recensão crítica de \emph{The Anxiety of influence}, de H. Bloom.

Dessas leituras resulta uma relação mais profunda entre as noções
de \emph{autoria} e sentido, que é exemplarmente solicitada pela crítica
de \index{Compagnon, Antoine}Compagnon,\footnote{Antoine Compagnon, \emph{O demônio da
  teoria} -- literatura e senso comum (tradução de Cleonice Paes
  Barreto Mourão e Consuelo Fortes Santiago), Belo Horizonte, Ed.
  \textsc{ufmg}, 2003.} cujas coordenadas são esclarecedoras para o
presente debate.~Para o teórico francês, a presunção da intencionalidade
permanece nos estudos literários.~O núcleo de sua argumentação consiste
em desvencilhar"-se de ter que decidir entre duas posições extremas e
contrárias: o subjetivismo determinista da tese intencionalista e o
objetivismo relativista da tese anti"-intencionalista.~Para o crítico francês,
``a intenção é o único critério concebível de validade da interpretação,
mas ela não se identifica com a premeditação clara e lúcida''.\footnote{Ibid.,
  p. 79.}~Segundo a sua argumentação, pode"-se procurar num texto o que ele diz com referência ao seu próprio contexto de origem, bem como aquilo que ele diz com referência ao contexto contemporâneo ao leitor. As
alternativas, colocadas dessa forma, deixam de ser excludentes e se
tornam complementares.

Ainda segundo \index{Compagnon, Antoine}Compagnon, compreender é recuperar a intenção, não a
premeditação, porque não existe outra evidência maior para se realizar
essa tarefa do que a própria obra. Resulta dessa linha de reflexão um
aprofundamento dessa noção, tal como requerido por \index{de Man, Paul}de Man com relação a
\index{Beardsley, Monroe}Beardsley e Wimsatt. O conceito"-chave de que o teórico francês lança mão
para essa discussão é o de \emph{intenção em ato}. O que está na base
das distinções entre o texto e sua intenção é a velha oposição falaciosa
entre pensamento e linguagem. Uma vez que essa distinção seja abolida, a
intenção torna"-se aquilo que se quis dizer com o texto, e não mais antes
dele.~Dessa perspectiva, ela não é mais o projeto, mas o sentido.

\section*{III. Escrever é heteronimizar}

\addcontentsline{toc}{section}{Escrever é heteronimizar
\medskip}

Para autores como \index{Eco, Umberto}Eco, \index{Compagnon, Antoine}Compagnon e \index{de Man, Paul}de Man, termos como \emph{intentio
operis}, \emph{intenção em ato} e \emph{estrutura intencional} parecem
resolver o problema da intencionalidade por meio de sua adesão ao texto,
e não ao autor. No entanto, o que venho procurando discutir não é
especificamente a intencionalidade, mas a autoria, enquanto a primeira é um
aspecto da segunda.~Sendo o \emph{autor}, por um lado, mais do que
simplesmente o ser de carne e osso, biografizável, psicologizável e
sociologizável, que o positivismo novecentista tratou de alimentar, e,
por outro lado, alguém que ultrapassa um conjunto de intenções, é
preciso repensar esse sistema de rejeições como um movimento que não
invalida outra sua dimensão, possivelmente mais significativa.

Como tentativa de aprofundamento dessa discussão, proponho, ainda que
sucintamente, pensar o \emph{problema da autoria} com base na própria
criação literária, em especial naquela que talvez o tenha encarado de
modo mais radical. Isso significa, em outras palavras, alterar um
contexto de discussão: deslocar a poética do poeta português \index{Pessoa, Fernando}Fernando
Pessoa, especificamente a discussão que suscita a respeito da noção de
\emph{heteronímia}, ao plano do debate teórico mais amplo sobre a
\emph{autoria}.

Lembremos que, ao compor poemas em diferentes estilos e engendrando
conjuntos distintos de ideias, \index{Pessoa, Fernando}Fernando Pessoa optou por produzir
personalidades com biografias, isto é, com nomes, uma certa aparência,
um reduzido número de hábitos, local e data de nascimento, e uma
genealogia.~Traçou o mapa astral dessas personalidades e delimitou
algumas de suas leituras. Atribuiu"-lhes conjuntos de poemas e,
posteriormente, fez essas personagens criadoras interagirem entre si,
deixando transparecer dúvidas, convicções e modos distintos de
argumentar, a ponto de nos informar sobre diferentes visões de mundo.~\index{Pessoa, Fernando}Pessoa nos forneceu, em suma, contextos fictícios de produção para sua obra, por meio de diferentes \emph{autorias}.

O expediente resultou num \emph{pacto ficcional}: quando nos referimos a
\index{Caeiro, Alberto}Caeiro, \index{Campos, Álvaro de}Campos e \index{Reis, Ricardo}Reis, imaginamos sujeitos com atributos intelectuais, e não perspectivas sem dono, ou com um único dono. Quando se diz que \index{Reis, Ricardo}Reis
é despido de afetos, alguém imobilizado diante do destino das coisas,
vem à mente um ser, um autor e um núcleo de ideias das quais esse autor
tem convicção e que nos reporta. Assim, se queremos nos referir às
``Odes'', fazemos menção a \index{Reis, Ricardo}Reis, ao que ``ele'' pensava e sentia, e que,
enquanto autor, ``exprimiu'' naqueles textos. Fazemos isso mesmo sabendo
que esse ``ele'', como indivíduo, nunca existiu.

É verdade que a adesão inadvertida ao que foi proposto por \index{Pessoa, Fernando}Pessoa levou
a um psicologismo, desenvolvido na década de 1950,\footnote{João Gaspar
  Simões, \emph{Vida e obra de Fernando Pessoa} -- história duma
  geração [1951], 6ª ed., Lisboa, Publicações Dom Quixote, 1991.}
responsável por cristalizar uma imagem edipiana do poeta. Embora
ressonâncias dessa leitura sejam comuns ainda hoje, uma resposta a ela
não tardou a ser formulada. Como notou de passagem \index{Monteiro, Adolfo Casais}Adolfo Casais
Monteiro\footnote{Adolfo Casais Monteiro, \emph{Estudos sobre a poesia
  de Fernando Pessoa}, Rio de Janeiro, Agir, 1958.} e, a seguir, \index{Lourenço, Eduardo}Eduardo
Lourenço,\footnote{Eduardo Lourenço, \emph{Fernando Pessoa revisitado} -- leitura estruturante do drama em gente [1973], 2ª ed., Lisboa, Moraes Editores, 1981.} \index{Pessoa, Fernando}Pessoa não criou personalidades que
produziram poemas; \index{Pessoa, Fernando}Pessoa escreveu poemas que só depois suscitaram
personalidades. Essa assertiva conduz a uma inversão simples no nosso
modo de falar, e que dificilmente adotaremos, mas que implica dizer, por
exemplo, que ``O Guardador de Rebanhos'' é que é autor de ``\index{Caeiro, Alberto}Caeiro'', e
não o contrário -- assim como a \emph{Ilíada} e a \emph{Odisseia} são
não apenas as matrizes do mundo grego antigo que conhecemos, mas o
código genético de ``Homero''.

Despertada essa consciência, à crítica foi dada a oportunidade de
repensar parte de seu vocabulário. \index{Lourenço, Eduardo}Eduardo Lourenço cunhou a expressão
``poemas"-Caeiro'' para esvaziar o nome de personalidade e inundá-lo de
sentido, bem como de estilo.~\index{Caeiro, Alberto}Caeiro é o estilo, o eu lírico resultante
daqueles poemas, sem carne ou osso. Mas mesmo após estar desvendado o
jogo de ideias e palavras que \index{Pessoa, Fernando}Pessoa criou, continuamos a nos reportar a
\index{Caeiro, Alberto}Caeiro do mesmo modo como nos reportamos a \index{Camoes, Luis Vaz de@Camões, Luís Vaz de}Camões ou a \index{Bocage, José Maria du}Bocage; isso porque, provavelmente, atribuir uma personalidade, e, portanto, uma
\emph{autoria} a uma escrita, é uma forma habitual de designar seu
estilo.

\index{Pessoa, Fernando}Pessoa nos havia habilmente fornecido as ferramentas para isso, não
apenas para metonimizar seus textos (para nos referirmos à obra por meio
do autor:~``ler \index{Caeiro, Alberto}Caeiro'', por exemplo), mas para que pudéssemos imaginar
um indivíduo que consubstanciasse a própria arte. Assim, não nos
apresentou \index{Campos, Álvaro de}Campos como um intelectual tímido; provavelmente porque a
ideia de um homem pacato por trás da fúria orgânica de seus versos
iniciais não os traduz em vida; não nos ajuda, afinal, a imaginar, a
cultuar, como se faz com um ídolo, a sua obra.

A atitude ``moderna'' de um crítico diante de um texto é a de ignorar a
vida que o alimenta; a de um poeta é a de despersonalizar"-se no estilo
construído.~Essas são lições de época.~Embora nem sempre nos interesse
colocar em prática esse preceito, aprendemos que devemos deixar de
querer entrever no autor as qualidades do homem e de explicar a obra
pelas características do indivíduo.~Falamos então em \emph{escrita},
\emph{texto}, e evitamos seu hipotético caráter expressivo, porque a
forte carga niilista e anti"-humanista que herdamos do pensamento crítico
dos anos 1960 e 1970 impede que incorramos na falácia afirmativa de que
um texto \emph{expressa} algo exterior ou anterior a si. Um texto só
pode expressar a si mesmo -- eis o resultado, digamos, algo deceptivo
dessa lição.

Na tentativa de substituir o olhar inespecífico, e persistente, que
tratava romanticamente o indivíduo como alvo de uma arqueologia de
saberes passíveis de serem descritos por uma explicação antropológica, o
louvor a uma suposta onipotência da linguagem tinha, sem dúvida, uma
conotação liberal.~Hoje, no entanto, essa mesma atitude configura em
nossas práticas críticas novos tabus.

A heteronímia, porém, carrega algo mais antigo que isso, e que ao mesmo
tempo o ultrapassa, porque concretiza uma ilusão de vida ditada por
estilos. Consequentemente, o homem pacato que vivia na casa da tia
Anica, e que trabalhava como correspondente estrangeiro num escritório
da Baixa, pode ser deixado de lado, apagado, e os eu"-líricos, incluso o
ortônimo (uma máscara disfarçada de ``\index{Pessoa, Fernando}Pessoa'', tão ou mais
biografizável que ele próprio), vêm para substituí-lo como
\emph{autores}.

O que significa afirmar a autoria através da heteronímia? Note"-se que a
descentralização que a heteronímia impõe ao sujeito da escrita exige uma
noção mais complexa de autoria. Uma vez desconfigurada como disposição
mental de seu autor, ou conjunto de eleições de perfis, a heteronímia
pode ser entendida não só como uma maneira de prolongar o ato criador,
mas, sobretudo, como a forma de existência dessa poesia. A heteronímia
se realiza, até se perder de vista, como um jogo de autorias. Isso
porque, se a entendemos dessa maneira, ela deixa de ser um apelo
original ao prestígio de um autor único e hipotético, de uma imagem
criadora anterior à escrita, e passa a significar um estado de concreção
poética: escrever é heteronimizar. O que a heteronímia revela, de um
modo luminar, é que a escrita não se realiza \emph{por} um autor; ela se
perfaz \emph{para} um autor.

Do ponto de vista crítico, decorre daí uma enorme inversão de
perspectiva, porque implica pensar que \index{Pessoa, Fernando}Pessoa não antecede seus poemas,
que não é um autor que os alimenta de características individuais, mas
alguém que se torna autor quando coincide com a escrita, no ato da
enunciação, e nunca antes dela. Plasmado em linguagem, não será \index{Pessoa, Fernando}Pessoa
seu autor, mas \index{Caeiro, Alberto}Caeiro, \index{Campos, Álvaro de}Campos, \index{Reis, Ricardo}Reis, Soares\ldots{} e o ortônimo, ou seja, não o Fernando António Nogueira Pessoa, mas um \index{Pessoa, Fernando}Fernando Pessoa, homônimo
e recém"-nascido, presentificado pela enunciação. O que se verifica,
portanto, na recorrência de uma escrita em diferentes domínios dessa
poesia, tanto não é uma voz una, porque uma ``voz'' é sempre a fala de
alguém, de um autor hipotético, ao qual poderíamos fornecer uma
biografia etc., como não é um espaço anônimo de elocução. Mas uma voz
que, desautorizada, autoriza -- é ela mesma, afinal, que produz a
autoria. Dessa perspectiva, o sentido do texto não \emph{estava} em seu
autor, \emph{estará} nele, o autor, criação do próprio texto. A
heteronímia reafirma o autor, não como o dono pretérito do sentido, mas
como efeito de sentido de um texto.

O fenômeno heteronímico empresta à autoria um novo estado de
legitimidade enunciativa, na medida em que possibilita pensá-la como
sendo a produção de um sujeito da linguagem, um sujeito que pode ser,
até mesmo, imaginado como um corpo orgânico e anterior ao texto (eis o
jogo, afinal), mas que foi constituído e então lançado para trás por um
material genético composto por traços de estilo.~Afinal, como afirmou o
poeta russo \index{Brodski, Joseph}Joseph Brodski, ``a biografia de um escritor está nos
meandros de seu estilo''.\footnote{Joseph Brodsky. \emph{Menos que um}
  (tradução de Sergio Flaksman), São Paulo, Companhia das Letras, 1994,
  p.11.}

Tornou"-se lugar"-comum entre os críticos referir"-se a autores como
\index{Proust, Marcel}Proust, \index{Eliot, Thomas Stearns}Eliot e \index{Borges, Jorge Luis}Borges como escritores"-críticos, isto é, semeadores de problemas cruciais para a Teoria da Literatura. Dada a sua natureza
especulativa e o grau de aprofundamento com que o \emph{literário} é
discutido na obra de \index{Pessoa, Fernando}Pessoa, causa certo estranhamento a constatação de
o quão pouco o fenômeno heteronímico foi explorado de uma perspectiva
teórica imprevista por seu autor.~No Brasil, entre possíveis explicações
para isso, está o diálogo ainda tímido, quando não unilateral, entre as
áreas de Literatura Portuguesa e Teoria Literária. Isso para dizer, em
suma, que as preocupações que \index{Pessoa, Fernando}Pessoa tem suscitado entre seus críticos
são em geral preocupações circunscritas pelo próprio \index{Pessoa, Fernando}Pessoa, ou por
questões relativas à tradição portuguesa.\footnote{Conferir, a esse
  respeito, o número 3 da revista \emph{Estranhar Pessoa}, dedicado a
  discutir a autoria em Pessoa. Editores Rita Patrício e Gustavo Rubim,
  Universidade Nova de Lisboa, Outono de 2016.} A recontextualização do
problema heteronímico no campo de debates sobre a autoria, aqui
entendido como central da moderna Teoria da Literatura, tanto é capaz de
arejar um espaço de discussão exclusivo de alguns autores, como de
solicitar um passo além no território explorado por nomes como \index{Jakobson, Roman}Jakobson,
\index{Eliot, Thomas Stearns}Eliot, Wimsatt, \index{Barthes, Roland}Barthes, \index{Foucault, Michel}Foucault, \index{Eco, Umberto}Eco, \index{Derrida, Jacques}Derrida e \index{de Man, Paul}de Man.

\chapter*{Depõe-se um rei, a~paternidade~do~poema:\\ \emph{\large``Chuva Oblíqua'' e a forma heteronímica~em~Fernando~Pessoa}}

\addcontentsline{toc}{chapter}{\large\versal{DEPÕE-SE~UM~REI,~A~PATERNIDADE~DO~POEMA:}\\ {\footnotesize\emph{``Chuva Oblíqua'' e a forma heteronímica em Fernando Pessoa }}}
\hedramarkboth{Depõe-se um rei, a paternidade do poema}{}

\begin{flushright}
{\footnotesize
\parbox{160pt}{\emph{Não sou eu quem descrevo. Eu sou a tela}.}

\parbox{70pt}{Fernando Pessoa}
}
\end{flushright}

\bigskip


\section*{I. Uma voz em surdina ou o jogo da escrita}
\addcontentsline{toc}{section}{Uma voz em surdina ou o jogo da escrita}

\index{Pessoa, Fernando}Fernando Pessoa hesitou em atribuir autoria a alguns de seus poemas.
Isso aconteceu de um modo especial com o poema ``Chuva Oblíqua''. Essa é
a pista para pensarmos sobre como se deu a construção autoral em sua
obra.

É no segundo número da revista \emph{Orpheu}, lançado em Lisboa, em
1915, que aparece pela primeira vez ``Chuva Oblíqua''. Já ali, a autoria
do poema está definida, designada a ``\index{Pessoa, Fernando}Fernando Pessoa'', e não mais
sofrerá mudanças.\footnote{\emph{Orpheu 2} {[}1915{]}.
  2\textsuperscript{a}. reed. (preparação do texto e introdução de Maria
  Aliete Galhoz), Lisboa, Edições Ática, 1979, pp. 115-123.}

A justificativa cabal para essa atribuição é muito posterior a ela. Na
famosa carta escrita em 13 de janeiro de 1935 para o crítico \index{Monteiro, Adolfo Casais}Adolfo
Casais Monteiro, \index{Pessoa, Fernando}Pessoa refere"-se a ``Chuva Oblíqua'' como ``o regresso
de Fernando Pessoa \index{Caeiro, Alberto}Alberto Caeiro a \index{Pessoa, Fernando}Fernando Pessoa ele só. Ou melhor, foi a reação de F. Pessoa contra a sua inexistência como \index{Caeiro, Alberto}Alberto
Caeiro''.\footnote{Fernando Pessoa, \emph{Correspondência -- 1923-1935}
  (edição de Manuela Parreira da Silva), Lisboa, Assírio \& Alvim, 1999,
  pp. 337-348.} O trecho é riquíssimo em implicações.

Vejamos que, pouco antes do lançamento de \emph{Orpheu} 2, essa
atribuição havia sido fruto de indecisão. Em carta enviada ao
escritor \index{Cortes-Rodrigues, Armando@Côrtes-Rodrigues, Armando}Armando Côrtes"-Rodrigues, de 4 de outubro de 1914 -- escrita, portanto, cerca de sete meses depois da escrita do poema, e num período
em que o Interseccionismo (termo com que \index{Pessoa, Fernando}Pessoa designa a poética de
``Chuva Oblíqua'') significa ainda a tentativa de criar um movimento
literário em Portugal --, o poeta planeja publicar uma \emph{Antologia
do Interseccionismo}. \index{Pessoa, Fernando}Pessoa agrupa ali poetas muito diferentes, mas o
que chama especialmente a atenção é, no item 6, a atribuição de ``Chuva
Oblíqua'' a \index{Campos, Álvaro de}Álvaro de Campos:

\begin{quote}
Agora o mais importante, o que era mais preciso não esquecer dizer"-lhe.

Em vez de uma revista interseccionista, contendo o manifesto e obras
nossas, decidimos (e v., estou certo, concordará), para evitar possíveis
fiascos e não poder continuar a revista, etc., e, ao mesmo tempo, ficar
coisa mais escandalosa e \emph{definitiva}, fazer aparecer o
interseccionismo, não em uma revista nossa, \emph{mas em um volume, uma
Antologia do Interseccionismo}. Seria este, mesmo, o título.

Seria publicado logo que fosse possível, logo depois de acabada a
guerra, é de supor. A composição do volume deve ser esta, pouco mais ou
menos:
\end{quote}

\begin{enumerate}
\def\labelenumi{\arabic{enumi}.}
\item
  \begin{quote}
  Manifesto (\emph{Ultimatum}, aliás).
  \end{quote}
\item
  \begin{quote}
  Poesias e prosas de \index{Pessoa, Fernando}Fernando Pessoa.
  \end{quote}
\item
  \begin{quote}
  Poesias e prosas (``Eu-próprio o Outro'', pelo menos) do Sá-Carneiro.
  \end{quote}
\item
  \begin{quote}
  Poesias e prosas de \index{Cortes-Rodrigues, Armando@Côrtes-Rodrigues, Armando}A. Côrtes"-Rodrigues. (Vá v. vendo o que de mais
  caracteristicamente interseccionista tem; e vá mandando, para não se
  perder tempo. Não sabemos ainda ao certo o espaço que competirá a cada
  um, mas, devendo o livro ter entre 96 e 128 páginas, v. deve fazer um
  cálculo aproximado.)
  \end{quote}
\item
  \begin{quote}
  Poesias e prosas de A. P. Guisado.
  \end{quote}
\item
  \begin{quote}
  Poesias de Álvaro de \index{Campos, Álvaro de}Campos (``Chuva Oblíqua'' -- Rei Cheops, etc.)
  \end{quote}
\item
  \begin{quote}
  \emph{O Interseccionismo explicado aos inferiores}. (É aquela
  explicação do interseccionismo por meio de gráficos que, uma vez, na
  \emph{Brasileira}, lhe delineei. Recorda"-se?)\footnote{Fernando
    Pessoa, \emph{Correspondência -- 1905-1922} (organização de Manuela
    Parreira da Silva), São Paulo, Companhia das Letras, 1999, pp.
    128-129.}
  \end{quote}
\end{enumerate}

Sucede que, não sabemos se pouco tempo antes ou depois de ter escrito
esse plano de publicação, \index{Pessoa, Fernando}Pessoa teria preferido assinar o poema com
outro nome. \index{Galhoz, Maria Aliete}Maria Aliete Galhoz descobre uma nota do autor, sem data, na
qual o poeta planeja incluir poesia no \emph{Livro do Desassossego}.
Ali, é a \index{Soares, Bernardo}Bernardo Soares que se confere a paternidade de ``Chuva
Oblíqua'':

\begin{quote}
\forceindent Bernardo Soares.

Ruas dos Douradores.

Os trechos vários (Sinfonia de uma noite inquieta, Marcha Fúnebre, Na
Floresta do Alheamento).

Experiências de ultra-sensação:
\end{quote}

\begin{enumerate}
\def\labelenumi{\arabic{enumi}.}
\item
  \begin{quote}
  Chuva Oblíqua.
  \end{quote}
\item
  \begin{quote}
  Passos da Cruz.
  \end{quote}
\item
  \begin{quote}
  Os poemas de absorção musical que incluem Rio entre sonhos.
  \end{quote}
\item
  \begin{quote}
  Vários outros poemas que representam iguais experiências (Distinguir o
  ``em congruência com a esfinge'' -- se valer a pena conservá-lo -- do
  ``Em horas indas louras'' meu.)
  \end{quote}
\end{enumerate}
\vspace{-12pt}
\begin{quote}
\index{Soares, Bernardo}Soares não é poeta. A sua poesia é imperfeita e sem a continuidade que
tem na prosa; os seus versos são o lixo de sua prosa, aparas do que
escreve a valer.\footnote{Fernando Pessoa, \emph{Obra poética}
  (organização de Maria Aliete Galhoz), 3ª ed., Rio de Janeiro, Nova
  Aguilar, 2001, pp. 659-660.}
\end{quote}

Não fosse o bastante, Teresa Rita Lopes, outra especialista em desvendar
o espólio pessoano, faz uma revelação: ```Chuva Oblíqua' andará, aliás,
de mão em mão: e, estranhamente, as primeiras em que esteve, foram\ldots{} as
de Caeiro!'' Segundo a autora, numa lista de poemas atribuídos a \index{Caeiro, Alberto}Caeiro
(Esp. \textsc{bn} 48-27), \index{Pessoa, Fernando}Pessoa anota o seguinte:

\begin{enumerate}
\def\labelenumi{\arabic{enumi}.}
\item
  \begin{quote}
  O Guardador de Rebanhos. 1911-1912
  \end{quote}
\item
  \begin{quote}
  Cinco Odes Futuristas. (1913) -- 1914
  \end{quote}
\item
  \begin{quote}
  Chuva Oblíqua. 1914
  \end{quote}
\end{enumerate}

\begin{quote}
\forceindent (Poemas Inters.)\footnote{Fernando Pessoa, \emph{Poesia -- Álvaro de
  Campos} (edição de Teresa Rita Lopes), São Paulo, Companhia das
  Letras, 2002, p. 30.}
\end{quote}

Essa transferência de paternidade, num primeiro momento, nos conduz a
olhar para a heteronímia como um jogo de atribuições que é secundário à
escrita do(s) poema(s). Se \index{Pessoa, Fernando}Pessoa imagina ``Chuva Oblíqua'' como um
texto passível de receber diferentes paternidades, é porque não foram
essas ``personalidades'' que produziram o poema, que o definiram -- o
poema já estava escrito --, mas sim porque considera que o poema é que
ajudará a defini"-las.~Como será \index{Campos, Álvaro de}Campos, sendo ele o autor de ``Chuva
Oblíqua''? O que deve ter se passado com \index{Caeiro, Alberto}Caeiro para tê-lo escrito? São
hipóteses que, por certo, requereriam justificativas no plano do jogo
heteronímico, alterações nos perfis que \index{Pessoa, Fernando}Pessoa traça a \index{Caeiro, Alberto}Caeiro e \index{Campos, Álvaro de}Campos.
Mas, mais importante que isso, é considerar que essas atribuições, uma
vez concretizadas, introduziriam enormes incongruências nessas escritas.~A ``escrita"-Campos'', por assim dizer, e, especialmente, a ``escrita"-Caeiro'', não seriam já a mesma coisa em 1914, o ano de início
de sua produção, tendo ``Chuva Oblíqua'' como uma de suas fontes. Isso
porque o caráter ``cerebral'' e descritivo de ``Chuva Oblíqua'' (\index{Pessoa, Fernando}Pessoa
diz ``estático'') pouco tem a ver com o estilo impulsivo e eloquente das
grandes odes assinadas como \index{Campos, Álvaro de}``Álvaro de Campos'', em 1914 e 1915 (datas
da escrita e publicação do poema), e menos ainda com a perspectiva
atribuída a ``\index{Caeiro, Alberto}Caeiro'', que busca apreender tão somente a materialidade
do mundo exterior.~O olhar dominante em ``Chuva Oblíqua'' é, por seu
turno, marcadamente analítico e subjetivo, através do qual os mundos
exterior e interior tornam"-se um só. A atribuição do poema a Bernardo
Soares, de resto, remodelaria por completo o perfil de sua
``personalidade criadora'', porque além de resultar no redirecionamento
de uma escrita, e dos planos sobre o \emph{Livro do Desassossego},
forçaria, segundo o próprio \index{Pessoa, Fernando}Pessoa, um prosador a escrever em versos.
Logo, nesses primeiros meses que se sucederam à escrita do poema, não
parece incerto considerar que ``Chuva Oblíqua'' surge como um
complicador na obra de \index{Pessoa, Fernando}Pessoa.\footnote{Um complicador que, não
  obstante, e conforme demonstrei em um longo estudo (\emph{Fernando
  Pessoa ou do Interseccionismo}, Tese de Doutorado -- Unicamp, 2005),
  deixou rastros seminais nesses outros domínios poéticos.}

A quarta, e derradeira, paternidade do poema, \index{Pessoa, Fernando}Pessoa confere a si mesmo,
ou melhor, à assinatura com o seu nome.~Devemos indagar o que deve ter
se passado com ``\index{Pessoa, Fernando}Fernando Pessoa'', o autor dos poemas anteriores a
``Chuva Oblíqua'', para tê-lo escrito.

Segundo \index{Pessoa, Fernando}Pessoa, nada. Essa opção é descrita por ele, na carta de 1935,
como uma medida natural: para o autor, teria apenas se tratado de um
regresso a si mesmo. De um retorno ao \emph{modus operandi} anterior aos
poemas d'``O Guardador de Rebanhos'' que acabara de escrever naquela
ocasião -- ocasião esta que ele batizaria de ``Dia Triunfal''.~Mas uma
rápida análise do texto em que o poeta justifica sua decisão revela o
contrário; isso porque, na carta, esse ``regresso'' não tem, e não pode
ter, sentido literário.~Caso tivesse, \index{Pessoa, Fernando}Pessoa não poderia associar
``Chuva Oblíqua'' a um novo estilo, o interseccionista, e julgá-lo uma
evolução das poéticas anteriores, como a simbolista e a ``paúlica'', por
ele criada em 1913.

O ``regresso'' é, na verdade, a tentativa de restituição de uma autoria
perdida, e não de uma escrita. De uma autoria que havia sido substituída
por \index{Caeiro, Alberto}Caeiro.~Daí \index{Pessoa, Fernando}Pessoa ter explicado, antropologicamente, tratar"-se de
uma ``reação (\ldots{}) contra sua inexistência''. Para \index{Pessoa, Fernando}Pessoa, o autor de
``Chuva Oblíqua'' é aquele que coincide com o indivíduo \index{Pessoa, Fernando}Pessoa, não um
ser fictício, mas o que existe.

Isso é o que está afirmado na referida carta.~Mas o que começa a se
verificar na poesia de \index{Pessoa, Fernando}Pessoa é que, em torno de um mesmo nome, vão"-se
reunindo multiplicidades de estilo que escapam ao suposto comando de
personalidades criadoras bem delineadas. ``\index{Campos, Álvaro de}Campos'' é, a rigor, um nome
que passará a reunir diferentes identidades literárias.~Fruto da poesia
que \index{Pessoa, Fernando}Pessoa lhes atribui, essas personalidades lançam"-se para fora da
carta sobre a gênese dos heterônimos, da ``Tábua Bibliográfica'', das ``
Notas para a Recordação do Meu Mestre \index{Caeiro, Alberto}Caeiro'' etc., para se tornarem eu
líricos cada vez mais complexos e, afinal, dispersos.\footnote{É
  flagrante, nesse sentido, a engenhosidade da crítica para reforçar
  essas personalidades, de modo a fazê-las acompanhar a trajetória da
  obra. Lembro, por exemplo, da classificação criativa (e a mais
  específica dentre as que foram feitas) que Teresa Rita Lopes propõe
  para \index{Campos, Álvaro de}Álvaro de Campos. A crítica fala em duas fases para esse
  heterônimo, uma anterior e outra posterior a \index{Caeiro, Alberto}Caeiro. A anterior, do
  autor de ``Opiário'', seria marcada pela imagem do ``poeta
  decadente'', abertamente construída por Pessoa (1913-1914). A
  posterior está subdividida em três momentos: a do ``Engenheiro
  Sensacionista'' (1914-1923), marcada pelas grandes odes; a do
  ``Engenheiro Metafísico'' (1923-1931), em que se nota a inquietação de
  ``Tabacaria'', p.e; e finalmente a do ``Engenheiro Aposentado''
  (1931-1935), em que assistimos à desistência dos planos e a uma
  aproximação notável do tom saudoso e angustiado do ortônimo. Cf.
  Teresa Rita Lopes, ``Este \index{Campos, Álvaro de}Campos'', in \emph{Poesia -- Álvaro de
  Campos}, op. cit., pp. 25-47.}

Esse processo, a menos que nos reportemos exclusivamente ao projeto
poético, não significa um desajuste interno à obra. O que se verifica é,
mais sugestivamente, que a dissolução da heteronímia, entendida como o
contorno ficcional fornecido a diferentes escritas, está prevista em seu
próprio âmago. O que amplia a dimensão da obra de \index{Pessoa, Fernando}Pessoa é, em outros
termos, uma diversidade de sujeitos poéticos que, a exemplo do que
ocorre no \emph{Livro do Desassossego}, sobressai à diversidade de
assinaturas inventadas pelo poeta.~Essa pluralidade é uma característica
que escapa às atribuições, que vai além delas, por se constituir como
temática nos poemas.

Esse fracionamento interno à escrita de \index{Pessoa, Fernando}Pessoa contribui para a
discussão sobre a autoria de ``Chuva Oblíqua''. Acompanhemos, em quatro
etapas, a problematização desse conceito sob um ângulo mais aberto,
considerando seu \emph{modus operandi} em toda a obra:

1) Num primeiro momento, de autoafirmação, ``\index{Campos, Álvaro de}Campos'' ``dirá'': ``E os
meus versos são eu não poder estoirar de viver''.\footnote{\emph{Poesia
  -- Álvaro de Campos}, op. cit., p. 157.} O que \index{Pessoa, Fernando}Pessoa nos apresenta é
um eu lírico sensacionista cuja energia vital se transfere para seus
versos. Mas quem fala é esse próprio sujeito, o eu lírico \index{Campos, Álvaro de}Campos, usando
o possessivo ``meus''.~Assim, é preciso imaginar um \index{Campos, Álvaro de}Álvaro de Campos que
de fato existe e que se vale da escrita para se realizar.~Mesmo que se
desconsidere o apetrecho heteronímico, a imagem definida de um
\emph{sujeito} ``\index{Campos, Álvaro de}Campos'' (de dentro do texto) que se pretende um
\emph{autor} \index{Campos, Álvaro de}Campos (de fora do texto) se depreende dos poemas,
portanto.~Do seu modo, ``\index{Caeiro, Alberto}Caeiro'' também ``existe'': ``Assim tem sido
sempre a minha vida, e assim quero que possa ser sempre -- / Vou onde o
vento me leva e não me deixo pensar''.\footnote{Fernando Pessoa,
  \emph{Poesia -- Alberto Caeiro} (edição de Fernando Cabral Martins e
  Richard Zenith), São Paulo, Companhia das Letras, 2001, p. 179.} Ou
então: ``Reparem bem para mim: / Se estava virado para a direita, /
Voltei"-me agora para a esquerda, / Mas sou sempre eu, assente sobre os
mesmos pés -- / O mesmo sempre, graças ao céu e à terra / E aos meus
olhos e ouvidos atentos / E à minha clara simplicidade de
alma\ldots{}''.\footnote{Ibidem, p. 66.} E \index{Reis, Ricardo}Reis: ``Meus dias, mas que um
passe e outro passe / Ficando eu sempre quasi o mesmo, indo / Para a
velhice como um dia entra / No anoitecer''.\footnote{Fernando Pessoa,
  \emph{Poesia -- Ricardo Reis} (organização de Manuela Parreira da
  Silva), São Paulo, Companhia das Letras, 2000, p. 183.} A recorrente
cristalização das identidades poéticas ganha organicidade, não apenas
estilística, mas existencial. Para que isso ocorra, esses diferentes eu
líricos se reportam ao leitor como seres coerentes e únicos. Em sua
máxima vitalidade, essa afirmação da identidade tende a migrar para a
afirmação do ser orgânico, portanto.

2) Mas a ideia de um sujeito poético com identidade própria e distinta
da do sujeito real Fernando António Nogueira Pessoa é a fronteira que a
encenação heteronímica não ultrapassa.~Sucede que a poesia que se aloja
sob esses nomes fictícios revela uma exaltação e, posteriormente, uma
dissolução desses sujeitos poéticos.~Sua especificidade tende sempre a
se dissolver em outras unidades menores, de uma tal maneira que se faz
sentir, em cada uma dessas escritas, traços característicos das demais e
de tantas outras que não foram identificadas por \index{Pessoa, Fernando}Pessoa. É assim, por
exemplo, que, num poema atribuído a ``\index{Campos, Álvaro de}Álvaro de Campos'', surge uma voz
que claramente se identifica como ``\index{Caeiro, Alberto}Alberto Caeiro'': ``Não há sossego,
/ E os grandes montes ao sol têm"-no tão nitidamente! // Têm"-no? Os
montes ao sol não têm coisa nenhuma do espírito. / Não seriam montes,
não estariam ao sol, se o tivessem''.\footnote{\emph{Poesia -- Álvaro de
  Campos}, op. cit., p. 283.}~E os entrelaçamentos se repetirão na
dicção de \index{Reis, Ricardo}Reis, transparecendo num poema assinado como ``\index{Pessoa, Fernando}Fernando
Pessoa'': ``A vida é terra e o vivê-la é lodo. / Tudo é maneira,
diferença ou modo. / Em tudo quanto faças sê só tu, / Em tudo quanto
faças sê tu todo''.\footnote{Fernando Pessoa, \emph{Poemas de Fernando
  Pessoa -- Rubaiyat}, Edição crítica de Maria Aliete Galhoz, Lisboa,
  \versal{INCM}, 2008, p.41.}

3) Ocorre que, dentre essa profusão de vozes alojadas num mesmo domínio
(já, a essa altura, desconfigurado, portanto, como hipotética unidade),
uma voz aparece transcendendo a todas elas, flagrada como um pensamento
voltado para elas, para a heteronimização em si mesma: ``O coração pulsa
alheio, / Impossível de escutar. / Quando é que passará este drama sem
teatro, / Ou este teatro sem drama, / E recolherei a casa?''.\footnote{\emph{Poesia
  -- Álvaro de Campos}, op. cit., p. 439.}~Quem diz isso?~O sujeito do
enunciado aqui não parece ser mais \index{Campos, Álvaro de}Campos, tampouco algum outro
heterônimo.~O tom é mais o de um desabafo sem identificação.~Em
``Tabacaria'', parece mais claro que a voz insurgente, essa mesma voz,
possa ser identificada como a de \index{Pessoa, Fernando}Pessoa. Mas, notemos bem, não o \index{Pessoa, Fernando}Pessoa
indivíduo, que respira, tampouco aquele que assina os versos do
\emph{Cancioneiro}, dos \emph{Poemas Ingleses}, de \emph{Mensagem}, do
\emph{Fausto} etc., e sim de um \index{Pessoa, Fernando}Pessoa pensado como uma
supra"-identidade, que estaria presente em toda a obra e que confirmaria
um ângulo de convergência comum a seus diferentes domínios:\footnote{Essa
  formulação me ocorreu a partir da tópica latina da \emph{reductio ad
  absurdum}, que funciona como clave de leitura para \index{Lopes, Óscar}O. Lopes ler a
  poesia de \index{Pessoa, Fernando}Pessoa. O crítico entrevê uma unidade mais profunda na obra,
  percebida nos momentos em que os heterônimos recaem em contradição,
  revelando, por decorrência, essa supra"-identidade. Cf. \index{Lopes, Óscar}Óscar Lopes,
  ``\index{Pessoa, Fernando}Fernando Pessoa'', in \emph{Entre Fialho e Nemésio} -- Estudos de
  Literatura Portuguesa Contemporânea, v. 2, Lisboa, Imprensa
  Nacional"-Casa da Moeda, 1987.} ``Génio? Neste momento / Cem mil
cérebros se concebem em sonho génios como eu, / E a história não
marcará, quem sabe?, nem um,''.\footnote{\emph{Poesia -- Álvaro de
  Campos}, op. cit., p. 290.} O mesmo tema de que trata essa passagem,
Pessoa aborda em carta de 5 de junho de 1914, que envia a sua mãe,
Madalena Nogueira:

\begin{quote}
Que serei eu daqui a dez anos -- de aqui a cinco anos, mesmo? Os meus
amigos dizem"-me que eu serei um dos maiores poetas contemporâneos --
dizem"-no vendo o que eu já tenho feito, não o que poderei fazer (senão
eu citava o que eles dizem\ldots{}). Mas sei eu ao certo o que isso, mesmo
que se realize, significa? Sei eu \emph{a que isso sabe?} Talvez a
glória saiba a morte e a inutilidade, e o triunfo cheire a
podridão.\footnote{Fernando Pessoa, \emph{Correspondência -- 1905-1922},
  op. cit., p.118.}
\end{quote}

Essa voz, numa primeira instância, poderia ser designada como a
supra"-identidade existencial dos poemas. Se passássemos a ler a obra de
\index{Pessoa, Fernando}Pessoa como detetives, encontraríamos essa voz emergindo dos desabafos,
das confissões, dos lapsos e das contradições em diferentes domínios
seus.~O primeiro biógrafo de \index{Pessoa, Fernando}Pessoa, \index{Simões, João Gaspar}João Gaspar Simões, entende ser
essa a voz \emph{real} e \emph{sincera} de \index{Pessoa, Fernando}Pessoa. Autores vários, como
\index{Brechon, Robert@Bréchon, Robert}Robert Bréchon e \index{Lopes, Óscar}Óscar Lopes, identificaram a recorrência dessa voz na temática da infância em \index{Pessoa, Fernando}Pessoa, que tem forte retomada em todos os
principais heterônimos.~Esses críticos se baseiam no argumento de que a
infância reportada em muitos poemas parece ser muitas vezes a mesma, a
do indivíduo nascido no Largo de São Carlos, n. 4, às 3 horas da tarde.

Por esse ângulo, o rei estaria nu. A despeito das decorrências
temerárias da afirmação, notemos que o que há de especialmente atraente
na tentativa de se buscar uma unidade para a poesia de \index{Pessoa, Fernando}Pessoa é o
exercício de descircunstancialização que ela exige.~Para se afirmar que
existe algo que entrelace seus diferentes domínios num núcleo comum, é
preciso que se pense essa poesia com independência, livre das
preceptivas traçadas no escopo do jogo heteronímico.

4) Mas afirmar uma \emph{unidade} na poesia também não significa buscar
na decifração de um \index{Pessoa, Fernando}Pessoa sem máscara a velha \emph{sinceridade}, um
comprometimento ingênuo do texto com aquilo que se passou num momento
anterior à escrita? O problema aqui parece ser a aproximação exagerada
entre os universos ficcional e real, ou a identificação do autor com o
indivíduo. Essa aproximação impede que pensemos a unidade do texto de
outro modo que não com base na sua origem.~A ideia de unidade, uma vez
discutida a partir da escrita, deixa de ser um problema em si mesmo,
porque em vez de ser pautada na origem do texto, volta"-se para o seu
destino. Isso implica deixarmos de lado a curiosidade antropológica
sobre a natureza das vozes nos poemas (que poderão ou não coincidir com
experiências pessoais) e passarmos a pensar sobre o produto dessas
escritas, isto é, sobre os diferentes sentidos que se arranjam nesse
novo escopo. Essa voz em surdina, que se pode surpreender nos poemas
situados em diferentes instâncias da obra, não é a voz da pessoa (de
\index{Pessoa, Fernando}Pessoa) da enunciação, afinal, mas de um sujeito da linguagem,
conformado na própria escrita, que percorre livremente a obra com seu
tom meditativo, metalinguístico e silente. Essas secretas confidências
só podem ser assim encaradas por serem um traço de estilo, desobediente
à compartimentação da escrita em \index{Caeiro, Alberto}Caeiro, \index{Campos, Álvaro de}Campos, \index{Reis, Ricardo}Reis, \index{Pessoa, Fernando}Pessoa etc.~Indisciplinado em sua natureza abissal, tal traço revela, no próprio
jogo heteronímico, o mecanismo do jogo, ou, melhor dizendo, do jogo da
escrita.

\section*{II. O sujeito plasmado em linguagem}
\addcontentsline{toc}{section}{O sujeito plasmado em linguagem}

O que ocorre em 1914, quando ``Chuva Oblíqua'' passa a compor o conjunto
da obra de \index{Pessoa, Fernando}Pessoa?~Fundamentalmente, o surgimento de um novo sujeito da
linguagem, isto é, de uma nova escrita, que, embora, pela leitura
retrospectiva de poemas anteriores, tivesse deixado indícios aqui e ali
(em ``Mar. Manhã'', ``Análise'' e ``Ó naus felizes que do mar vago''),
se configura, de fato, em ``Chuva Oblíqua''. Não com outro propósito
afirmativo, \index{Gil, José}José Gil entenderá ``Chuva Oblíqua'' como o poema que desata
``o ato inaugural da escrita heteronímica''.\footnote{José Gil,
  \emph{Diferença e negação na poesia de Fernando Pessoa}, Rio de
  Janeiro, Relume Dumará, 2000, p. 59.} Lembrando que é só a partir de
1914 que \index{Pessoa, Fernando}Pessoa poderá declarar que não há evolução em sua poesia --
``não evoluo: viajo'' --, Gil entende que, ao escrever ``Chuva
Oblíqua'', o que de fato \index{Pessoa, Fernando}Pessoa produz é (para dizer como \index{Sena, Jorge de}Jorge de Sena
já o fizera há mais de meio século) um outro heterônimo mascarado de
ortônimo. \index{Sena, Jorge de}Sena dizia a \index{Pessoa, Fernando}Pessoa, postumamente, e entre parênteses: ``(e
V., quando escreveu em seu próprio nome, não foi menos heterónimo do que
qualquer deles)''.\footnote{Jorge de Sena, ``Carta a Fernando Pessoa''
  {[}1944{]}, in \emph{Fernando Pessoa \& Cª Heterônima} -- Estudos
  coligidos, 1940-1978, 3ª ed., (organização de Mécia de Sena), Lisboa,
  Edições 70, 2000, p. 20.} Esses parênteses têm uma importância seminal
na leitura crítica de \index{Sena, Jorge de}Sena sobre \index{Pessoa, Fernando}Pessoa, posteriormente desenvolvida em
muitos ensaios, e foram sabiamente assimilados pela crítica subsequente,
que passou a conferir ao trecho sua devida importância. Tratava"-se, por
certo, de um heterônimo que, afinal, como já destacou \index{Lopes, Óscar}Óscar Lopes, não
foi levado à pia de batismo.\footnote{Cf. Óscar Lopes, op. cit.} Mas por
que falar em ``inauguração'', se os poemas"-Caeiro são, tal como \index{Pessoa, Fernando}Pessoa
afirma na carta, aqui mencionada, a \index{Monteiro, Adolfo Casais}Casais Monteiro, imediatamente
anteriores a ``Chuva Oblíqua''? Voltaremos a isso mais adiante.

A vantagem desse ponto de vista é que ele tanto chama a atenção para a
novidade do poema interseccionista, quanto funciona como uma
interpretação consistente da máscara que engendra o declarado ``retorno
a si'' de \index{Pessoa, Fernando}Pessoa, após a escrita dos poemas d'``O Guardador de
Rebanhos''. O seu ponto fraco é que essa é uma visão que complementa o
jogo heteronímico. Se afirmarmos o nascimento de um outro heterônimo no
ortônimo, consideraremos também, e por decorrência, que, para se tratar
adequadamente de ``Chuva Oblíqua'', há a necessidade de se recorrer ao
prestígio de uma \emph{autoria}. Se agirmos dessa forma, reforçaremos,
por fim, o reinado do autor sobre o texto. Eis o ponto crítico desta
reflexão.

Não será possível conceber um texto sem que se conceba sua autoria?

Não seria ``Chuva Oblíqua'', em síntese, um texto sem autoria, um poema
que promove o apagamento dessa instância anterior a si, ao se
configurar, ele mesmo, como texto fundador? Entendamos que a escrita do
poema tem uma característica fundadora, justamente por descentralizar a
imagem dessa literatura da imagem de seu autor.

Para entender como isso se dá enquanto processo poético, partamos da
constatação de um procedimento que se repete.

Em todas as partes de ``Chuva Oblíqua'', verifica"-se uma bipolaridade
entre duas paisagens distintas. De um lado, as árvores ao sol
(\textsc{i}), a igreja (\textsc{ii}), o papel em que se escreve
(\textsc{iii}), o quarto (\textsc{iv}), a feira com o \emph{carroussel}
(v), e a música que se ouve no teatro (\textsc{vi}). Do outro, um porto
com grandes navios (\textsc{i}), a chuva (\textsc{ii}), a Grande Esfinge
do Egito (\textsc{iii}), o lado de fora do quarto, onde é Primavera
(\textsc{iv}), as árvores (\textsc{v}), e a infância (\textsc{vi}).~É
possível distinguir essas diferentes paisagens, ainda que, desde o
início dos textos, elas já se apresentem conectadas. Seja pela
intromissão de um passado no ato perceptivo, seja pela entrega ao
devaneio, a escrita desaloja as sensações de um \emph{agora} exterior ao
eu lírico e o reporta para uma outra paisagem, que, em geral,\footnote{Digo``em
  geral'' porque, no poema \textsc{ii}, as duas paisagens também podem
  ser lidas como exteriores ao sujeito da linguagem.} é por ele evocada.~Subitamente, essas duas paisagens se fundem, de modo a anular a referência exterior. São ambas interiorizadas nesse processo, e a
distinção temporal (entre presente e passado) e espacial (entre dentro e
fora) é perdida. A percepção cola"-se ao devaneio e aquilo que se
descreve é já uma terceira paisagem, desordenada e ilógica. Em alguns
momentos, a escrita lembrará um registro surrealista \emph{avant la
lettre} nessa fase final do processo.~Assim, lemos: (\textsc{i})
``Súbito toda a água do mar do porto é transparente / E vejo no fundo,
como uma estampa enorme que lá estivesse desdobrada, / Esta paisagem
toda, renque de árvores, estrada a arder em aquele porto,'';
(\textsc{ii}) ``Até só se ouvir a voz do padre água perder"-se ao longe /
Com o som de rodas de automóvel\ldots{}''; (\textsc{iii}) ``E sobre o papel
onde escrevo, entre ele e a pena que escreve / Jaz o cadáver do Rei
Quéops, olhando"-me com os olhos muito abertos, / E entre os nossos
olhares que se cruzam corre o Nilo''; (\textsc{iv}) ``E num canto do
teto, muito mais longe do que ele está, / Abrem mãos brancas janelas
secretas / E há ramos de violetas caindo / De haver uma noite de
primavera lá fora / Sobre o eu estar de olhos fechados\ldots{}'';
(\textsc{v}) ``E, misturado, o pó das duas realidades cai / Sobre as
minhas mãos cheias de desenhos de portos / Com grandes naus que se vão e
não pensam em voltar\ldots{}''; (\textsc{vi}) ``E do alto dum cavalo azul, o
maestro, jockey amarelo tornando"-se preto / Agradece, pousando a batuta
em cima da fuga dum muro, / E curva"-se, sorrindo, com uma bola branca em
cima da cabeça, / Bola branca que lhe desaparece pelas costas
abaixo\ldots{}''. As coisas, em suma, vestem"-se de outras, perdem suas linhas
de demarcação.

O que se percebe em comum nesses trechos é um processo de dissolução da
percepção clara, através da sobreposição dos elementos constituintes dos
universos de duas realidades (exterior e interior), implicando a perda
da referência espaçotemporal:~``De repente alguém sacode esta hora dupla
como numa peneira''. No poema \versal{III}, o eu lírico escreve, não no presente,
tampouco no passado, mas num presente do passado: ``De repente paro\ldots{} /
Escureceu tudo\ldots{} Caio por um abismo feito de tempo\ldots{} / Estou soterrado
sob as pirâmides a escrever versos à luz clara deste candeeiro / E todo
o Egito me esmaga de alto através dos traços que faço com a pena\ldots{}''
Sem um referente temporal preciso no poema, e com um espaço exterior que
também é já um fluxo móvel de paisagens intercambiáveis, dissolve"-se a
unidade individual do eu lírico, que é fracionado à medida que sua
percepção deixa de se respaldar em parâmetros racionais. Assim, a sua
alma ganhará um outro lado, como se o interior tivesse ainda um outro
interior: ``Liberto em duplo, abandonei"-me da paisagem abaixo\ldots{}'',
``Não sei quem me sonho'', ``{[}a sombra duma nau mais antiga{]} chega
ao pé de mim, e entra por mim dentro, / E passa para o outro lado da
minha alma\ldots{}''. É como se a bipolaridade constituinte do poema,
prevista em sua estrutura particular, fosse introjetada no sujeito da
linguagem.~Passa a existir, portanto, um espaço intersticial nesse
sujeito, aberto por essa diagonal, e provavelmente não mais restaurado:
um ``entre mim e o que eu penso''.

Na medida em que se coloca em questão a identidade do sujeito da
linguagem, surge no poema um espaço da escrita que parece gerar
processos de linguagem independentes do eu lírico. No poema
\textsc{iii}, o sujeito do primeiro verso não é o eu lírico das partes
anteriores; ali, quem fará esse papel é a ``Grande Esfinge do Egito''. É
ela que sonha pelo papel: ``A Grande Esfinge do Egito sonha por este
papel dentro\ldots{}''. Isso significa que a imagem que tomamos como evocada
não é objeto de evocação, mas o próprio sujeito da linguagem. Note"-se
que o que escreve não é um \emph{alguém}, mas uma mão transparente que
deixa transparecer a Esfinge através de si. Não há consistência orgânica
na prática dessa escrita. Mais adiante, o sujeito da escrita passa a ser,
não essa mão transparente, que já carrega em si algo do automatismo
surrealista (a escrita concentra"-se na mão, não na mente), mas a própria
``pena'': ``E sobre o papel onde escrevo, entre ele e a pena que escreve
(\ldots{})''. Aqui, é efetivamente a linguagem que fala, não seu autor. Essa
escrita destitui o autor do centro do processo criativo, porque lhe
relega o papel de instrumento de sua realidade inconsciente, sobre a
qual não exerce controle.

Note"-se que essa descentralização do sujeito da escrita exige uma noção
mais profunda da heteronímia. Uma vez desconfigurada como disposição
mental de seu autor, ou conjunto de eleições de perfis, a heteronímia
pode ser entendida não só como uma maneira de prolongar o ato criador,
mas, sobretudo, como a forma de existência dessa poesia. Isso porque, se
a entendemos dessa maneira, ela deixa de ser um apelo original ao
prestígio de um autor hipotético, de uma imagem criadora anterior à
escrita, e passa a significar um estado de concreção poética, a qual é
feita das intertextualidades internas dessas escritas.

Do ponto de vista crítico, decorre daí uma enorme inversão de
perspectiva, porque implica pensar que \index{Pessoa, Fernando}Pessoa não antecede seus poemas,
que não é um autor que os alimenta de características individuais
independentes da escrita. Plasmado em linguagem, não é mais voz que se
expressa, mas \emph{autor} apenas quando presente no momento da
enunciação. O que se verifica, portanto, na recorrência de uma escrita
em diferentes domínios dessa poesia, não é uma voz una, porque uma
``voz'' é sempre a fala de alguém, de um autor hipotético, ao qual
poderíamos fornecer uma biografia, psicologia etc. O que se nota é um
espaço sem identificação, um espaço de elocução em que sensações e
sentimentos variados produzem para o leitor um sujeito da linguagem, um
sujeito sem corpo, constituído por traços de estilo que parecem ser os
mesmos em diferentes momentos dessa poesia.~Esse espaço se concretiza,
em última instância, como o lugar da escrita pessoana a partir de 1914:

\begin{verse}
Vivem em nós inúmeros;\\
Se penso ou sinto, ignoro\\
Quem é que pensa ou sente.\\
Sou somente o lugar\\
Onde se pensa ou sente.\\[5pt]
Tenho mais almas que uma.\\
Há mais eus do que eu mesmo.\\
Existo todavia\\
Indiferente a todos.\\
Faço"-os calar: eu falo.\\[5pt]
Os impulsos cruzados\\
Do que sinto ou não sinto\\
Disputam em quem sou.\\
Ignoro"-os. Nada ditam\\
A quem me sei: eu escrevo.\footnote{\emph{Poesia -- Ricardo \index{Reis, Ricardo}Reis}, op.
  cit., p. 139-140.}
\end{verse}

\section*{III. Eu sou a tela}
\addcontentsline{toc}{section}{Eu sou a tela
\medskip}

Em ``Chuva Oblíqua'', portanto, do fracionamento do eu sensível dos
poemas decorre a dissolução da \emph{autoria} e a afirmação de um estado
de escrita, a transformação de um sujeito sensitivo em, note"-se bem,
lugar das sensações. O que vale frisar é que, ao engendrar uma escrita
fundadora, ``Chuva Oblíqua'' se revela o poema de uma realeza
destituída, rasurada, um poema que rejeita filiação por converter"-se,
ele mesmo, em matriz. Pela sua natureza autoral, a sua dignidade é,
pois, profundamente paradoxal. E não será outra a condição primeira para
a escrita heteronímica, que, de resto, é temática retomada em todas as
instâncias dessa poesia: ``Não sou eu quem descrevo. Eu sou a tela / E
oculta mão colora alguém em mim''.\footnote{Poema \textsc{xi} de
  ``Passos da Cruz''. In: Fernando Pessoa, \emph{Poesia -- 1902-1917}
  (edição de Manuela Parreira da Silva, Ana Maria Freitas e Madalena
  Dine), São Paulo, Companhia das Letras, 2006, p. 396.} É assim em
muitos poemas posteriores do ortônimo:

\begin{verse}
Emissário de um rei desconhecido,\\
Eu cumpro informes instruções de além,\\
E as bruscas frases que aos meus lábios vêm\\
Soam"-me a um outro e anómalo sentido\ldots{}\footnote{Poema \textsc{xiii}.
  Ibid., p. 397.}
\end{verse}

Na verdade, o que acontecerá a partir de então é que \index{Pessoa, Fernando}Pessoa sofrerá as
consequências do poema.~Sua escrita engendrará o fracionamento do
sujeito da linguagem, a metalinguagem, bem como uma série de aporias
típicas de ``Chuva Oblíqua'', aspectos que constituirão o núcleo da
escrita heteronímica.

Essa afirmação de um lugar da escrita não é, portanto, apenas comum nos
poemas assinados como ``\index{Pessoa, Fernando}Fernando Pessoa''. Em ``\index{Reis, Ricardo}Reis'' ela também se
repete:

\begin{verse}
Severo narro. Quanto sinto penso.\\
Palavras são ideias.\\
Múrmuro, o rio passa, e o som não passa,\\
Que é nosso, não do rio.\\
Assim quisera o verso: meu e alheio\\
E por mim mesmo lido.\footnote{\emph{Poesia -- Ricardo \index{Reis, Ricardo}Reis}, op. cit.,
  p. 128.}
\end{verse}

Em ``\index{Campos, Álvaro de}Campos'', numa passagem muito próxima àquela da Grande Esfinge, do
poema \textsc{iii} de ``Chuva Oblíqua'', lemos:

\begin{verse}
Às vezes tenho ideias felizes,\\
Ideias subitamente felizes, em ideias\\
E nas palavras em que naturalmente se \qb{}despejam\ldots{}\\[5pt]
Depois de escrever, leio\ldots{}\\
Por que escrevi isto?\\
Onde fui buscar isto?\\
De onde me veio isto? Isto é melhor do que \qb{}eu\ldots{}\\[5pt]
Seremos nós neste mundo apenas canetas \qb{}com tinta\\
Com que alguém escreve a valer o que nós \qb{}aqui traçamos?\footnote{\emph{Poesia
  -- Álvaro de Campos}, op. cit., p. 482.}
\end{verse}

Mas, ao contrário do que se verifica tanto no ortônimo, quanto em
``\index{Reis, Ricardo}Reis'' e em ``\index{Campos, Álvaro de}Campos'', esse espaço da escrita não existe nos
poemas"-Caeiro. Ali é sempre, sem exceção, a voz de um eu lírico que se
afirma, como condutor absoluto da escrita. Daí a forte presença do
pronome possessivo, ``meu(s) versos'', do pessoal do caso reto, ``o que
\emph{eu} escrevo'', ou do oblíquo, ``os que \emph{me} lerem'', sempre
em primeira pessoa:

\begin{verse}
\textsc{i}\\[5pt]
Não tenho ambições nem desejos.\\
Ser poeta não é uma ambição minha.\\
É a minha maneira de estar sozinho.\\
(\ldots{})\\
Saúdo todos os que me lerem,
\end{verse}

\begin{verse}
\textsc{xiv}\\[5pt]
Não me importo com as rimas. Raras vezes\\
Há duas árvores iguais, uma ao lado da \qb{}outra.\\
Penso e escrevo como as flores têm cor\\
Mas com menos perfeição no meu modo de \qb{}exprimir"-me\\
Porque me falta a simplicidade divina\\
De ser todo só o meu exterior
\end{verse}

\begin{verse}
\textsc{xxviii}\\[5pt]
Por mim, escrevo a prosa dos meus versos\\
E fico contente,
\end{verse}

\begin{verse}
\textsc{xxix}\\[5pt]
Nem sempre sou igual no que digo e escrevo.\\
Mudo, mas não mudo muito.
\end{verse}

\begin{verse}
\textsc{xlvi}\\[5pt]
Procuro dizer o que sinto\\
Sem pensar em que o sinto.\\
Procuro encostar as palavras à ideia\\
E não precisar dum corredor\\
Do pensamento para as palavras.\\
(\ldots{})\\
E assim escrevo, querendo sentir a Natureza, \qb{}nem sequer como um homem,\\
Mas como quem sente a Natureza, e mais \qb{}nada.
\end{verse}

\begin{verse}
\textsc{xlviii}\\[5pt]
Da mais alta janela da minha casa\\
Com um lenço branco digo adeus\\
Aos meus versos que partem para a \qb{}humanidade.
\end{verse}

\index{Caeiro, Alberto}Caeiro é sempre um sujeito lírico que se expressa.

O que significará essa completa ausência em \index{Caeiro, Alberto}Caeiro de um espaço de
escrita que suplanta a afirmação autoral? A resposta mais óbvia é que
\index{Caeiro, Alberto}Caeiro foi designado por \index{Pessoa, Fernando}Pessoa como \emph{mestre}, a instância autoral
seminal da obra. E, portanto, ele deve se afirmar como tal. \index{Caeiro, Alberto}Caeiro é
mestre, sobretudo, por representar esse papel.

Mas, a essa altura, já podemos compreender essa designação como
circunstanciada numa esfera que não vai muito além do fabular nessa
obra.~Como vimos em ``Chuva Oblíqua'', o agente multiplicador da
perspectiva e fracionador da consciência, do espaço e do tempo, na
poesia de \index{Pessoa, Fernando}Pessoa, é a escrita sem condutor.\footnote{Expressão que
  substituo ao que \index{Gil, José}José Gil chamou de ``puro ato de escrever''. Cf. José
  Gil, \emph{Diferença e negação na poesia de \index{Pessoa, Fernando}Fernando Pessoa}, op. cit,
  p. 60.} Os poemas"-Caeiro não engendram esse espaço porque eles são uma
voz, são enunciados atrelados à imagem fictícia de um enunciador que é
dono dos enunciados.~Em oposição a esse traço, nas demais instâncias da
poesia de \index{Pessoa, Fernando}Pessoa, a certa altura, como vimos, a autoafirmação do eu
lírico como sujeito enunciador é invertida.

Se pensarmos exclusivamente no desenvolvimento da escrita de \index{Pessoa, Fernando}Pessoa, de
um estágio pré-heteronímico para um estágio heteronímico, sinto"-me
inclinado a afirmar aqui o contrário do que \index{Pessoa, Fernando}Pessoa declarou na carta
sobre a gênese dos heterônimos: que, textualmente considerado, é ``Chuva
Oblíqua'' que heteronimiza a obra, e que, portanto, ele é um poema
anterior aos poemas"-Caeiro. Não pretendo, com isso, polemizar a respeito
de seu aparecimento cronológico, mas adotar um procedimento de leitura.
Mais precisamente, é ali que, no poema \textsc{iii}, uma ``mão enorme
varre'' todo o passado da escrita de \index{Pessoa, Fernando}Pessoa para um canto esquecido,
atrás de si, e (note"-se a ``paisagem'') sobre o \emph{papel} em que
escreve ``jaz o cadáver do Rei Quéops''.

É sobre o papel -- é na escrita, portanto -- que desaparece o mediador,
o rei enunciador, e surge a capacidade irrefreável de enunciar de
diferentes modos. Nesse ponto, o diálogo com o texto de J. Gil é
esclarecedor:

\begin{quote}
(\ldots{}) ou seja, é sobre o que se interpõe e medeia a superfície da
inscrição e a pena que inscreve, que uma nova escrita se torna possível
-- sobre um mediador morto, mediador do novo, do recém"-nascido. Cheops
morre deslizando entre o papel e a pena, indicando que já não constitui
um obstáculo, ele que surge em perfil no bico de pena.\footnote{Ibidem,
  p. 59.}
\end{quote}

Como mecanismo de pensamento e escrita, é a partir do poema \textsc{iii}
de ``Chuva Oblíqua'' que se desencadeia o processo heteronímico.~Não por
acaso, com exceção concedida a ``\index{Reis, Ricardo}Reis'', \index{Pessoa, Fernando}Pessoa cogitará atribuir a
\emph{autoria} do poema a todos os seus principais heterônimos.~\emph{Grosso modo}, quando bate o martelo em sua própria assinatura, \index{Pessoa, Fernando}Pessoa (heteronimizado) funda o ortônimo.\footnote{Afirmando isso,
  procuro superar uma série de leituras e de apagamentos que não
  depreendem o sentido profundo dessas diferentes atribuições de
  autoria, e que têm por eixo a seguinte colocação: ``Pouco importa que,
  tempos depois, quando, juntamente a \index{Caeiro, Alberto}Alberto Caeiro, aparecer no coro
  interno de \index{Pessoa, Fernando}Fernando Pessoa o whitmaniano \index{Campos, Álvaro de}Álvaro de Campos, ele
  considere por um momento atribuir"-lhe a paternidade de ``Chuva
  Oblíqua''; ou que projete incluir o poema interseccionista no
  \emph{Livro do Desassossego}, transferindo"-lhe a paternidade a
  Bernardo Soares. (\ldots{}) Naquele período, o que ocupava \index{Pessoa, Fernando}Pessoa era o
  problema da ``sinceridade''. E, nesse sentido, o que ele condenava não
  era ``Chuva Oblíqua'' em si, mas o movimento que o sucedeu, o
  prolongamento mundano e vanguardístico.'' {[}Tradução minha.{]}. Cf.
  Luciana Stegagno Picchio, ``Chuva Oblíqua: dall'Infinito turbolento di
  F. Pessoa all'Intersezionismo portoguese'', in \emph{Quaderni
  Portoguesi}, n. 2, Pisa, Outono 1977, pp. 51-52.} Agora fica claro que
\index{Pessoa, Fernando}Pessoa não precisava, enfim, conferir \emph{autoria} a uma escrita que,
por princípio, rejeita qualquer autoria. E, ainda, que seria uma
contradição referi"-la a uma outra ``personalidade criadora'', como pode
sugerir a interpretação de Gil: ``Assim nasce o novo heterônimo,
\index{Pessoa, Fernando}Fernando Pessoa, aquele cujas distâncias temporais vão alimentar
constantemente o pensamento poético''.\footnote{Op. cit., p. 59.}~Eu
modalizaria essa afirmação, partilhando da constatação seniana de que o
ortônimo nada mais é do que um outro heterônimo (na medida em que ele é
fruto de um processo de heteronimização, tal como referi), mas
acrescentando que é essa mesma assinatura, sem o arcabouço ficcional que
está por trás das demais, que melhor condiz com a criação do mecanismo
de escrita que passará a operar daí em diante. É \index{Pessoa, Fernando}Pessoa escrevendo um
texto muito diferente de si, do que escrevera até 1913, porque, afinal
de contas, esse sujeito não dirá a nova escrita, será dito por ela -- e
daí os poemas"-Caeiro não engendrarem, em si mesmos, um processo de
heteronimização.

O \emph{sistema poetodramático}, para falar como Seabra, não se entrevê
em \index{Caeiro, Alberto}Caeiro, não está em \index{Caeiro, Alberto}Caeiro. E, de fato, será apenas em 1935 que
\index{Pessoa, Fernando}Pessoa traçará uma sequência ao seu processo de criação no 8 de março de
1914. Se dermos crédito a ele, em vez de afirmar, como fiz há pouco, a
anterioridade de ``Chuva Oblíqua'' a ``O Guardador de Rebanhos'' como
dispositivo de escrita e, por decorrência, de cisão da consciência e do
tempo, bastará inverter os termos dessa formulação, sem, contudo,
alterar o seu sentido: é a (suposta) tentativa de ``regresso'' a um
``eu'' que foi deixado para trás, a qual podemos datar (1908-1913), o
que põe em marcha o sistema poetodramático. Assim, afirmará \index{Gil, José}José Gil:
``\index{Pessoa, Fernando}Fernando Pessoa tem agora a capacidade de se multiplicar, de opor de
mil maneiras o seu Eu presente ao seu Eu passado"-presente, ou o seu
pensamento às suas sensações. Quer dizer, possui o poder de devir, de
deixar de ser ele para se tornar outro''.\footnote{Ibid., p. 60.} Mas
nada impede que nos apoiemos nos testemunhos antropológicos que
\index{Pessoa, Fernando}Pessoa nos deu com duas décadas de atraso à escrita do poema, na carta
sobre a gênese dos heterônimos, ou que, sob a assinatura ``\index{Campos, Álvaro de}Álvaro de
Campos'', nos forneceu nas ``Notas para a Recordação do Meu Mestre
\index{Caeiro, Alberto}Caeiro'', algumas delas publicadas na revista \emph{Presença}, em 1931: ``Mas o
\index{Pessoa, Fernando}Fernando Pessoa era incapaz de arrancar aqueles extraordinários poemas
do seu mundo interior se não tivesse conhecido \index{Caeiro, Alberto}Caeiro. Mas, momentos
depois de conhecer \index{Caeiro, Alberto}Caeiro, sofreu o abalo espiritual que produziu esses
poemas''.\footnote{Fernando Pessoa, \emph{Poemas completos de Alberto
  Caeiro} (edição de Teresa Sobral Cunha), Lisboa, Presença, 1994, pp.
  161-162.} \index{Caeiro, Alberto}Caeiro é a construção autoral mais pungente dessa obra. Ele
surge como uma necessidade de fabular, após a revelação magistral de
``Chuva Oblíqua'', um ente gerador, um mestre ficcional a quem se pudesse atribuir
o processo já iniciado.

Embora não se associe claramente a nada produzido
anteriormente,\footnote{``A atestá-lo está toda a poesia de Fernando
  Pessoa ortônimo de 1908 a 1913, em que a singularidade (heteronímica)
  do poeta está longe de se encontrar estabelecida: nem clivagens do Eu
  (\ldots{}), nem centração da paisagem no espaço interior, nem oposições
  fortes entre sensação e pensamento, ou entre presente e passado --
  tudo o que marcará o regime poético depois de 1914''. In: José Gil,
  op. cit., pp. 59-60.} ``Chuva Oblíqua'' não se apresenta sob a
mediação de uma ``personalidade criadora'', como, afinal, acontece com
os ``trinta e tantos poemas'' (na verdade, 49) de ``O Guardador de
Rebanhos'', ou com a ``Ode Triunfal'', por exemplo.~Não há um suposto
\emph{alguém}, como \index{Caeiro, Alberto}Caeiro ou \index{Campos, Álvaro de}Campos, que aponte seus holofotes de
ficção sobre o poema; trata"-se, pois, de um deceptivo ``mesmo'' \index{Pessoa, Fernando}Fernando
Pessoa, só que transformado.~E não será função da crítica armar esses
holofotes.~Em contraponto com \index{Caeiro, Alberto}Caeiro, a novidade do poema surge, afinal, como um ente desalojado; é
essa sua forma de apresentação. E é preservando sua orfandade que,
conforme procurei mostrar, tiramos maior significado desse poema
seminal: um texto que, ao apagar sua própria autoria, retraça o percurso
de uma obra.

\chapter*{Autor, autoria e autoridade:\\ \emph{\large Argumentação e ideologia em Roland Barthes}}

\addcontentsline{toc}{chapter}{\large\versal{AUTOR, AUTORIA E AUTORIDADE:}\\ {\footnotesize\emph{Argumentação e ideologia em Roland Barthes}}}
\hedramarkboth{Autor, autoria e autoridade}{}



Passemos à discussão de um dos marcos teóricos do século \textsc{xx}, um
dos textos mais influentes a respeito da \emph{autoria} da obra
literária. Trata"-se do símbolo de uma época em que a teoria da
literatura buscou, como em nenhuma outra, afirmar"-se como uma
\emph{ciência do texto}, excluindo de seus interesses tudo aquilo que
não considerasse próprio da linguagem -- um largo espectro que
compreende e destaca a figura do autor. Esse pequeno artigo, ``A morte
do autor'' (1968), pode ser lido como um manifesto, e sua enorme
repercussão se deve em parte a isso, em parte à chancela de seu autor,
Roland \index{Barthes, Roland}Barthes, um dos intelectuais de maior notoriedade na segunda
metade do século passado.

A expressão ``a morte do autor'' é hoje um símbolo da crítica
estruturalista.~Dentre os textos sobre o conceito de autoria, o de
\index{Barthes, Roland}Barthes foi aquele que mais radicalmente procurou banir o autor das
abordagens literárias, sem, no entanto, e ironicamente, ter se libertado
da própria autoria: é justamente \index{Barthes, Roland}Barthes, afinal, com sua magnética
presença autoral, um de seus propulsores e constituidores de sentido.
Esse é o índice de utopia de sua tese; aquilo que, afinal, abre uma
fissura para a investigação de suas bases. Bloom desvia do problema nos
seguintes termos:

\begin{quote}
Era moda, até pouco tempo atrás, falar da ``morte do autor'', mas essa
noção também já virou lixo. O gênio morto está mais vivo do que nós,
assim como Falstaff e Hamlet estão bem mais vivos do que muita gente que
conheço. A vitalidade é a medida do gênio literário. Lemos em busca de
mais vida, e só o gênio é capaz de nos prover de mais vida.\footnote{Harold
  Bloom, \emph{Gênio} -- os 100 autores mais criativos da história da
  literatura. (tradução de José Roberto O´Shea), Rio de Janeiro,
  Objetiva, 2003, p.27.}
\end{quote}

Passadas quase cinco décadas, aprendemos que ler um texto de acordo com
seu autor é uma atitude acrítica, falaciosa do ponto de vista da lógica
do texto, mas essa restrição não nos indispõe contra sua figura. O autor
não só continua vivo, como o consideramos ponto referencial em nossa
cultura, e um dos temas mais controversos entre os estudos literários
contemporâneos. Por isso, vale a pena retornar aos termos específicos
daquele texto e procurar compreender suas motivações, estratégias e o
legado que deixou para a teoria literária contemporânea.

\section*{I. A argumentação}
\addcontentsline{toc}{section}{A argumentação}

A tese de que o autor está fora do espaço textual é, antes, o apelo por
um novo modo de encarar a escrita: como um campo neutro, responsável
pela dissolução do sujeito, perda da identidade e destruição de toda a
voz.~Para argumentar em favor dessa hipótese, \index{Barthes, Roland}Barthes parte da citação
de um trecho da novela de \index{Balzac, Honoré de}Balzac, \emph{Sarrasine}, a respeito da qual
realizaria, pouco depois, em \emph{S/Z} (1970), um impressionante exercício de
dissecação crítica. Segundo a narrativa balzaquiana, em meados do séc.
\textsc{xviii}, Sarrasine, um escultor francês, apaixona"-se por La
Zambinella, uma cantora italiana, cuja beleza compara às obras da Grécia
antiga. Ao encontrar a \emph{prima donna}, Sarrasine descobre tratar"-se, na
verdade, de um \emph{castrato}. Do constrangimento inicial, Sarrasine sacraliza
sua musa, cuja beleza clássica e ideal atribui, justamente, ao fato de
se tratar de uma hermafrodita. Em ``A morte do autor'', \index{Barthes, Roland}Barthes parte da
seguinte passagem da novela de \index{Balzac, Honoré de}Balzac: ``Era a mulher, com seus medos
repentinos, seus caprichos sem razão, suas perturbações instintivas,
suas audácias sem causa, suas bravatas e sua deliciosa finura de
sentimentos''. O crítico dirige, então, uma série de perguntas retóricas
ao texto:

\begin{quote}
Quem fala assim? É o herói da novela, interessado em ignorar o castrado
que se esconde sob a mulher? É o indivíduo \index{Balzac, Honoré de}Balzac, dotado, por sua
experiência pessoal, de uma filosofia da mulher? É o autor \index{Balzac, Honoré de}Balzac,
professando ideias ``literárias'' sobre a mulher? É a sabedoria
universal? A psicologia romântica?\footnote{Roland Barthes, ``A morte do
  autor'', in \emph{O rumor da língua}, op. cit., p. 65.}
\end{quote}

Diante da constatação de que é impossível identificar o sujeito da
enunciação, \index{Barthes, Roland}Barthes conclui que ``a escritura é a destruição de toda a
voz''.

Retomemos, agora, o trecho citado de \index{Balzac, Honoré de}Balzac. Ali, qual é a imagem que se
produz da mulher?

Primeiramente, trata"-se não mais do que um lugar"-comum que a traduz como
ser eminentemente instintivo e sensível.~O tom empregado no trecho é
revelador de uma posição com relação a ela: há a enumeração de cinco
características definidoras (seus ``medos'', ``caprichos'',
``perturbações'', ``audácias'' e ``bravatas'' são considerados defeitos,
porque interpretados como manifestações injustificáveis, daí os
qualificativos ``repentinos'', ``sem razão'', ``instintivas'' e ``sem
causa''), ao passo que, como aparente traço positivo, oferece"-se apenas
a ``deliciosa finura de sentimentos''. Notemos: sucedendo a enumeração
negativa, o adjetivo ``deliciosa'' atua sobre o período como uma
antífrase, isto é, um elemento da retórica segundo o qual se emprega uma
palavra com sentido oposto ao pretendido. À luz do que vem antes, a
expressão ``finura dos sentimentos'' é evidentemente irônica.

No entanto, e segundo o próprio \index{Barthes, Roland}Barthes, se a fala no trecho citado é do
narrador, sua autoria é apenas suposta, já que é bem possível que ela
atue no campo linguístico do elocutor como uma presença estranha, uma
zona de influência da fala de outra personagem ou talvez do próprio
autor, entre outras possibilidades. Numa expressão, o narrador pode ter
adotado em sua fala uma perspectiva alheia.

Aventemos uma possibilidade de focalização: a de um uso bastante
consciente e crítico da expressão ``finura de sentimentos''.~Ela
revelaria uma zombaria do qualificativo normalmente atribuído à mulher.~A frase encerraria uma apropriação irônica desse lugar"-comum, posto que desdenharia dele ao mesmo tempo que o trataria como atributo.~Assim,
o início da passagem, ``era a mulher'', não significaria ``era
verdadeiramente uma mulher'', se ``verdadeiramente'' for entendido como
``uma mulher considerada integralmente, em sua profundidade
psicológica'', e sim algo similar a ``era como as demais mulheres'' --
que sofrem, portanto, a estigmatização do autor. Mas se pensarmos, por
exemplo, que a focalização se dirige para o herói da novela, segundo
outra hipótese do próprio \index{Barthes, Roland}Barthes, seu sentido seria completamente
diverso em relação a este.

Esse exercício de atribuições poderia se estender longamente, na medida
em que dá asas à imaginação a partir das sugestões colhidas de \index{Barthes, Roland}Barthes.
Mas o que ele revela é que a impossibilidade de fixar uma atribuição
autoral (afirmar ``é desse modo, e não daquele'') não elimina a
necessidade de definição de uma autoria para se compreender o sentido do
texto; bem ao contrário, a interpretação a reclama.

Quem fala assim?~Ora, o trecho conforma, de um modo ou de outro, uma
imagem autoral.~Sua definição está diretamente associada à compreensão
do texto. Cogitar que a referida fala represente a visão do próprio
autor é uma possibilidade legítima e constatável, sem que seja preciso,
para se chegar a essa conclusão, recorrer a nada que esteja fora do
espaço textual. Dizer algo do gênero ``\index{Balzac, Honoré de}Balzac quis dizer que\ldots{}'' é uma
atribuição que pode ser baseada em indícios textuais e que oferece um
vetor interpretativo para o texto. De modo análogo, para compreender o
artigo de \index{Barthes, Roland}Barthes, nós não deveríamos nos perguntar ``o que Barthes quis
dizer com isso?''~Essa é uma pergunta implícita nas hipóteses lançadas,
nos argumentos, nas deduções e nos exemplos, tal como acontece em ``A
morte do autor''.

O problema parece estar no alcance e na acepção que se confere ao autor
quando a pergunta é feita.~Ao indagarmos a respeito do que \index{Balzac, Honoré de}Balzac
pensava, isso é feito sempre sob a perspectiva do texto. \index{Balzac, Honoré de}Balzac não é o
indivíduo anterior ao texto, mas seu emissor, o sujeito que apenas
existe atrelado a \emph{Sarrasine}. Caso contrário, se o considerássemos como
anterior à novela, teríamos de supor absurdamente que \index{Balzac, Honoré de}Balzac é um
indivíduo que jamais mudaria de opinião, com plena convicção de suas
posições, e que \emph{Sarrasine} já era um produto acabado, tal como o
conhecemos, em sua mente. Há muitos Balzacs. Aquele ao qual dirigimos
uma pergunta é a imagem autoral produzida por \emph{Sarrasine}, o emissor de
suas falas, alguém que só existe -- e que só pode existir -- em conjunto
com elas.

Mediante essa ponderação -- de filiação, aliás, barthesiana --, podemos
afirmar que a interpretação do trecho citado é uma tentativa de
responder à seguinte pergunta: o autor procurou produzir sobre o leitor
determinado efeito ao denunciar uma visão irônica e machista do sexo
feminino ou será esta realmente a sua visão? Se for esta a sua visão, a
mulher é o alvo de sua crítica e o castrado tornou"-se, a seu ver, de
fato mulher. Mas a crítica à mulher abre a possibilidade de se criticar
o crítico. Sendo assim, não seria mais o sexo feminino o alvo, e sim o
sujeito que o julga de determinado modo. Consequentemente, o castrado
não seria mais que a reprodução do estereótipo feminino.

Estamos diante de leituras opostas (e haverá outras), uma direta, a
outra enviesada, numa espécie de dilema interpretativo. E essas duas
interpretações estão circunscritas na pergunta: qual a perspectiva
adotada por \index{Balzac, Honoré de}Balzac? Ou ainda: quem é o \index{Balzac, Honoré de}Balzac que fala?

Essa não é, certamente, uma curiosidade psicológica. Só é possível
compreender o trecho respondendo a essa pergunta.

Do contrário, se o autor está morto, o sentido está morto.

\section*{II. Quem é o autor?}
\addcontentsline{toc}{section}{Quem é o autor?}

Na acepção que acabou de ser empregada, descobrir o significado é o
mesmo que descobrir a intenção -- a intenção entendida como um projeto
de texto não premeditado, isto é, que vai se formulando e alterando
simultaneamente à atividade de escrita. Um projeto que se altera, na
medida em que o ser humano não pode permanecer o mesmo, indiferente à
passagem do tempo, à interferência das circunstâncias e à metamorfose
das ideias, que, afinal, é a metamorfose do mundo que o rodeia e o
constitui enquanto autor e leitor de seu próprio texto. Uma intenção,
portanto, que se ajusta à escrita tanto quanto a escrita se ajusta a
ela, porque se altera, conjuntamente, o grau de envolvimento do escritor
com sua criação, sob influência de tudo o que sente (e sentiu), imagina
(e imaginou), conhece (e conheceu), deseja (e desejou). A intenção sofre
a ação do texto na mesma medida em que atua sobre ele.

Recusar que a escrita seja uma prática guiada por um propósito, e que
esse propósito tenha um peso na sua recepção, é isolar o texto do mundo,
negar a literatura como fonte de conhecimento de si e do outro, de troca
de experiências e de aproximação das diferenças. Sem intencionalidade,
escritor e leitor são seres isolados de tudo aquilo que não diz respeito
a si mesmos. Isolados, inclusive, da linguagem. O texto, à luz dessas
considerações, é convertido em fetiche: objetual, vazio e distante.

Mas não é esta, a princípio, a acepção que \index{Barthes, Roland}Barthes confere ao autor, e o
que se discutiu até aqui é, em boa parte, aquilo que ele deixa de fazer.
É preciso, é claro, considerar sobre que contexto atua a sua tese, ou, em
outras palavras, que conotações carrega o conceito de autoria em seu
texto.

O crítico trata o autor como uma ``personagem moderna'', o que significa
que ele é um produto da cultura recente: ``ao sair da Idade Média, com o
empirismo inglês, o racionalismo francês e a fé pessoal da Reforma, ela
(a nossa sociedade) descobriu o prestígio do indivíduo ou, como se
costuma dizer mais nobremente, da `pessoa humana'''\footnote{Ibid, p.
  66.}. Essa formulação revela um posicionamento ideológico. Por um
lado, historicizar o autor significa relativizar seu papel na história
da cultura, enfraquecer seu teor de verdade e submetê-lo ao regime de
oscilação das circunstâncias. Por outro, lembremos que o termo
``moderno'' é tomado como sinônimo de ``positivista'', e este, por sua
vez, compreendido como sendo ``resumo e ponto de chegada da ideologia
capitalista''.

Eis os pomos da discórdia que se alojam nessa tese. Prestar atenção a
eles ajuda a esclarecer algo importante: ao rejeitar o autor, \index{Barthes, Roland}Barthes
ataca, numa expressão, a mentalidade burguesa moderna.

A partir desse ponto, para que se entenda o que representa a recente
tentativa de exclusão do autor do universo de sentido do texto, talvez
seja necessária uma pequena digressão para se considerar o que
representa o surgimento do autor do ponto de vista da escritura.

Lembremos que a figura do autor se fortalece à medida que a ideia
clássica de gênero se enfraquece na tradição literária. As noções antes
vigentes, de \emph{unidade de tom} e \emph{pureza estética}, reclamavam a separação
absoluta entre os diferentes tipos de texto, tal como realizado no Livro
\textsc{iii} da \emph{República}, de \index{Platão}Platão, mais empiricamente na \emph{Poética}, de
Aristóteles, e na \emph{Epistola ad Pisones}, de \index{Horácio}Horácio, que é o texto que
talvez mais tenha influenciado a poética e a retórica dos séculos
\textsc{xvi} ao \textsc{xvii}. Ali, o poeta formula a divisa que
sintetiza a concepção de gênero: ``singula quaeque locum teneant sortita
decentem'' (que cada assunto ocupe o seu devido lugar).

O enfraquecimento dessa mentalidade não é, em sua origem, um produto do
romantismo. Já no barroco espanhol, o surgimento de autores como \index{Lope de Vega y Carpio, Félix}Lope de
Vega e \index{Calderón de la Barca}Calderón de la Barca é simultâneo ao de um novo gênero, misto e,
portanto, ``impuro'' -- a tragicomédia. A substituição da \emph{estética do
gênero} pela \emph{estética do gênio}, por assim dizer, ocorre com mais nitidez
no século \textsc{xviii}. É esse o momento em que afloram os
lugares"-comuns românticos em torno da concepção de literatura, que passa
a ser progressivamente encarada menos como trabalho ou aplicação de uma
técnica, e mais como irrupção da interioridade do poeta (com o \emph{Sturm und
Drang}, na Alemanha).

Estamos no terreno mais que conhecido da historiografia literária, e
antes que o leitor se aborreça com seu caráter inevitavelmente
generalizante, atentemos para o que ele nos esclarece: se a ``verdade do
texto'' passa a ser identificada no indivíduo que o escreve, a concepção
de gênero é historicizada, o que significa dizer que ele se torna
passível de evolução, e com isso nega"-se seu caráter supostamente
estático e normativo.

É nesse sentido que, no prefácio à \emph{Estética}, de \index{Hegel, Georg Wilhelm Friedrich}Hegel, \index{Lukács, György}Lukács lê os gêneros como determinados pelas necessidades das sociedades. Já na
\emph{Correspondência}, entre \index{Goethe, Johann Wolfgang von}Goethe e \index{Schiller, Friedrich}Schiller, acompanha"-se a transição da
ideia de pureza para a de hibridismo. De resto, esse conceito é a pedra
de toque do famoso prefácio de \index{Hugo, Victor}Victor Hugo, ``Do grotesco e do
sublime'', que anuncia outra forma híbrida, o drama, como amálgama entre
opostos.

Essa síntese, necessariamente esquemática, é suficiente para atender a
um propósito específico à presente argumentação, que é o de afirmar que
a ênfase depositada na figura do autor na tradição literária ocidental
não tem a mesma conotação que lhe é atribuída por \index{Barthes, Roland}Barthes.

Se este prefere entender o autor como limitador do sentido do texto, a
recapitulação de seu contexto de origem põe em evidência justamente o
contrário dessa concepção: a presença de um novo personagem, o escritor
criativo, como parte integrante do universo textual, aponta para a
libertação da literatura de suas amarras prescritivas, atuando como um
dos vetores da individualidade e autonomia da obra literária.

O alvo de \index{Barthes, Roland}Barthes está, portanto, velado. Talvez não seja o autor, mas
sim a imagem que metaforiza uma outra noção, de origem romântica e
relacionada ao autor: a de literatura como confissão, como expressão de
um eu. Ela não está, evidentemente, desvinculada do autor, mas tem a
especificidade de conferir a ele o poder de explicar o texto, de
tratá-lo como o esteio do sentido de tudo o que escreve, reduzindo a
escrita a um meio diáfano e transparente de registrar um conteúdo
pré-formado.

Para refletir sobre essa hipótese, voltemos ao texto de \index{Barthes, Roland}Barthes. Ali ele
explicita em que circunstâncias costuma aparecer essa personagem:

\begin{quote}
O autor reina ainda nos manuais de história literária, nas biografias de
escritores, nas entrevistas dos periódicos, e na própria consciência dos
literatos, ciosos por juntar, graças ao seu diário íntimo, a pessoa e a
obra; a imagem da literatura que se pode encontrar na cultura corrente
está tiranicamente centralizada no autor, sua pessoa, sua história, seus
gostos, suas paixões; a crítica consiste ainda, o mais das vezes, em
dizer que a obra de Baudelaire é o fracasso do homem Baudelaire, a de
van Gogh é a loucura, a de Tchaikovski é o seu vício: a explicação da
obra é sempre buscada do lado de quem a produziu, como se, através da
alegoria mais ou menos transparente da ficção, fosse sempre afinal a voz
de uma só e mesma pessoa, o autor, a entregar a sua
``confidência''.\footnote{Ibid., p. 66.}
\end{quote}

O trecho é muito claro. Por essa passagem, \index{Barthes, Roland}Barthes identifica como autor
o indivíduo que escreve, isto é, o eu biográfico, com data de
nascimento, características físicas e hábitos mundanos.~Trata"-se,
primeiramente, de considerá-lo em sua acepção mais restrita, portanto.
Essa figura de carne e osso deve representar algo de revoltante para o
crítico, mas talvez não seja odiosa em si mesma, porque com a expressão
``a morte do autor'' \index{Barthes, Roland}Barthes certamente não sugeria o extermínio do
escritor criativo. Sua insatisfação é com os procedimentos críticos de
seu tempo, responsáveis por desviar a atenção destinada ao texto
literário para o sujeito que o produziu, e sua crítica se dirige, por
consequência, a uma prática: interpretar segundo as declarações, a
biografia e a psicologia do escritor.

Essa prática implica recuperar no texto a experiência vivida (nas
biografias, diários íntimos e histórias literárias), ou ajustar o texto
às declarações daquele que o escreveu (nas entrevistas e nos documentos
revirados no espólio de um escritor).~De um modo ou de outro,
interpretar \emph{segundo o autor} significa, pois, pressupor a coerência do eu
individual e buscar a unidade de sentido do texto numa figura ao mesmo
tempo anterior (na suposta premeditação consciente ou inconsciente do
texto) e posterior a ele (nas declarações ulteriores a seu respeito).

O trecho de \index{Barthes, Roland}Barthes retoma, na verdade, uma discussão um pouco mais
antiga, travada com grande propriedade por \index{Eliot, Thomas Stearns}T. S. Eliot, cuja filosofia
crítica, a exemplo de sua poesia, é fundada na concepção de que a
natureza humana é ``impura'' e finita. Conforme salienta \index{George, Arapura Ghevarghese}A. G. George,
em ``Eliot as literary critic'', seu antecessor principal é \index{Hulme, Thomas Ernest}Hulme, que
repudiava a ideia rousseauniana de que a essência do homem é boa, o
``nobre selvagem'', antes defendendo a necessidade da disciplina e da
impessoalidade. Se a personalidade humana é impura, a sinceridade e a
clareza com que ela é ``expressa'' não podem constituir critério para se
julgar o texto. Daí \index{Eliot, Thomas Stearns}Eliot apontar para a necessidade de se estabelecerem
padrões externos que possam servir como critério de avaliação para a
literatura. Para este crítico, a defesa da autonomia estética da
literatura entrava, contudo, em conflito com sua religiosidade crescente
e sua consciência social.\footnote{Cf. o cap. \textsc{viii} de A. G.
  George, \emph{T. S. Eliot} -- his mind and art, Londres, Asia
  Publishing House, s/d.}

Explicar é o contrário de interpretar. Se a interpretação atua sobre
possibilidades, a explicação presume que só pode haver um sentido, e que
este sentido é passível de ser revelado. A interpretação é mutuamente
válida: eu interpreto de acordo com algumas justificativas que não
excluem uma segunda interpretação, construída com base em outras
justificativas. Já a explicação não permite a variedade. Ela é um ponto
final em qualquer discussão. A interpretação não assume que possa haver
uma verdade imutável, porque trata o texto como um organismo maleável,
que se modifica de acordo com o momento histórico de sua recepção e com
o próprio leitor, e carrega em si a ideia de que as verdades são ilusões
criadas. Já a explicação se move pela satisfação na crença em uma
verdade definitiva, pelo mistério ainda maior de uma causalidade única,
fria e indiferente à diversidade humana.\footnote{Cf. Nietzsche,
  \emph{Sobre verdade e mentira} (organização e tradução de Fernando de
  Moraes Barros), São Paulo, Hedra, 2007.}

Explicar um texto \emph{segundo seu autor} implica conceber a intenção como uma
deliberação premeditada à escrita. A crítica a essa prática, já muito
conhecida, e ao menos três décadas mais antiga do que o manifesto de
\index{Barthes, Roland}Barthes (lembremos os ensaios de \index{Eliot, Thomas Stearns}Eliot, a crítica de \index{Jung, Carl Gustav}Jung a \index{Freud, Sigmund}Freud e a ``falácia intencional'' de \index{Beardsley, Monroe}Beardsley e Wimsatt), parte da ideia"-base de
que, se fosse possível explicar um texto segundo a intenção de seu
autor, teríamos que concordar com a possibilidade de as palavras
traduzirem literalmente nossas ideias, e, mais absurdamente, com a
hipótese de o escritor ter pensado todo o texto antes de começar a
escrevê-lo. Encarada dessa forma, a escrita seria uma consubstanciação,
isto é, o mesmo que transportar ideias para o papel, sem que outras
ideias (e mesmo outras palavras) interferissem naquilo que foi
previamente elaborado mentalmente. Desse modo, para se compreender a
intencionalidade como premeditação, requer"-se que se adote uma concepção
tão somente mecânica da escrita.

Estabelecida essa distinção, \index{Barthes, Roland}Barthes supostamente defenderia a
interpretação no lugar da explicação. Mas não é isso o que ocorre, e ``a
morte do autor'' terá implicações mais radicais que esta.

\section*{III. De que é feita a escritura?}
\addcontentsline{toc}{section}{De que é feita a escritura?}

Da impossibilidade de ancorar o sentido do texto num porto seguro,
estável e definível como é o autor, decorre uma noção bastante flexível
de interpretação, que \index{Barthes, Roland}Barthes distingue da \emph{decifração}, concebendo"-a
simplesmente como \emph{leitura possível}. Em outros termos, a atividade
crítica deixa de visar a uma verdade excludente das demais e se
apresenta como inclusiva, mutuamente verossímil. Segundo \index{Barthes, Roland}Barthes, o
texto pode ser ``desfiado'', nunca ``decifrado'', porque ``não há
fundo'' que se revele.

Nesse universo teórico, a questão que se coloca não é mais conceitual do
que de nomenclatura. Reformular nossos modos habituais de compreender o
mundo, bem como nossas práticas de leitura e ensino, significa, nesse
debate, reformular o vocabulário que empregamos.~O exemplo maior disso é
a conversão do conceito de \emph{literatura}, matizado pelos séculos de
tradição retórica, em \emph{escritura}, concebida como prática que ``propõe
sentido sem parar'', mas com o propósito de ``evaporá-lo: ela procede a
uma isenção sistemática do sentido''.\footnote{Roland Barthes. ``A morte
  do autor'', op. cit., p. 69.} Sendo assim, no universo barthesiano, a
suposição de um sentido principal é considerada como autoritária.

Haver muitos sentidos significa não haver um ``sentido último''. Esse
modo de encarar o texto destitui o crítico de seu trono de leitor
profissional, porque desierarquiza sua interpretação ao considerá-la
mais uma leitura. \index{Barthes, Roland}Barthes continua a rejeitar as formas de poder, a
denunciar os donos do texto, a que se refere através do emprego irônico
de maiúsculas alegorizantes: ``não é de se admirar, portanto, que o
reinado do Autor tenha sido o reinado do Crítico, nem tampouco que a
crítica (mesmo a nova) esteja hoje abalada ao mesmo tempo que o
Autor.''\footnote{Ibid., p. 69.}

Desalojada de um autor -- isto é, de um \emph{passado} e de um \emph{fundo} --, a
``escritura'' se constituirá como um espaço de ``dimensões múltiplas,
onde se casam e se contestam escrituras variadas, das quais nenhuma é
original: o texto é um tecido de citações, saídas dos mil focos da
cultura''. Essa passagem, em \index{Barthes, Roland}Barthes, é seminal para a teoria da
\emph{intertextualidade} -- expressão cunhada por \index{Kristeva, Julia}Julia Kristeva, em 1966, e
sistematizada como teoria em \emph{Semiótica}, de 1969. Ali, a noção de
intertextualidade substitui a de intersubjetividade, isto é, como forma
de designar o processo de produção do texto literário para além da
expressão de um eu. O texto se constitui, segundo a semióloga, da
absorção e transformação de outros textos. Toda a escrita será,
portanto, e inevitavelmente, reescrita. Nesse campo de atuação, entram
em jogo, de modo decisivo para se pensar a evolução da tradição
literária, noções como as de \emph{paródia}, \emph{paráfrase}, \emph{citação}, \emph{colagem},
\emph{alusão} e \emph{apropriação}, pensadas sempre como rupturas e desvios.

À luz dessas considerações, o ``escriptor'', segundo \index{Barthes, Roland}Barthes,

\begin{quote}
(\ldots{}) não possui mais em si paixões, humores, sentimentos, impressões,
mas esse imenso dicionário de onde retira uma escritura que não pode ter
parada: a vida nunca faz outra coisa senão imitar o livro, e esse mesmo
livro não é mais que um tecido de signos, imitação perdida,
infinitamente recuada.\footnote{Ibid., p. 69.}
\end{quote}

A concepção de que os textos dialogam entre si e de que a escrita é um
arranjo de vozes não é, evidentemente, uma invenção de \index{Barthes, Roland}Barthes, tampouco
de \index{Kristeva, Julia}Kristeva.~Ela existe desde Homero, e a prática da emulação entre
autores confunde"-se com a própria prática da escrita. Modernamente, já nas primeiras décadas
do século \textsc{xx} ao menos três teóricos russos trataram do tema com
profundidade: \index{Eikhenbaum, Boris}Eikhenbaum (``Sobre a teoria da prosa'', 1925) e \index{Tynianov, Yuri}Tynianov
(``Da evolução literária'', 1927), vinculados ao chamado formalismo
russo, e, principalmente, \index{Bakhtin, Mikhail}Bakhtin, cujas teorias a respeito do
dialogismo do discurso literário e da polifonia no romance (\emph{A poética de
Dostoiévski}, 1929, e \emph{Questões de literatura e estética}, 1975) conferem o
escopo teórico mais denso sobre o tema. Se avaliarmos ``A morte do
autor'' com base em seus antecedentes, encontraremos uma gama tão
variada de autores relevantes e de estudos profundos, que seremos
levados a concluir que ``a morte do autor'' é, sobretudo, um panfleto.

Um panfleto em favor de um outro polo constituinte do fenômeno
literário.~E que aponta, surpreendentemente, para a construção de um outro \emph{império}.

\section*{IV. O império do leitor}
\addcontentsline{toc}{section}{O império do leitor}

Como já adiantado, o tom do texto de 1968 é o de manifesto. A lógica do
discurso barthesiano não é estritamente dedutiva -- uma proposição não
leva necessariamente a outra. \index{Barthes, Roland}Barthes trabalha, melhor dizendo, com um
jogo de compensações: ``(\ldots{}) o autor entra na sua própria morte, a
escritura começa''. O desligamento do autor é compreendido, nesses
termos, como condição necessária para a escritura. Ele afirma, por
exemplo, que ``nas sociedades etnográficas, a narrativa nunca é assumida
por uma pessoa, mas por um mediador, xamã ou recitante, de quem, a
rigor, se pode admirar a performance (isto é, o domínio do código
narrativo), mas nunca o `gênio'''. Seria possível, diante dessa
remissão, contra"-argumentar dizendo simplesmente que a nossa sociedade não
é mais etnográfica, e que escritores não são xamãs, ou então que essa
contextualização não é mais verdadeira e aplicável do que as demais,
como com relação à sociedade romântica, na qual a narrativa é
assumida por um único indivíduo, ``homem de gênio'', cuja obra é
admirada em correlação estreita com sua psicologia e experiências. Mas
esse caminho seria adequado se \index{Barthes, Roland}Barthes não tivesse revelado, com a tese
da ``morte do autor'', o real propósito de instaurar um outro império:
``o nascimento do leitor deve pagar"-se com a morte do autor''.\footnote{Ibid.,
  p. 70.}

Essa é a frase axiomática que encerra seu texto. Trata"-se de um apelo,
que merece atenção mais detida, porque, se atentarmos bem, ele significa
que, embora critique o ``império do Autor'', \index{Barthes, Roland}Barthes não trata a
``escritura'' como fenômeno integralmente autônomo.

Se ``o verdadeiro lugar da escritura'' é a ``leitura'', quem é o leitor?

Essa pergunta identifica na estratégia argumentativa de \index{Barthes, Roland}Barthes seu foco
tendencioso. Isso porque, se o autor era antes tomado em sua acepção
mais restrita -- o escritor, o indivíduo --, o leitor é concebido de
modo bastante diferente. Primeiramente, ele não é encarado como um
indivíduo: ``o leitor é um homem sem história, sem biografia, sem
psicologia''. Ele representa, na verdade, um espaço: ``onde se
inscrevem, sem que nenhuma se perca, todas as citações de que é feita
uma escritura''. O leitor é concebido como aquele que confere unidade ao
texto: ``a unidade do texto não está em sua origem, mas no seu
destino''. Um destino impessoal, portanto. Mas não seria possível usar
esse mesmo recurso para idealizar o autor?

Em ``Escrever a leitura'' (1970), \index{Barthes, Roland}Barthes expõe seus propósitos em \emph{S/Z},
a análise de Sarrasine, afirmando que em vez de falar de \index{Balzac, Honoré de}Balzac e do seu
tempo, da psicologia de suas personagens, da temática do texto ou da
sociologia do enredo, preferiu ``ler levantando a cabeça''\footnote{Roland
  Barthes, ``Escrever a leitura'' [1970], in \emph{O rumor da
  língua}, op. cit., p.40.}, isto é, sistematizando todas as digressões,
os momentos que interrompeu a leitura para interrogá-la. \emph{S/Z} é a
demonstração prática do que \index{Barthes, Roland}Barthes chama de ``texto"-leitura'',
expressão que substitui o termo ``crítica'', e que pretende retirar o
privilégio ao lugar de onde supostamente parte a obra, pessoal ou
histórico, e redirecioná-lo para seu destino, o leitor. Assim, ao invés
de se perguntar ``o que o autor quis dizer'', \index{Barthes, Roland}Barthes pergunta ``o que o
leitor entende''.\footnote{Ibid., p. 41.} Mas nota"-se, novamente, em
contraste com a descrição da figura autoral, o tratamento idealizador
conferido ao leitor: ``Não reconstituí o leitor (fosse eu ou você), mas
a leitura. Quero dizer que toda a leitura deriva de formas
transindividuais''.\footnote{Ibid., p. 42. Um tratamento
  ``transindividual'' conferido ao autor se verifica em \index{Foucault, Michel}Foucault, em ``O
  que é um autor'' e em \emph{A ordem do discurso}, textos de que
  tratarei noutra ocasião.}

Diante de um postulado a tal ponto dogmático, refaçamos a pergunta: se
concordarmos em compreender como equivocado o espaço antes conferido ao
autor como autoridade sobre o texto, será legítimo assumir a censura ao
autor como um substituto da antiga censura ao leitor? Ou ainda: por que
tratar o leitor como tutor do sentido do texto seria mais legítimo do
que atribuir esse papel ao autor?

Para essa pergunta, é inevitável considerar uma resposta muito simples,
mas grávida de decorrências: porque os leitores são muitos, ao passo que
o autor é um só. Os leitores, sendo em maior número e diferentes entre
si, tornam tudo ao mesmo tempo possível e impossível. Se tudo pode ser
comunicado numa frase, o que de fato ela comunica?

O império do leitor não é outra coisa senão uma tentativa inicial de
dissolução do sentido e, com ele, do status do texto.~Ainda em
``Escrever a leitura'', lemos: ``a composição canaliza; a leitura, pelo
contrário (esse texto que escrevemos em nós quando lemos), dispersa,
dissemina''. O autor, ou a ``lógica da razão'', é substituído pelo
leitor, ou a ``lógica do símbolo''.\footnote{Ibid., p. 41.} A concepção
de que o ``texto"-leitura'' deve reestabelecer a ``verdade lúdica'' da
``escritura'' é o substituto, no vernáculo barthesiano, à concepção de
que a crítica deve estabelecer a verdade objetiva/subjetiva da
literatura. Está aberto, assim, o curso dos questionamentos para que
envereda a crítica pós"-estruturalista. E essa constatação nos encaminha
para um estágio importante nessa discussão.

\section*{V. Pela rota mallarmaica}
\addcontentsline{toc}{section}{Pela rota mallarmaica}

A concepção de que a linguagem poética é essencialmente não"-comunicativa
é de origem simbolista e atraiu para o poema uma aura mística, que
encontrou amplo respaldo na literatura francesa da segunda metade do
século \textsc{xix}. No entanto, não se pode generalizá-la. Em ``The
Social Function of Poetry'' (1945), \index{Eliot, Thomas Stearns}T. S. Eliot -- conhecido por advogar
em favor da despersonalização do escritor no texto e por antecipar a
visão dos \emph{new critics}, sobretudo no que diz respeito às
falaciosas projeções psicobiográfias de críticos como \index{Sainte-Beuve, Charles Augustin}Sainte"-Beuve --
não abre mão do propósito comunicativo para tratar da lírica. O título
do livro de que consta este ensaio, \emph{Sobre poesia e poetas}, é
autoexplicativo nesse sentido:

\begin{quote}
Além de qualquer intenção específica que a poesia possa ter (\ldots{}) há
sempre a comunicação de alguma nova experiência, ou uma inédita
compreensão do familiar, ou a expressão de algo que vivenciamos mas para
o qual não temos palavras, o que alarga nossa consciência ou refina
nossa sensibilidade.\footnote{T. S. Eliot, ``The Social Function of
  Poetry'', in \emph{On poetry and poets}, Londres/Boston, Faber and
  Faber, 1984, p. 18. {[}Tradução minha.{]}.}
\end{quote}

Em contraste com a visão eliotiana de que a poesia lida sempre com a
comunicação de alguma experiência nova, ou proporciona um novo
entendimento do que já é familiar, a concepção de escritura resulta, em
\index{Barthes, Roland}Barthes, num texto que não pode representar nada que esteja antes dele.
Essa forma de encarar o texto, como realidade ontológica, não é estranha
à sua base de enunciação linguística -- a concepção algo misteriosa, mas
muito difundida, de ``função poética da linguagem'', proposta em 1960
pelo linguista russo \index{Jakobson, Roman}Roman Jakobson. A linguagem poética não narra, não
expressa, porque ela é um fim em si mesma. O que \index{Jakobson, Roman}Jakobson afirma é que a
``função da linguagem poética projeta o princípio de equivalência do
eixo de seleção sobre o eixo de combinação''\footnote{\index{Jakobson, Roman}Roman Jakobson,
  ``Linguística e poética'' {[}1960{]}, in \emph{Linguística e
  comunicação} (tradução de Izidoro Blikstein e José Paulo Paes),
  5\textsuperscript{ª} ed., São Paulo, Cultrix, 1971, p. 130.} da
linguagem. Mas \index{Barthes, Roland}Barthes radicaliza essa perspectiva, por defender que a
``verdade do texto'' é o próprio texto, não no sentido de que ele possa
comunicar algo, mas de que ele, enquanto escritura, só pode ser
autorreferencial. \index{Jakobson, Roman}Jakobson não foi tão longe: ``A adaptação dos meios
poéticos a algum propósito heterogêneo não lhes esconde a essência
primeira, assim como os elementos da linguagem emotiva, quando
utilizados em poesia, conservam ainda sua nuança emotiva''\footnote{Ibid.,
  p. 131.}.

Essa diferença se deve a um viés específico de leitura. O modo como
\index{Barthes, Roland}Barthes lê \index{Jakobson, Roman}Jakobson, e com ele a \emph{nouvelle critique} e o grupo \emph{Tel quel},
tem em \index{Mallarmé, Stéphane}Mallarmé um filtro radical, que singulariza a linguagem poética
como objeto anticomunicativo. ``No fundo, o mundo foi feito para acabar
num belo livro'', dizia o poeta de ``Le livre''. Em um dos fragmentos de
\emph{Igitur}, lemos ainda: ``Profiro a palavra, para afundá-la de novo em sua
inanidade''.\footnote{Mallarmé, \emph{Poemas} (organização e tradução de
  José Lino Grünewald), Rio de Janeiro, Nova Fronteira, 1990, p. 115.} O
caminho de desmistificação do autor prosseguia, pela rota mallarmaica,
para a mistificação da linguagem. Estava aí o novo índice de
literariedade do texto. Em vez de pensar os mecanismos próprios de uma
função poética da linguagem, a crítica de \index{Barthes, Roland}Barthes cedia à recusa não só
da explicação, mas também da interpretação.

A esse respeito, recentemente o ensaísta italiano \index{Berardinelli, Alfonso}Alfonso Berardinelli
(um intelectual nada ``moderno'', diga"-se de passagem, na medida em que
julga que a literatura não só tem um sentido, como confere sentido à
experiência) tomou partido de modo bastante incisivo nessa questão,
denunciando a influência e a hegemonia pós"-estruturalista enquanto base
de teorização do texto literário. Segundo ele, a rota traçada através de
\index{Mallarmé, Stéphane}Mallarmé teria levado a poesia para um ``caminho de depuração
anticomunicativa, progressivamente se enfraquecendo e esvaziando''. As
cores com que \index{Berardinelli, Alfonso}Berardinelli pinta o cenário da nova poesia são pálidas:
``a maior parte dos jovens autores que começaram a publicar a partir dos
anos 1970 não ultrapassaram os limites e o âmbito restrito fixados pela
estética formalista e pelas vanguardas informais, segundo as quais tudo
era possível em poesia, tudo era permitido, exceto dizer alguma
coisa''.\footnote{Alfonso Berardinelli, ``As fronteiras da poesia''
  {[}1993{]}, in \emph{Da poesia à prosa} (organização de Maria Betânia
  Amoroso; tradução de Maurício Santana Dias), São Paulo, Cosac Naify,
  2007, p.16.}

A leitura teria deixado de ser o compartilhamento de uma experiência
nova, uma vez que, simultaneamente às tendências poéticas, \index{Berardinelli, Alfonso}Berardinelli
constata um danoso movimento de abstração do texto por parte dos
críticos. Ele considera que enquanto a crítica literária, em pouco
tempo, ocupou"-se, não da literatura, mas da \emph{ideia de literatura}, sua
noção de linguagem traduziu"-se como hermetismo, um todo fechado em si
mesmo, estranho a quaisquer valores semânticos: ``As fronteiras da
Literatura, entendida como máquina textual que devora a si mesma,
dilatavam"-se enormemente, impedindo que a ideia e a essência literária
entrassem de fato em atrito com algo de diferente e de
estranho''.\footnote{Ibid., p. 16.} O texto torna"-se, então, um código
restrito a um círculo diminuto de escritores, o que, num outro ensaio,
\index{Berardinelli, Alfonso}Berardinelli chama de ``antimundo''.\footnote{Ibid., ``Quatro tipos de
  obscuridade'' [1991], op. cit., p.141.} Com a literatura ascética
e afastada de tudo, a crítica a seu respeito, que havia estigmatizado o
fundo semântico da ``escritura'', converte"-se numa fala verborrágica e
conjectural sobre o silêncio -- sua forma mais depurada de
autodestruição. Ambas, narcisicamente, formulando jargões inovadores a
respeito de si mesmas.

Desmistificava"-se o autor, mistificava"-se a linguagem.

Com isso, fazia"-se uma negação generalizada: do sujeito, do tema (o
objeto), da sociedade, da história; de tudo que não fosse a afirmação de
um vazio, a própria linguagem.

A leitura de \index{Berardinelli, Alfonso}Berardinelli, da década de 1990, é afim à perspectiva de
Hayden White sobre a crítica contemporânea a um ensaio seu, de 1976, que
falava num ``momento absurdista'' na moderna crítica literária. Para o
estudioso americano, a crítica não apenas não mais teria uma ideia
segura do que representava a literatura, como não saberia traçar a linha
que separa a literatura da linguagem: ``nada é interpretável como
fenômeno especificamente literário, a literatura como tal não existe, e
a tarefa da moderna crítica literária (se a questão for levada às
últimas consequências) é comandar sua própria dissolução''.\footnote{Hayden
  White, ``O momento absurdista na teoria literária contemporânea'', in
  \emph{Trópicos do discurso} -- ensaios sobre a crítica da cultura
  (tradução de Alípio Correia de Franca Neto), São Paulo, Edusp, 1994,
  p. 286.}

É o caso, então, de voltarmos às bases dessa discussão.

Num ensaio esclarecedor sob vários aspectos, ``The Frontiers of
Criticism'' (1956), \index{Eliot, Thomas Stearns}T. S. Eliot chamou a atenção para dois perigos do
método explicativo ou psicobiográfico: 1) pressupor que há apenas uma
interpretação correta, sem perceber que o poema pode ter significados
diferentes para leitores de diferentes sensibilidades; 2) pressupor que
sendo válida a interpretação dum poema, ela passa a ser aquilo que seu
autor tentara transmitir consciente ou inconscientemente.~A tese da
``morte do autor'' não desconsiderou esses riscos, pelo contrário: ela
extrapolou seus limites, e, ao se opor em demasia à ótica individualista
e histórica das correntes de explicação do texto, acabou por se
equiparar à política de certos governantes, que, sob o pretexto de
eliminar a pobreza, mais ou menos voluntariamente expulsaram os pobres
de suas vistas.

\section*{VI. Autoria e autoridade: fora~os~rinocerontes!}
\addcontentsline{toc}{section}{Autoria e autoridade: fora os rinocerontes!
\medskip}

O manifesto de \index{Barthes, Roland}Barthes é escrito num momento excepcional, em que Paris é
o epicentro de explosões de radicalismo estudantil assistidas em todo o
mundo.~Cinquenta anos depois, não é tarefa simples mensurar o estado de
exaltação em que se vivia naquele período, porque a realidade que
conhecemos é em muitos aspectos diametralmente oposta àquela. Basta
dizer que hoje a tirania do mercado, associada à falta de oportunidade
de trabalho e ao individualismo, levou boa parte dos jovens a abandonar
o discurso revoltado e a lutar justamente pela inserção nesse sistema. A
imagem do profissional ``bem"-sucedido'' está andares acima em sua escala
de valores daquela outra, contestadora, ora malvista como romantismo
ultrapassado, sem lenço e sem documento, ora digerida pelo próprio
mercado, que a devolve em formas mais domesticadas, como a imagem de
``Che'' nas camisetas, pôsteres e broches.

Já em 1972, \index{Pasolini, Pier Paolo}Pasolini aludia ao retrocesso dos jovens de sua geração com
relação à dos anos 1960. Para ele, os cabelos compridos, então signos
não"-verbais que, embora nascidos no seio da burguesia, exprimiam o enjoo
da civilização de consumo, haviam se tornado ``coisas de televisão ou
dos anúncios publicitários''. Eram, desse modo, signos que ressuscitavam
o conformismo servil e o convencionalismo vulgar que seus pais haviam
momentaneamente superado.\footnote{Pier Paolo Pasolini, ``O discurso dos
  cabelos'', in \emph{Os jovens infelizes} -- antologia de ensaios
  corsários (tradução de Michel Lahud e Maria Betânia Amoroso), São
  Paulo, Brasiliense, 1990, pp. 37-44.}

\index{Pasolini, Pier Paolo}Pasolini, aliás, percebeu, tão logo começou a se agravar, o processo de
``coisificação'' e ``desidentificação'' social originado no capitalismo
monopolista, e que em termos bem atuais é designado por
``globalização''. A uniformização dos gostos, hábitos, ideias -- ou da
falta delas --, trajes, léxico, valores etc., deve"-se, em suma, à
vontade de aburguesamento social que teria feito da pobreza algo
repugnante aos nossos olhos. Passa a ser aceita a música dos negros, mas
não os negros: ``Em 1961, os burgueses viam no subproletariado o mal,
exatamente como os racistas americanos o viam no universo
negro''.\footnote{Idem, ``Meu `Accattone' na TV após o genocídio'', op.
  cit., p. 138.}

Como descreve \index{Hobsbawm, Eric}Hobsbawm, o espantoso juvenescimento da sociedade
assistido após a década de 1950 significa a afirmação absoluta do
capitalismo. O jovem, quer pelo rock, pelo jeans, pelo cinema
hollywoodiano (James Dean), pelas drogas, pela liberalização sexual,
pela tecnologia mutante e acelerada, ou pelo culto do corpo e pela
valorização dos esportes, passa a ser, depois da Segunda Guerra, o
símbolo máximo das economias de mercado desenvolvidas. A cultura jovem
torna"-se a matriz da revolução cultural do século \textsc{xx} e acentua
o que \index{Hobsbawm, Eric}Hobsbawm chama de ``abismo histórico'' entre os nascidos antes de
1925 e depois de 1950.\footnote{Cf. Eric Hobsbawm, ``Revolução
  cultural'', in \emph{Era dos Extremos} -- o breve século \textsc{xx}
  (1914-1991) (tradução de Marcos Santarrita), 2\textsuperscript{a}
  ed., São Paulo, Companhia das Letras, 1994, pp. 314-336.}

Hoje, pautado no julgamento das aparências e formado num sistema
educacional pragmatista, esse jovem ``bem informado'', cujas ansiedades
são anestesiadas, quando não confundidas, pelo consumo,\footnote{A esse
  respeito, \index{Pasolini, Pier Paolo}Pasolini fala em um novo fascismo, o ``fascismo de
  consumo''.} não duvida de nada; prefere, em síntese, a adaptação à
contestação. Os \emph{mass media}, ou o ``hedonismo do poder consumista'', nas
palavras de Pasolinni, tornaram"-se a ideologia do poder no mundo jovem.

Isso para dizer que a escrita de ``A morte do autor'', realizada
justamente em Paris, por um intelectual de espírito jovem (já na casa
dos cinquenta anos), libertário e contestador, comprometido com as
mudanças e os discursos de seu tempo, não pode ser bem compreendida se
inteiramente desvinculada de seu momento de produção, caracterizado por
uma embriaguez política e cultural cujo raio de disseminação abrange
todo o Ocidente.

Entre fevereiro e maio de 1968 eclodem os movimentos anti"-stalinistas em
Varsóvia e Praga, bem como as revoltas estudantis em Bonn, na Alemanha,
assiste"-se à tomada da Universidade de Nanterre, na França, liderada por
\index{Cohn-Bendit, Daniel}Daniel Cohn"-Bendit, à invasão da Universidade de Brasília pelos
estudantes, e aos protestos em todo o Brasil pela morte do estudante
\index{Souto, Edson Luiz de Lima}Edson Luiz de Lima Souto, em confronto com a polícia durante a invasão
do restaurante universitário Calabouço, no Rio de Janeiro.~Nos
\textsc{eua}, inicia"-se uma onda de conflitos raciais após o assassinato
de \index{King, Martin Luther}Martin Luther King, ocorre a ocupação da Universidade Columbia e o
protesto contra a guerra do Vietnã no Central Park, em Nova Iorque. Em
Paris, levantam"-se barricadas de até três metros de altura nas ruas do
Quartier Latin e coloca"-se em marcha uma sequência de greves operárias
que culminam na paralisação, no dia 25 de maio, de nada menos que 10
milhões de estudantes e funcionários públicos, sendo considerada a maior
da história.

As lutas de 68 foram sempre antiautoritárias. E essa reação contra o
poder se estendeu do âmbito político ao familiar, educacional e
comportamental.~Mas o que as singulariza na França, e torna sua
compreensão uma tarefa menos óbvia do que pode aparentar ser, é que o
movimento não se resumia, simplesmente, a uma reivindicação financeira
dos operários, ou à discórdia política dos estudantes e intelectuais. Há
um certo infantilismo que compõe o movimento e o torna inaudito desde a
origem: um grupo de estudantes que rejeitavam a disciplina, as regras e
as formalidades da vida no campus. E esse espírito muito mais lúdico do
que propriamente revolucionário se manteve. A rebeldia, distante do
drama do esfomeado ou do desespero aflitivo do desempregado, tal como
assistimos hoje no terceiro mundo, era mais propriamente de um colorido
niilista de quem não pretendia tomar o poder, mas contestar a existência
de qualquer instância que pudesse exercê-lo. Tal contestação poderia se
sintetizar na permissão, por exemplo, de as mulheres usarem calças
compridas. Sem um projeto político real, sem uma ditadura para combater,
e com as barrigas cheias, mais propriamente do que realizar uma
revolução, o que a juventude parisiense buscava era um discurso de
revolta.

Maio de 68 não triunfou em todos os planos, e o que se assistiu na
década de 1970 foi a um endurecimento dos regimes políticos, a tomada de
Praga pelos tanques soviéticos, a eleição de Nixon nos \textsc{eua}, o
\textsc{ai"-5} no Brasil e o revigoramento do poder de \index{de Gaulle, Charles}de Gaulle na
França. Mas seu efeito sobre os costumes e a cultura foi bastante
visível. \index{Barthes, Roland}Barthes se referiu ao período como ``a Libertação'' --
libertação esta que está longe de ter passado em branco em seu manifesto
sobre o destino do autor.

Essencialmente, o final da década de 1960 na França é marcado por uma
transição geracional bastante nítida, ligada ao sensacional progresso da
alfabetização e à extraordinária corrida para a Universidade, que
aprofundaram enormemente as diferenças entre pais e filhos, tanto no que
se refere à visão de mundo quanto aos hábitos e valores adotados. O
jovem universitário não representava mais um pequeno percentual da
população em mera transição para as obrigações adultas, mas uma classe
composta por uma massa de estudantes ainda sem espaço definido na
sociedade.~O abismo geracional se agrava se considerarmos a diferença de
mentalidade entre aqueles que, tendo enfrentado a Segunda Guerra
Mundial, presenciavam um momento incomparavelmente mais estável e com
ares de prosperidade nos anos 1960, e a geração pós"-guerra, ávida por
encontrar sentidos na eleição de inimigos. O ressentimento contra um
tipo de autoridade, representado pela própria universidade, alimentou o
radicalismo no seio de uma classe social curiosamente não afetada de
modo direto pela insatisfação econômica. Na França, a associação da
figura de \index{de Gaulle, Charles}de Gaulle com a de um tirano, embora comum, era injustificável
em muitos aspectos, e derivava da necessidade de união de toda a
esquerda, que buscava afirmar"-se e instaurar um governo popular que
discriminasse qualquer forma de afirmação de poder pessoal. Naquele
momento, ainda não estava claro que o inimigo não era mais uma pessoa
(os fantasmas de \index{Stalin, Josef}Stalin, \index{Hitler, Adolf}Hitler e \index{Mussolini, Benito}Mussolini) ou categoria social (a monarquia, a burguesia), mas algo invisivelmente nefasto, sem rosto, que
alimentava e educava a juventude rebelde: o poder socioeconômico --
ditador de hábitos, necessidades, valores e modos de relações humanas.

É nessa atmosfera libertária, em que o argumento de autoridade perdia a
razão e é empenhado em desabonar quaisquer formas de poder, que \index{Barthes, Roland}Barthes
escreve ``A morte do autor''.

Não é difícil compreender a inclinação ideológica de seu gesto, bem como
a vinculação que estabelece entre a figura autoral e a imagem de um
tirano. Seu manifesto é fruto de um momento coletivo de grande \emph{élan}
revolucionário. Não por acaso, está entremeado de expressões que apontam
para um mesmo universo conotativo: o autor é o dono de um Império muito
poderoso, diz \index{Barthes, Roland}Barthes. Portanto, a autoria é considerada sinônimo de
autoritarismo, aprisionamento, restrição, centralização, e contra ela
operam termos como ``destruição'', ``apagamento'', ``desligamento'',
``afastamento'' e ``dessacralização''.

A iniciativa de se afastar de quaisquer formas de poder é representativa
em \index{Barthes, Roland}Barthes. Em sua aula inaugural no Colégio da França (1977), ele
retoma o mesmo tema: ``Uma outra alegria me vem hoje, mais grave porque
mais responsável: a de entrar num lugar que pode ser dito rigorosamente:
fora do poder''.\footnote{Roland Barthes, \emph{Aula} (tradução de Leyla
  Perrone"-Moisés), 7ª ed, São Paulo, Cultrix, 1997, p.
  9.} Essa afirmação se faz presente diante da constatação de que o
conjunto de rupturas conquistadas pela sociedade intelectual pouco menos
de uma década antes do pronunciamento de sua ``aula'' não produzira um
desligamento do poder, pelo contrário:

\begin{quote}
(\ldots{}) evidenciou"-se que, à medida que os aparelhos de contestação se
multiplicavam, o próprio poder, como categoria discursiva, se dividia,
se estendia como uma água que escorre por toda a parte, cada grupo
opositor tornando"-se, por sua vez e à sua maneira, um grupo de pressão,
e entoando em seu próprio nome o próprio discurso do poder.\footnote{Ibid.,
  p. 34.}
\end{quote}

Os produtos social, cultural e sexual da ruptura não se traduziam:
``vangloriavam"-se de pôr em evidência o que havia sido esmagado, sem ver
o que, assim fazendo, se esmagava alhures''.\footnote{Ibid., pp. 34-5.}~Sua famosa ``Aula'' é em parte um modo de justificar o texto como espaço de libertação total das formas de poder: ``Se a semiologia de que falo
voltou então ao Texto é que, nesse concerto de pequenas dominações, o
Texto lhe apareceu como o próprio índice de despoder''.\footnote{Ibid.,
  p. 35.} \index{Barthes, Roland}Barthes definia ao mesmo tempo um posicionamento diante da
literatura e a sua natureza.~A literatura lhe serve, assim, como estado
privilegiado da linguagem, como escritura emancipada de um autor, que
``permite ouvir a língua fora do poder'', e esse poder é representado,
no âmbito da linguagem, pelo ``mito da criatividade pura'', ou seja, o
homem de gênio.

A matéria em jogo é fundamentalmente ideológica: a literatura deve
distanciar"-se da palavra gregária, dos donos do sentido, e ser encarada
como escrita ``longe dos \emph{topoi} da cultura politizada''.\footnote{Ibid.,
  p. 35.} A ideia de libertação insere"-se, de resto, num conjunto de
atitudes que se traduzem de modo similar, tal como a de se buscar a
emancipação do Estado com relação à Igreja, do cidadão com relação ao
Estado, do homem com relação a Deus.

Se o pensamento intelectual nesse momento dirige"-se para a necessidade
de deixar"-se guiar a si mesmo, sua radicalização será imaginar a
possibilidade de o texto falar por si. Diante dessa liberdade, também o
papel do professor deve ser revisto, e \index{Barthes, Roland}Barthes se mostra atento a isso:
``O que eu gostaria de renovar, cada um dos anos em que me será dado
aqui ensinar, é a maneira de apresentar a aula ou o seminário, em suma,
de `manter' um discurso sem o impor''.\footnote{Ibid., p. 43.}

É ainda na ``Aula'', ministrada com distanciamento do momento de
publicação do texto aqui enfocado, que \index{Barthes, Roland}Barthes associa diretamente a
tese da ``morte do autor'' com os acontecimentos ocorridos no ano de sua
escrita:

\begin{quote}
Por um lado, e antes de mais nada, desde a Libertação, o mito do grande
escritor francês, depositário sagrado de todos os valores superiores,
desgasta"-se, extenua"-se e morre pouco a pouco com cada um dos últimos
sobreviventes do período entre as duas Guerras; é um novo tipo que entra
em cena, que não se sabe mais -- ou não se sabe ainda -- como chamar:
escritor? Intelectual? Escriptor?\footnote{Ibid., p. 41.}
\end{quote}

A desmistificação de verdades, práticas e modos de ver é um procedimento
típico em \index{Barthes, Roland}Barthes. Talvez o principal traço de união entre seus textos
seja a perseguição ao senso"-comum, aos mecanismos naturais de
cristalização de ideologias. Matar o autor é uma decisão que evidencia
sua fuga ao consensual, e a ruptura com os hábitos foi sempre uma
prática autojustificável para o crítico.

Autoria e autoridade, portanto, só poderiam ter sido tomadas como
sinônimos por \index{Barthes, Roland}Barthes.

Mas se a exclusão do autor da recepção da obra literária foi uma prática
que a geração de 68 sedimentou na crítica literária, este duplo
retrospecto, tanto retórico quanto histórico, permite"-nos compreender
que o grande gesto teórico da morte do autor, uma vez tornado consenso
estruturalista para as sucessivas gerações de críticos -- muitos dos
quais ainda atuantes --, deve ser encarado como um passo para que o
autor renasça sob uma perspectiva ao mesmo tempo menos projetiva e mais
humanizada.

\chapter*{Disfarce e fraude autoral:\\ \emph{\large Memória, testemunho e ficção}}

\addcontentsline{toc}{chapter}{\large\versal{DISFARCE E FRAUDE AUTORAL:}\\ {\footnotesize\emph{Memória, testemunho e ficção}}}
\hedramarkboth{Disfarce e fraude autoral}{}


\section*{I. Do disfarce autoral}
\addcontentsline{toc}{section}{Do disfarce autoral}

Em que pese o conhecimento geral a respeito do amplo e tradicional uso
da pseudonímia nas artes e na literatura, não deixa de surpreender que
aproximadamente 70\% dos romances publicados nos últimos trinta anos do
século \textsc{xviii} tenham sido veiculados sob o expediente do
disfarce autoral.\footnote{Robert J. Griffin, ``Anonymity and
  Authorship'', \emph{New Literary History}, v. 30, n. 4, The Johns
  Hopkins University Press, 1999, pp. 877-895.} Tendo se disseminado nos
séculos \textsc{xviii} e \textsc{xix} como uma pista falsa, a ocultação
da autoria pelo uso de um \emph{pen name}, um nome artístico diferente
daquele do autor real, é atualmente um recurso muito utilizado na
literatura para atender a propósitos editoriais que vão desde a
divulgação do material produzido, oportunamente associado a um nome
``vendável'', até a eliminação do risco de superexposição de um escritor
muito prolífico, por exemplo.

Para evitar possíveis perseguições ou simplesmente manter intacta a
própria reputação, muitos escritores aventuraram"-se disfarçadamente em
folhetins ou livros de teor contrastante com a sua imagem pública ou os
valores da época.~Não é novidade que durante séculos a pseudonímia --
compreendida, assim, como forma de anonimato -- foi encarada como meio
para preservar o direito à livre expressão do risco da censura e da
retaliação.

Um dos subgêneros do romance que, devido à própria natureza, costuma
suscitar esse artifício é o \emph{roman à clef}, que caracteriza
criticamente pessoas reais como personagens, incluindo, eventualmente, o
próprio autor. Um bom exemplo desse expediente ocorreu, no Brasil, com a
publicação de \emph{O encilhamento}, de \index{Taunay, Visconde de}Visconde de Taunay,
primeiramente veiculado em folhetim e depois editado em livro. Durante
cerca de trinta anos a autoria do romance, que se vale de uma fachada
ficcional para retratar aspectos da crise que assolou a aristocracia
carioca em 1891, foi atribuída a um certo Heitor Malheiros, somente
identificado como \index{Taunay, Visconde de}Visconde de Taunay na edição de 1923, por seu filho,
\index{Taunay, Afonso}Afonso Taunay.

Analogamente, a crônica de costumes, de verve satírica ou humorística,
costuma reclamar o disfarce. \index{Andrade, Oswald de}Oswald de Andrade, por exemplo, inventor da
chamada ``crônica de imigração'', no jornal \emph{O Pirralho}
(1912-1917), escreveu suas ``Cartas D'Abax'o Pigues'' sob o pseudônimo
de \index{Andrade, Oswald de}Annibale Scipione.~A paródia linguística do português falado pelos
italianos em São Paulo teve continuidade, após a saída de Oswald, por
meio de ``Juó Bananere'', na verdade \index{Machado, Alexandre Marcondes}Alexandre Marcondes Machado,
criador do ``Rigalegio''.\footnote{Vera Maria Chalmers, ``A crônica
  humorística de `O Pirralho''', \emph{Revista de Letras}, v. 30,
  Unesp"-Araraquara, São Paulo, 1990, pp. 33-42.}

É também sensível, por outro lado, que para superar o machismo que
entravava a publicação e a justa apreciação de seus textos, que poderiam
ser facilmente pré-julgados como superficiais, muitas escritoras se
fizeram passar por homens.~A inglesa \index{Evans, Mary Ann}Mary Ann Evans, mais conhecida como
George Eliot, autora de \emph{Middlemarch}, e \index{Dupin, Amantine Aurore Lucille}Amantine Aurore Lucille
Dupin, ou George Sand, o sempre referido autor de romances eróticos e
psicológicos do romantismo francês, adotaram, em síntese, mecanismos de
defesa não apenas do próprio gênero sexual, mas de uma autoria
específica e distinta do modo de escrita anterior.\footnote{Margaret J.
  M. Ezell, ``Reading Pseudonyms in Seventeenth"-Century English Coterie
  Literature'', \emph{Essays in Literature}, n. 21, Spring, 1994, p. 23.}

Com razoável frequência, o oposto também ocorre, embora com propósitos
diferentes: a adoção de nomes femininos por escritores está geralmente
associada a algum tipo de conflito entre a \emph{persona} autoral e o
indivíduo. \index{Ribeiro, João}João Ribeiro, por exemplo, chegou a assinar como \index{Ribeiro, João}Elza Lentz e
\index{Ribeiro, João}Maria de Azevedo; sem esquecer, é claro, \index{Rodrigues, Nelson}Suzana Flag, o pseudônimo que
assinava as novelas rocambolescas escritas por \index{Rodrigues, Nelson}Nelson Rodrigues, em
1944.\footnote{Outros escritores brasileiros de renome que ocultaram
  momentaneamente sua autoria são \index{Andrade, Carlos Drummond de}Carlos Drummond de Andrade, que
  chegou, muito brevemente, a assinar como Antônio Crispim e Artur L.
  Gomes; e Ciro dos Anjos, por vezes Belmiro Rocha. Já o escritor
  português Adolfo Rocha é até hoje mais conhecido como Miguel Torga, e
  muitos ainda pensam que Alceu Amoroso Lima e Tristão de Ataíde foram
  críticos diferentes. Um caso à parte no uso da pseudonímia é o de
  Coelho Neto, que abusou de seu emprego por motivos e para fins
  diversos: seja para verificar se o êxito de seus livros se devia, de
  fato, à sua qualidade literária, seja para marcar diferentes fases de
  sua escrita. Charles Rouget, Blanco Canabarro, Anselmo Ribas e Caliban
  foram alguns desses pseudônimos.}

Há casos, contudo, em que o uso da pseudonímia obliterou completamente a
identidade de seus autores. Foi o que sucedeu com \index{Saucer"-Hall, Fréderic}Fréderic Saucer"-Hall,
\index{Blair, Eric}Eric Blair, \index{Farrigoule, Louis}Louis Farrigoule, \index{Porter, William Sidney}William Sidney Porter e \index{Grindel, Eugène}Eugène Grindel, os nomes reais, e pouco conhecidos, de \index{Saucer"-Hall, Fréderic}Blaise Cendrars, \index{Blair, Eric}George Orwell,
\index{Farrigoule, Louis}Jules Romains, \index{Porter, William Sidney}O. Henry e \index{Grindel, Eugène}Paul Éluard, respectivamente. A maior parte dos leitores de \index{Neruda, Pablo}Pablo Neruda, para citar um último caso, provavelmente
desconhece que seu nome de batismo é \index{Neruda, Pablo}Neftalí Ricardo Reyes
Basoalto.\footnote{Para um estudo formal a respeito dos conceitos e
  listas de autores, cf. Mello Nóbrega, \emph{Ocultação e
  disfarce da autoria} -- do anonimato ao nome literário, Fortaleza, Ed.
  \textsc{ufc}, 1981. Na internet, há inúmeras listas de pseudônimos e
  nomes artísticos relativos a todas as atividades.}

A julgar que o nome que assina um texto, mesmo que homônimo ao do
sujeito empírico, é já a identificação de um sujeito relativamente
diferente deste, essa curiosa seleção de casos apenas revela de modo
mais explícito que o disfarce autoral sempre foi um recurso não apenas
absolutamente legítimo, como constitutivo do fenômeno literário.

\section*{II. Da fraude autoral}
\addcontentsline{toc}{section}{Da fraude autoral}

Os casos de disfarce autoral se restringem à alteração na assinatura,
podendo ser simplesmente referidos como \emph{alonímia}. Seja por se
restringirem à esfera da nomenclatura, seja por ocultarem a real
identidade de seu autor ou de suas personagens, constituem um fenômeno
bastante diferente daquele ao qual podemos nos referir por \emph{fraude
autoral}, cuja dimensão ultrapassa a alteração da assinatura e não pode
ser compreendida simplesmente como uma questão de resguardo pessoal ou
conveniência artística. Esse é já um expediente que, ao invés de
simplesmente ocultar, \emph{forja} ou \emph{corrompe} o dado real.
Embora possa ser catalogado historicamente, sua prática encontrou
especial relevo na contemporaneidade, por vir sendo adotada por
escritores como estratégia de inserção no mercado editorial. Vamos aos
fatos mais recentes.\footnote{Dada a sua pouca repercussão no Brasil,
  optei por resenhar as seguintes matérias: Lindesay Irvine, ``Nazi
  flight memoir was fiction, author confesses'' {[}3 mar. 2008{]},
  disponível em
  \textless{}http://www.theguardian.com/books/2008/mar/03/news.film\textgreater{},
  acesso em 22 nov. 2017; Blake Eskin, ``Crying Wolf: Why did it take so
  long for a far"-fetched holocaust memoir to be debunked?'' {[}29 fev.
  2008{]}, disponível em
  \textless{}http://archive.is/iHRk6\textgreater{}, acesso em 22 nov.
  2017. Nesses sites, há links para matérias complementares sobre o
  tema, igualmente consultadas e disponíveis na bibliografia.}

Em fevereiro e março de 2008, alguns jornais europeus e americanos
noticiaram que \index{Wael, Monique De}Misha Defonseca, escritora belga radicada nos
\textsc{eua}, autora do best"-seller \emph{Misha: A Mémoire of the
Holocaust Years}, publicado em 1997 e traduzido para dezoito línguas,
inventou suas memórias.

As memórias de \index{Wael, Monique De}Misha são mirabolantes. Uma garota judia nascida em
Bruxelas viaja cerca de 3.000 quilômetros, a pé e sozinha, da Bélgica à
Ucrânia, durante a Segunda Guerra, fugindo da perseguição nazista. No
percurso, ao cruzar a Polônia, ela consegue escapar do Gueto de
Varsóvia, que, lembremos, reuniu e exterminou meio milhão de judeus a
partir de 1939. Como se já não fosse o bastante, \index{Wael, Monique De}Misha conta ter passado
meses em uma floresta, sobrevivido a dois longos encontros com lobos, e
esfaqueado até a morte um soldado alemão que a teria estuprado dos 7 aos
11 anos.

Apesar das fortes suspeitas levantadas a respeito da veracidade de suas
memórias, muitos deram crédito a \index{Wael, Monique De}Misha, fazendo da inverossimilhança um
ingrediente a mais para a narrativa.~Na época, chamou atenção a
estrondosa repercussão que o livro obteve. Lançada em janeiro de 2008 no
cinema francês como \emph{Survivre avec les loups}, a saga de \index{Wael, Monique De}Misha
rendeu à sua autora e à co"-autora, Vera Lee, a quantia de 32,4 milhões
de dólares num processo por direitos autorais.

Tamanha repercussão,~se bem considerada,~não chega a causar estranheza:
em culturas pautadas pelo excesso, é exemplar a necessidade insurgente
de se acreditar que o extraordinário tenha realmente acontecido. É
também verdade, contudo, que muito pouco do que realmente se passou
durante a Segunda Guerra, como, por exemplo, os campos de concentração,
possa ser encarado como ``verossímil''.~Este foi, em síntese, o
argumento usado pela produtora do filme, Vera Belmont, para defender o
teor de verdade da história. Sua aparente inconsistência apenas
confirmava sua autenticidade -- era preciso acreditar que a história
tinha de fato ocorrido. Por outro lado, dificilmente se acreditaria num
relato como esse se o narrador fosse, não a vítima, mas o
verdugo.~A ideia frequentemente difundida é, afinal, a de que as vítimas
têm sempre razão. Enquanto não se provava o contrário, a narrativa de
\index{Wael, Monique De}Misha contava com a condescendência de alguns leitores e a curiosidade
de muitos outros.

\index{Wael, Monique De}Misha Defonseca -- nome, aliás, afrontosamente inverossímil --, já na
casa dos 70 anos, ocupou novamente as manchetes dos principais jornais
americanos e europeus ao confessar que não é judia, que seu verdadeiro
nome é \index{Wael, Monique De}Monique De Wael, que é filha de dois membros católicos da
resistência belga, e que falsificou suas memórias.~A ``confissão'' só
ocorreu, no entanto, depois de a autora ter sido investigada pelo jornal
belga \emph{Le Soir}, e sua história ser colocada à prova por
jornalistas e especialistas em diferentes áreas, que apontaram,
inclusive, em uma rede da televisão belga, para equívocos relacionados à
descrição do comportamento de animais com seres humanos, realizada no
livro, e para importantes incongruências de datas históricas.~Foi
descoberta, por exemplo, a matrícula da autora numa escola belga um ano
depois do período que ela diz ter iniciado sua peregrinação pela Europa,
cuja data informada, aliás, é bem anterior ao ano em que os nazistas
passaram a perseguir os judeus residentes em Bruxelas.

Apesar de seu livro ter, desde o início, levantado suspeitas, a
confirmação de que ele foi em grande parte inventado causou espanto,
porque a farsa que a autora sustentou por nada menos que onze anos tanto
era uma atitude literária quanto compunha sua imagem social junto a
editores, a comunidade semita e até amigos.~Mesmo tendo sido
desmascarada, a autora não admitiu que suas memórias fossem fruto de uma
invenção deliberada. Segundo Monique, depois de seus pais terem sido
presos e sua guarda entregue ao avô e ao tio, ela passou a ser chamada
de ``filha do traidor'', por seu pai ter sido acusado de delator. Embora
não fosse judia, Monique declarou ``sentir"-se'' como tal. Por isso,
segundo ela, seu livro ainda mantém uma verdade emocional: ``O livro é
uma história, a minha história'', declarou. ``Não é a verdadeira
realidade, mas a minha realidade. Há momentos em que eu acho difícil
diferenciar a realidade do meu mundo interior''.\footnote{``The book is
  a story, it's my story (\ldots{}) It's not the true reality, but it is my
  reality. There are times when I find it difficult to differentiate
  between reality and my inner world''. In Lindesay Irvine, ``Nazi
  flight memoir was fiction, author confesses'', op. cit.}~A autora
acusou seu editor de oportunismo e de a ter convencido a publicar sua
ficção como um livro de memórias, o que não passa de mais uma de suas
histórias: na verdade, a mitômana Monique já sustentava a farsa muito
antes de publicar seu livro, tendo"-a relatado como sua, inclusive, numa
sinagoga.

A exemplo desse caso escandaloso de fraude autoral, outras falsificações
chamaram a atenção na Europa e nos \textsc{eua}.

Em 1992, um suposto descendente de judeus publicou, sob o pseudônimo de
\index{Wilkomirski, Binjamin}Binjamin Wilkomirski, \emph{Fragments}, livro que narra suas supostas
experiências como um menino de quatro anos num campo de concentração
nazista na Polônia. Mais tarde, \index{Wilkomirski, Binjamin}Wilkomirski revelou ter passado a
infância, durante o período da Segunda Guerra, na Suíça.

Recentemente, \index{Albert, Laura}Laura Albert fingiu ser o autor \index{Albert, Laura}J. T. LeRoy, supostamente
o filho viciado de uma prostituta. Um grupo de jornalistas australianos
também acusou \index{Beah, Ishmael}Ishmael Beah, autor das memórias traduzidas para o
português como \emph{Muito longe de casa}, de distorcer sua vida como um
menino soldado em Serra Leoa.~Já \index{Brougham, Bernard}Bernard Brougham, sob o pseudônimo de
\index{Brougham, Bernard}Holstein, autor de \emph{Stolen Soul}, de 2004, relatou ser um
sobrevivente do Holocausto, e foi tão longe com sua história a ponto de
fazer uma tatuagem similar à dos prisioneiros dos campos de
concentração, na tentativa de conferir autenticidade a seu livro.

A conhecida apresentadora de \textsc{tv}, Oprah Winfrey, viu"-se, por
duas vezes, envolvida em casos de fraude autoral. Em 2006, o volume
traduzido como \emph{Um milhão de pedacinhos}, de \index{Frey, James}James Frey, foi
escolhido para compor seu popularíssimo clube do livro, e mais tarde se
descobriu que a história não era completamente verdadeira. Dois anos
depois, \index{Seltzer, Margaret}Margaret B. Jones, autora de \emph{Love and Consequences},
elogiado pela \emph{Oprah Winfrey Magazine}, reconheceu que as memórias
de uma suposta órfã, mestiça de nativos americanos e brancos num bairro
predominantemente negro de Los Angeles, nos anos 1980, não passam de
pura ficção. \index{Seltzer, Margaret}Jones teria sido transferida de um lar adotivo a outro
durante a infância, entrado para o mundo das drogas aos 13 anos, ganho
seu primeiro revólver aos 14 e sido violentada. Suas memórias, narradas
do ponto de vista da menina -- note"-se o padrão, da vítima, portanto --,
foram muito bem recebidas nos \textsc{eua}, não apenas por reproduzir a
gíria dos guetos de L.A., ou pelo elogiado trabalho de descrição
realizado pela autora, mas sobretudo por contarem a história de
reinserção social de uma órfã traficante de drogas que deixa o subúrbio
para se formar na Universidade de Oregon. \index{Seltzer, Margaret}Jones, que concretiza o sonho
de superação americano, é na verdade \index{Seltzer, Margaret}Margaret Seltzer, uma mulher
branca, hoje na casa dos 50 anos, e que viveu com seus pais biológicos
de uma família de classe média. Ela foi denunciada pela própria irmã,
que se surpreendeu com um perfil da autora no \emph{The New York Times}
e decidiu revelar a verdade ao jornal.\footnote{Para uma resenha
  detalhada e positiva do livro, anterior à descoberta da fraude, cf.
  Michiko Kakutani, ``However Mean the Streets, Have an Exit Strategy''
  {[}26 fev. 2008{]}, disponível em
  \textless{}https://nyti.ms/2AHzTYj\textgreater{},
  acesso em 22 nov. 2017. Ainda no \emph{The New York Times}, a
  revelação do embuste, publicada na semana seguinte, pode ser
  consultada em: Motoko Rich, ``Gang Memoir, Turning Page, Is Pure
  Fiction'' {[}4 mar. 2008{]}, disponível em
  \textless{}https://nyti.ms/2QgwLYV\textgreater{},
  acesso em 22 nov. 2017.}

Esse apanhado de fraudes autorais, que só aumenta ano após
ano, expõe um mecanismo perverso de produção e circulação da
\emph{literatura de mercado} contemporânea, que não se deve confundir,
portanto, com a noção de disfarce. Em outras palavras, a noção de
\emph{fraude} aplica"-se melhor à \emph{literatura de mercado}, voltada a
um público ávido por histórias reais. Mais do que isso, ela nos permite
adotar um ângulo pouco comum para discutir um dos temas centrais da
moderna teoria literária: o binômio presença / ausência do sujeito na
escrita.

\section*{III. Os mundos de mentira e as~reações~à~sua~descoberta}
\addcontentsline{toc}{section}{Os mundos de mentira e as reações à sua descoberta}

A última fraude autoral por que passamos, de \index{Seltzer, Margaret}Seltzer (Jones), a jovem de
classe média residente em Los Angeles que se fez passar por órfã e
mestiça de índios e brancos, provocou uma reação significativa na
sociedade americana. A editora Riverhead Books, um selo da Penguim,
recolheu 19 mil cópias do livro, pediu desculpas aos leitores e cancelou
uma turnê da autora. Além disso, a editora ofereceu a devolução do
dinheiro aos seus leitores. Também a \emph{Oprah Winfrey Magazine} reviu
seus elogios a \emph{Love and Consequences} e o reclassificou como
ficção.

É interessante constatar que desde o escândalo com \index{Frey, James}Frey, cujas memórias
são parcialmente reais, as editoras passaram a checar melhor a
veracidade dos relatos, e que, paralelamente, os escritores passaram a
aperfeiçoar seus embustes.~No caso de \index{Seltzer, Margaret}Seltzer (Jones), a escritora teria
apresentado à editora falsos irmãos adotivos, um suposto professor e um
escritor, além de fotos e cartas que comprovariam seu envolvimento com
gangues. E, no entanto, sua história foi integralmente fabricada.

Revelada a farsa, \index{Seltzer, Margaret}Seltzer argumentou que seu livro, baseado na
convivência com ex"-drogados durante um trabalho de prevenção e combate
às drogas em Los Angeles, era uma oportunidade para dar voz a essas
minorias, discriminadas pela sociedade.\footnote{Outro artigo consultado
  sobre a repercussão da fraude autoral de Seltzer (\index{Seltzer, Margaret}Jones) é Carol
  Memmott, ``Author´s `Love and consequences' memoir untrue'' {[}3 mai.
  2008{]}, disponível em \textless{}https://bit.ly/2yL3gaF\textgreater{}, acesso em 22 nov. 2017.}

A exemplo de \index{Wael, Monique De}Misha, \index{Seltzer, Margaret}Jones usava uma \emph{persona} na vida real, e seus
colegas acreditavam que ela realmente havia crescido em lares adotivos
de famílias negras no subúrbio de Los Angeles.~Durante os três anos de
contato com Ms. McGrath, sua editora, \index{Seltzer, Margaret}Jones, até então a desconhecida
frequentadora de uma oficina literária, não levantou a menor suspeita
sobre a veracidade da história. Depois de publicado o livro, a
verossimilhança da narrativa convenceu também a jornalista Mimi Read,
que publicou um perfil de \index{Seltzer, Margaret}Jones (isto é, tal como ela o construiu) no
prestigiado \emph{The New York Times}.

Em 2006, na época do escândalo que envolveu o livro de \index{Frey, James}James Frey -- o
ponto inaugural dessa onda de descoberta de \emph{ficções biográficas}
--, \index{O'Rourke, Meghan}Meghan O'Rourke escreveu um artigo questionando o tratamento dado a
esses livros pelas editoras, que, muitas vezes, mesmo antes de serem
descobertos como fraudulentos, são reeditados com notas de advertência a
respeito da alteração de nomes e fatos, como ocorre com o livro de \index{Frey, James}Frey.
A seu ver, ao invés de essas notas funcionarem simplesmente como
precaução para evitar que seus editores sejam processados, o que elas na
verdade ``esclarecem'' é que em alguma instância aquelas personagens e
aqueles fatos narrados são baseados em personagens e fatos reais.~Ou
seja, o que no caso de um processo judicial garantiria a ficcionalidade
do relato, até então funcionava como um atestado de veracidade,
sugerindo ao leitor uma leitura biográfica do texto. Por esse motivo,
\index{O'Rourke, Meghan}O'Rourke considera que seria mais apropriado tratar esses autores como
escritores que se apropriaram da licença poética para fantasiar sobre
algum nível de experiência pessoal. Que não se trata, portanto, de
livros de memórias, tampouco ficcionais, mas de um produto híbrido, que
ela sugere reclassificar, não sem certa dose de ironia, como ``real
ficction''.\footnote{Cf. Meghan O´Rourke, ``Lies and Consequences: Why
  are book editors so bad at spotting fake memoirs?'' {[}4 mar. 2008{]},
  disponível em:
  \textless{}https://bit.ly/2Rt6YfZ\textgreater{},
  acesso em 22 nov. 2017.} Note"-se, a propósito dessa expressão, que,
quando são flagrados em delito, os autores das fraudes autorais muitas
vezes adotam em defesa própria justamente o discurso da teoria
literária, que relativiza os limites entre ficção e realidade.

O uso da ficção num livro de memórias ou numa autobiografia não causa
estranhamento desde que ela seja empregada para reforçar, intensificar,
reavivar a visão dos fatos. Pode"-se dizer que a ficcionalização é
aceitável nesses casos, conquanto não desabone o caráter de depoimento
que se pode atribuir à narrativa.

Narrar, mesmo que se trate de reportar uma experiência real, recuperada
pela memória, é já conferir sentido, significa acrescentar algo ao
vivido -- o que implica, portanto, a realização de operações ficcionais.

Feitas essas ressalvas, cabe considerar que os casos relatados na parte
anterior deste ensaio são considerados fraudulentos não por serem de
narrativas de memórias ou autobiografias que se valeram do expediente
ficcional em sua realização, uma vez que este é, conforme ponderamos,
intrínseco à arte narrativa. Antes, será mais adequado considerar que,
nesses casos, a ficcionalização não atua como um \emph{recurso}, e sim
como um \emph{substituto} sistemático da memória. Ela participa
arbitrariamente da narrativa. E ali se mantém como elemento oculto: nas
\emph{fake memoirs}, o dado ficcional jamais deve ser encarado como tal.

É verdade que caracterizar esse tipo de narrativa como fraude significa
indicar um desvio no \emph{processo} de escrita; a fraude não toca, em
princípio, o que entendemos por literário, não diz respeito ao resultado
obtido, e sim ao seu \emph{modus operandi}.~Mas essa é uma impressão
falsa: num livro de memórias, é preciso considerar que esse processo não
é um dado anterior à experiência de leitura; ao invés disso, ele vem
exposto no texto, como aquelas enormes estruturas de aço que se exibem
aos olhos, do lado de fora dos edifícios. Num livro de memórias, a
memória é para ser vista, como uma cicatriz que expõe as marcas de
contato com o mundo.

O leitor que procura essa relação, no lugar de simplesmente se ater
credulamente à fabulação, não é o leitor que melhor contribui para as
``memórias'' de \index{Wael, Monique De}Misha e \index{Seltzer, Margaret}Jones. Isso porque suas escritas projetam como
leitor ideal o mesmo leitor de uma narrativa de ficção, isto é, um
sujeito menos interessado no mundo do que em firmar um \emph{pacto
ficcional} com o autor. O caráter fraudulento desse tipo de narrativa
reside, desse modo, na inadequação de seus recursos ao gênero a que ela
se vincula, uma vez que somente a encarando como um romance, e não como
um livro de memórias, é que o leitor se sentiria plenamente
desencorajado para uma averiguação factual.~Averiguação esta que poria a
perder o caráter supostamente experiencial da narrativa.

O \emph{abuso do ficcional}, nesses casos, atua como um elemento
estranho ao gênero, porque no lugar de simplesmente intensificar
experiências embaçadas pelo tempo e produzir engates entre fragmentos de
memória, superdimensiona a fabulação. São fraudes, portanto.

\section*{IV. Revendo o lugar do sujeito empírico~da~escrita}
\addcontentsline{toc}{section}{Revendo o lugar do sujeito empírico da escrita}

O fato de esses autores terem se desculpado publicamente por seus
livros, ou atribuído sua concepção a iniciativas dos próprios editores,
significa que assumem como moralmente condenável sua ficção biográfica.

É, de resto, bastante provável que essas fraudes tenham sido motivadas
por oportunismo, funcionando como ``jogadas de marketing'' para a
inserção desses autores no mercado editorial.~A discussão sobre as
\emph{fake memoirs} dificilmente pode desconsiderar que se trata de um
``gênero'' consonante a uma gorda fatia dos mercados editoriais
americano e europeu. Vivemos tempos em que os artistas fazem pesquisa
com o consumidor para identificar tendências de mercado.

Mas o que deve chamar a nossa atenção não são as motivações. Não cabe a
nós questionar se, por exemplo, o autor agiu de má fé. Esse e outros
tipos de especulação resultariam em matéria para psiquiatras e juristas.
O que nos parece realmente significativo do ponto de vista literário é,
mais propriamente, discutir as reações a esses embustes.

Tais reações permitem afirmar algo aparentemente inadmissível até hoje
para qualquer crítico que não postule raça, nacionalidade, classe
social, etnia, sexo ou conduta sexual como motivações suficientes em si
mesmas para avaliar a literatura.~Permite"-nos compreender, afinal, que
mesmo o autor tomado como sujeito biográfico não é um dado exterior à
leitura. Esse é o nervo teórico desta reflexão.

Para perseguir essa ideia, proponho uma pergunta hipotética que tende a
colocar em xeque as nossas impressões (ou certezas) sobre os fatos
mencionados: em que resultaria o livro de memórias de um esquizofrênico?
E, como decorrência desta: seria plausível acusar seu autor de faltar
com a verdade?

Essa suposição permite esclarecer que os leitores de \index{Wael, Monique De}Defonseca, \index{Seltzer, Margaret}Jones,
\index{Frey, James}Frey, \index{Wilkomirski, Binjamin}Wilkomirski e \index{Brougham, Bernard}Holstein, uma vez desvendadas as farsas biográficas dos responsáveis pelas obras, não reclamaram uma outra história (esta
foi perfeitamente aceita até a revelação de sua farsa autoral); eles
acusaram a ausência repentina de um autor cuja imagem lhes acompanhou
durante a leitura e conferiu credibilidade a uma história. Eles
reclamaram um outro autor. Foi essa ausência que implicou um desfalque
na experiência de leitura.

É preciso ponderar a esse respeito. Não resta dúvida de que o editor
disposto a devolver o dinheiro desses leitores acredita que vendeu um
produto não simplesmente danificado, mas falsificado.~Em bom português,
ele entende que seu consumidor ``levou gato por lebre''.~Mas notemos
que, para esses leitores, o que lhes foi retirado não foi exatamente o
produto, e que esse produto, na prática, não apresenta nenhum defeito de
fabricação; ele tem exatamente o número de páginas que deveria ter, a
capa apresenta as cores e a definição corretas, as palavras continuam
todas ali etc. O que lhes foi retirado foi o valor simbólico, a
possibilidade de usá-lo da maneira como ele sugeria que se fizesse. E
esse valor simbólico não pode ser compreendido como algo exclusivamente
fornecido pelo texto. O que conferia sentido a esses livros era a sua
relação com a biografia de seus autores, a possibilidade de creditá-los
como histórias reais. Autor e escrita estavam mutuamente implicados.

Ora, mas se esses leitores leram essas memórias como legítimas, e só
depois elas foram desvendadas como ficcionais (``ilegítimas'', a seus
olhos), não lhes foi propriamente tirada a possibilidade de uma leitura;
o que surgiu foi, a rigor, uma alternativa: a de realizarem uma outra
leitura, a de conferirem um outro sentido aos textos. E se, mesmo assim,
essa alternativa não lhes pareceu realmente um ganho, seja aos olhos
desses leitores, seja aos olhos de seu editor, é porque o efeito que o
desmascaramento autoral produziu sobre eles foi somente o de
deslegitimar a leitura que fizeram. Devolver"-lhes o dinheiro significa,
por mais absurdo que isso possa parecer, dizer que o prazer que tiveram
durante a leitura foi equivocado, que não foi um prazer real, que eles,
afinal, estavam errados.

De modo um tanto irônico, livrarias americanas passaram a vender
biografias do ciclista Lance Armstrong, banido do esporte por
\emph{doping} depois de ter vencido por sete vezes o \emph{Tour de
France} e ter saído vitorioso na luta contra o câncer, nas prateleiras
de textos ficcionais. Assim como ocorre com Armstrong, se \index{Wael, Monique De}Monique De
Wael não espelha mais o que a protagonista de \emph{Misha: A Mémoire of
the Holocaust Years} viveu, e, por isso, essa leitura perde
completamente o interesse, temos uma constatação empírica de que o autor
não é uma entidade exterior ao texto; ele está presente no ato de
leitura.

Quando nós, do alto de nossas plataformas acadêmicas, defendemos que não
faz diferença se o sujeito empírico que escreve viveu ou não a história
que garante ter vivido, que isso não muda em nada a história, porque,
afinal de contas, esta passou a ter uma vida própria na imaginação do
leitor, fazemos isso fingindo"-nos leitores apartados do mundo real, sem
nos darmos conta do cinismo intelectual do argumento. Provavelmente, a
maior parte das pessoas que defendem esse ponto de vista não é
suficientemente radical para conferir legitimidade à assombrosa corrida
que o, alguns anos antes, mediano atleta canadense \index{Johnson, Ben}Ben Johnson realizou,
em 1988, nas Olimpíadas de Seul, sob efeito de uma poção mágica chamada
estanozolol. Ao invés disso, essas pessoas sentiram"-se enganadas e
aliviadas ao saber que o falso super"-homem das pistas teve sua
aposentadoria antecipada ao ser banido do esporte.

Sempre haverá um exemplo flagrante em nossas consciências de que
importam, sim, as condições em que uma obra é realizada. A estética não
sobrevive à ética.

Considerando a situação pelo prisma mercadológico (fortemente presente,
afinal, na recepção literária da contemporaneidade), a imagem de um
sujeito empírico testemunhal fornece uma orientação aos leitores que se
estabelece desde a confecção da capa até as estratégias de marketing
usadas para a divulgação do livro.~Tudo isso alimenta expectativas no
leitor, que é, antes de mais nada, um consumidor daquele objeto, e que,
portanto, pretende usá-lo com um mínimo de controle sobre suas
finalidades. A literatura, uma vez inserida na sociedade de consumo,
depara"-se com a mesma exigência dos demais produtos circulantes -- a de
oferecer uma garantia a seus leitores.

Quando a verdadeira identidade de \index{Wael, Monique De}Misha foi revelada, o que com ela se
revelou não foi exatamente um outro livro, mas um outro modo de ler o
mesmo livro -- um modo, aliás, imprevisto por seus editores, impossível
de ser avaliado e, por conseguinte, rebelde à sua inserção no mercado.
Uma analogia interessante pode ser feita com o mercado de arte: a mesma
tela pode valer ``X'' se atribuída a \index{Rembrandt Harmenszoon van Rijn}Rembrandt, ``X/10'' se atribuída a
algum pintor do seu círculo, ou ``X/100'' se for considerada uma
falsificação. Seu valor é diretamente dependente do estabelecimento de
sua autoria. Por quanto será multiplicado o valor de mercado das duas
pequenas esculturas em bronze escuro, atualmente expostas em Cambridge e
atribuídas ao escultor holandês \index{Tetrode, Willem Danielsz Van}Willem Danielsz Van Tetrode, se
confirmada a sua real autoria a \index{Michelangelo di Lodovico Buonarroti Simoni}Michelangelo? E como elas passarão a ser
reavaliadas artisticamente?

Vamos extrapolar esses exemplos, rumo à ficção.

É bastante sensível que, embora haja uma diversidade enorme entre
escritores como \index{Marinetti, Filippo Tommaso}Marinetti, \index{Saucer"-Hall, Fréderic}Cendrars, \index{Apollinaire, Guillaume}Apollinaire, \index{Andrade, Mário de}Mário de Andrade, \index{Hemingway, Ernest}Hemingway, \index{Pasolini, Pier Paolo}Pasolini ou \index{Llosa, Mario Vargas}Vargas Llosa, por exemplo, todos eles foram homens de ação, que projetaram sobre sua escrita uma excitação pelo
mundo que é comum entre si, e ao mesmo tempo bastante diferente daquela
força predominantemente meditativa que escritores como \index{Proust, Marcel}Proust, \index{Pessoa, Fernando}Pessoa,
\index{Eliot, Thomas Stearns}Eliot, \index{Andrade, Carlos Drummond de}Drummond, \index{Lispector, Clarice}Lispector ou \index{Borges, Jorge Luis}Borges conferiram a seus textos. Essa impressão, evidentemente, deve"-se em grande parte ao teor próprio dos textos desses autores, mas seria insensível postular que ela não tenha
relação com as imagens sociais projetadas por quem os escreveu. Se
alguns escritores procuram apagar seus traços biográficos para ressaltar
aqueles estritamente literários, em outros é virtualmente impossível
separar o campo artístico do biográfico. O apreço que por muito tempo se
alimentou pelo teatro de \index{Marcos, Plínio}Plínio Marcos, no Brasil, não estava dissociado
(e ainda não está) de sua posição política e de sua figura percorrendo
as filas dos teatros para vender seus livros, numa sacola a tiracolo,
durante os anos de chumbo.~O mesmo se pode dizer da ausência misteriosa
e ao mesmo tempo magnética da figura autoral de \index{Trevisan, Dalton}Dalton Trevisan.
Personalismo? Não confundamos. As obras de ambos se compõem em parte
dessas imagens autorais, e estas se tornam mais nítidas à luz dessas
obras. Trata"-se de uma simbiose, não de uma binariedade.

Notemos que é também possível afirmar que mesmo a leitura de um texto de
autoria desconhecida, ou anônima, ou apagada (penso em \index{Mallarmé, Stéphane}Mallarmé), leva
em conta essa ausência como um espaço a ser preenchido ou, no último
caso, que não deve ser preenchido (o que atesta, de resto, sua presença
como um proibitivo). Mesmo o anonimato é uma marca presente no ato de
leitura.

Quantos leitores não abandonaram um escritor devido às suas posições
políticas ou à sua conduta social, ou, ao contrário, quantos de nós não
passamos a ler um escritor justamente privilegiando esses critérios --
depois de assistir a uma entrevista sua, por exemplo? Não é, na verdade,
em grande parte das vezes, a imagem social ou cultural de um autor
contemporâneo o que nos cativa e nos convida a iniciar uma leitura? E
não é, precisamente, consciente disso e das possibilidades de lucro que
a exploração dessa imagem pode gerar, que seu editor comercializa essa
imagem?

Entre os fenômenos literários mundiais que melhor exemplificam esse
processo está o escritor e intelectual italiano \index{Eco, Umberto}Umberto Eco. Os leitores
de \index{Eco, Umberto}Eco são realmente capazes de abstrair da leitura que realizam a
imagem autoral mundialmente projetada do intelectual inteligente,
acessível e bem"-humorado, cujo interesse e contribuição vão desde os
quadrinhos e a alta tecnologia até a estética tomista e a história dos
templários? A leitura não exerce um efeito justamente oposto a essa
suposta separação, isto é, o de complementar essa imagem a cada novo
passo, a cada nova descoberta que seus leitores fazem de suas
personagens?~Se supusermos que hoje fosse revelado que os romances
atribuídos a \index{Eco, Umberto}Eco são, na verdade, de uma discreta senhorinha italiana,
que depois de se aposentar como costureira resolvera escrever romances,
não nos veríamos quase que condenados a refazer nossas leituras de
\emph{O nome da rosa} e \emph{O pêndulo de Foucault}? E não seriam,
mesmo, experiências diferentes reler esses romances projetando sobre si
a imagem dessa nova, e inusitada, autora empírica?

É um privilégio poder iluminar essa suposição com o conto ``Pierre
Menard, autor do \emph{Quixote}'', de \index{Borges, Jorge Luis}Jorge Luis Borges, autor de alguns
dos mais curiosos exercícios intelectuais da literatura moderna. No
texto do escritor argentino, Menard teria reescrito os capítulos 9 e 38
do Livro \textsc{i} da obra de \index{Cervantes, Miguel de}Cervantes, e, ao reescrevê-los, o teria
feito de forma idêntica ao original. Apesar disso, ao confrontar dois
fragmentos perfeitamente idênticos, o narrador borgeano os considera
totalmente diferentes. E, através dessa confrontação aparentemente
absurda, algo se revela: ``o texto de \index{Cervantes, Miguel de}Cervantes e o de Menard são
verbalmente idênticos, mas o segundo é quase infinitamente mais rico.
(Mais ambíguo, dirão seus detratores; mas a ambiguidade é uma
riqueza.)''.\footnote{Jorge Luis Borges, ``Pierre Menard, autor do
  Quixote'', in \emph{Ficções} (tradução de Davi Arrigucci Jr.), São
  Paulo, Companhia das Letras, 2007, p. 42.}~O deslocamento temporal dos
textos, conquanto permaneçam idênticos, recompõe seu sentido para além
do previsto no original. Quando \index{Cervantes, Miguel de}Cervantes escreve: ``\ldots{} a verdade, cuja
mãe é a história, êmula do tempo, depósito das ações, testemunha do
passado, exemplo e aviso do presente, advertência do futuro'', esse
trecho não parece fazer mais, aos olhos do narrador borgeano, do que uma
enumeração que consiste num ``mero elogio retórico da
história''.\footnote{Ibid., pp. 42-43.} Em contrapartida, o mesmo trecho
(citado novamente no conto, \emph{ipsis litteris}), mas agora atribuído
a Menard, acena"-lhe um sentido completamente diferente:

\begin{quote}
A história, \emph{mãe} da verdade, a ideia é assombrosa. Menard,
contemporâneo de William James, não define a história como uma indagação
da realidade, mas como sua origem. A verdade histórica, para ele, não é
o que aconteceu; é o que julgamos que aconteceu. As cláusulas finais --
``exemplo e aviso do presente, advertência do futuro'' -- são
descaradamente pragmáticas. Também é vívido o contraste de estilos. O
estilo arcaizante de Menard - estrangeiro, afinal -- padece de alguma
afetação. Não assim o do precursor, que maneja com desenfado o espanhol
corrente de sua época.\footnote{Ibid., p. 43. O leitor notará que citei
  e tratei do mesmo trecho de \index{Borges, Jorge Luis}Borges no capítulo 1. À parte o fato de
  que o trecho mereça ser relido, terminei por emular \index{Borges, Jorge Luis}Borges: o mesmo
  trecho, aqui, já adquire, afinal, outro sentido.}
\end{quote}

A atribuição de uma nova autoria encerra, em si mesma, um novo sentido
para o texto.~\emph{O nome da rosa} escrito por nossa hipotética
costureira aposentada seria já um outro romance.

A expressão ``ler \index{Eco, Umberto}Umberto Eco'' é mais do que uma metonímia. Sem
desfazer esse preconceito teórico, continuaremos fingindo que lemos
apenas pelas palavras que se encontram no papel, como se elas não nos
conduzissem para nada que estivesse para além daquele retângulo branco,
e como se esses mundos além"-margem não fossem capazes de povoar de
peixes esse exíguo aquário.

O autor está realmente presente no ato da leitura, e não me refiro,
neste momento, ao ``autor implícito'', que pode ser depreendido ali por
indícios, e que reclama, por sua vez, um ``leitor modelo''. Essas
expressões mantêm a discussão ainda no plano textual, e são, decerto,
mais do que pertinentes. Mas, neste momento, o que é necessário afirmar
é algo muitíssimo mais simples do que isso, e talvez por isso mesmo mais
radical também para nós, que estamos habituados aos conceitos e muitas
vezes a malabarismos teóricos cada vez mais complicados: quando lemos
\emph{O nome da rosa}, o que procuramos, em certa medida, é nos
aproximar do \index{Eco, Umberto}Umberto Eco autor empírico, aquele senhor sorridente que
circulava por Bologna e era chamado de ``il professore''. E fazemos isso não
porque nutrimos especial simpatia por sua figura, mas porque ela se
estende ao universo próprio que ele tateou e transformou com suas
ideias, olhares e emoções. E porque esse universo os transformou
conjuntamente. Fazemos isso porque esse senhor, que por vezes acompanha
nossas leituras espiando por uma pequena foto nas orelhas dos volumes,
tem sempre algo a acrescentar sobre o que está dito, e porque o que está
dito tem, reciprocamente, algo a dizer a seu respeito.

Se, no entanto, para o leitor não italiano, \index{Eco, Umberto}Umberto Eco for
fundamentalmente o autor de \emph{O nome da rosa}, basta pensar em casos
nacionais: é pouco provável que o leitor brasileiro dos romances de
\index{Hollanda, Chico Buarque de}Chico Buarque de Hollanda deixe de pensar na figura de seu autor como um
compositor consagrado. Tanto o valor de mercado desses romances, medida
por sua tiragem, quanto algo do efeito de sentido que se atribui a eles
dependem diretamente dessa imagem autoral.

Quando lemos os contos e romances de \index{Tolstói, Liev}Tolstói, como nos mantermos
desinteressados sobre as censuras que ele fez à medicina de sua época e
sobre suas utopias sociais e religiosas, e que depois ficou conhecido
como "tolstoísmo"? Não serão legítimas essas curiosidades? E elas não
devolverão ao texto sentidos suscitados por ele? Por que, afinal,
deveríamos encarar como um dever do leitor diante do texto ser míope a
todo o resto que se mantém ao seu redor, se o próprio texto não o faz?

Quando lemos ``O nariz'', de \index{Gogol, Nikolai@Gógol, Nikolai}Gógol, estando cientes de que o autor o
escreveu já num estágio avançado de sua progressiva doença mental --
cientes, afinal, de que ele realmente percebera seu nariz fora do corpo
--, não fazemos uma leitura bastante mais grave de seu conto do que a
daquele leitor que, desconhecendo o dado, prefere encarar o fantástico
episódio como ``ficcional'', simplesmente? Afinal, \index{Gogol, Nikolai@Gógol, Nikolai}Gógol, como \index{Wael, Monique De}Monique De Wael (\index{Wael, Monique De}Misha Defonseca), acreditava em sua mentira. E se uma dessas
leituras tiver que ser (porque não tem que ser) declarada preferível à
outra, a primeira não levaria vantagem sobre a segunda, por ser a mais
inclusiva e informada?

Não se trata de reduzir essa relação à curiosidade, de resto legítima,
sobre o que há de documental num texto literário; e isso simplesmente
porque a imaginação de um escritor é a parte fundamental de sua
biografia. Reconhecer esse seu aspecto significa reconhecer que a
leitura está longe de ser uma experiência solitária, apartada do mundo.

Ela é, antes, a transformação desse mundo enraizado em fatos no universo
mais livre e flutuante das possibilidades.

\section*{V. Para além do risco romântico}
\addcontentsline{toc}{section}{Para além do risco romântico
\bigskip}
%\medskip}

É claro que não seria possível afirmar de uma maneira tão francamente
aberta que a leitura é uma atividade dialógica que inclui o autor, sem
supor que já nos desvencilhamos da crítica explicativa e das ilações
psicobiográficas que resultavam em abordagens redutoras, limitadoras do
alcance e da magia do texto literário.~E talvez ainda não seja o caso de
defendê-lo para alguém que não esteja realmente familiarizado com as
sólidas contrapartidas a esse respeito, e que incorra desavisadamente em
raciocínios causalistas para tentar justificar o que leu.~Pessoalmente,
também evitaria insistir nesse ponto diante de professores que mantêm um
arsenal de argumentos prontos para serem disparados diante de qualquer
brecha causalista, simplesmente porque eu compartilho desses argumentos,
e apenas não vejo como eles possam ser aplicados para o que venho
defendendo.~Penso, por outro lado, nos nossos alunos iniciantes, a quem
caberá primeiramente dizer que há um bom tempo a teoria literária
empenhou"-se num processo de recusa sistemática do autor como tutor do
sentido do texto, tarefa que parece ter diminuído sensivelmente as
possibilidades de equívoco explicativo e que, por outro lado, pelos
exageros que cometeu, tornou o texto bastante mais insípido.\footnote{Veja"-se,
  a esse respeito, a \emph{mea culpa} de Todorov, a partir do balanço
  histórico que propõe de sua fase estritamente formalista, que traduz
  como uma ``maneira ascética de falar da literatura''. \emph{A
  literatura em perigo}, (tradução de Caio Meira), Rio de Janeiro,
  Difel, 2009, p.31.}

A confusão mais importante a se evitar nesse processo de retomada do eu
empírico é a de identificá-la com uma visão romântica de literatura,
isto é, que pressupõe a transparência do sujeito, o que permitiria ao
crítico ler o texto como \emph{expressão} do conteúdo do eu criador.
Isto porque é próprio dessa perspectiva confundir o eu lírico ou o
narrador com o sujeito empírico anterior à escrita.

A concepção de emissores autônomos com relação ao eu empírico nasce da
crise filosófica do sujeito no século \textsc{xix}, que encontra em
Nietzsche, e sua concepção de uma arte livre de subjetividade, uma
referência fundamental. A vinculação da ideia de ``máscara'' ao sujeito
artístico é caracteristicamente nietzscheana e está associada ao ideal
baudelairiano de uma poesia impessoal, à afirmação rimbaudiana de uma
poesia objetiva, em que o ``eu é um outro'', e ao projeto mallarmaico do
desaparecimento do sujeito como locutor. Estes autores compõem um
ideário estético que, em síntese, reage aos excessos da sensibilidade
romântica.~Sua obscuridade simbolista pode ser compreendida como a busca
da destruição dos dados referenciais e do sujeito empírico.

No âmbito da teoria da literatura, já em 1916, \index{Walzel, Oskar}Oskar Walzel falava em
\emph{desegotização} da poesia moderna.\footnote{Apud Dominique Combe,
  ``La référence dédoublée, le sujet lyrique entre fiction et
  autobiographie'', in \emph{Teorías sobre la lírica}. Figures du Sujet
  Lyrique (organização de Dominique Rabaté), Paris, \versal{PUF},1996, p. 136.}
Em nome da lógica interna da obra, \index{Benjamin, Walter}Walter Benjamin, em 1922, em seu
ensaio sobre \index{Goethe, Johann Wolfgang von}Goethe, ``Afinidades eletivas'', defendia a adoção do
conceito de obra como \emph{produto} e não como \emph{expressão} de um
indivíduo. O ``eu lírico'' e o ``narrador'' não são um eu em seu sentido
empírico, mas o que podemos chamar de \emph{a forma de um eu}, ou seja,
uma criação de ordem mítica. A despersonalização do sujeito está
estreitamente vinculada à desrealização do mundo na arte moderna, um
movimento mais geral de abstração que recebe de \index{Friedrich, Hugo}Hugo Friedrich, em seu
\emph{Estrutura da lírica moderna},\footnote{Hugo Friedrich,
  \emph{Estrutura da lírica moderna} -- da metade do século \textsc{xix}
  a meados do século \textsc{xx}, tradução de Marise M. Curioni, São
  Paulo, Duas Cidades, 1978.} aquele que talvez seja seu mais conhecido
corolário. Em ``Introduction à l´analyse structurale des récits'',
\index{Barthes, Roland}Barthes esclarece uma importante distinção a esse respeito: ``o autor de
uma narrativa em nada pode ser confundido com seu narrador''.\footnote{Roland
  Barthes, ``Introduction à l'analyse structurale des récits'',
  \emph{Communications}, n. 8, Paris, 1996, pp. 238-241.} De resto, era
esta a pedra de toque da narratologia francesa, que encontrou em
\index{Genette, Gérard}Genette, em \emph{Figures \textsc{iii}} (1972), seu principal
articulador conceitual. Naquele contexto de debates e formulações, a
identificação entre autor e narrador correspondia não menos do que a um
erro grosseiro; o narrador seria, antes, uma criatura fictícia, como as
demais personagens, mas a quem é atribuído o papel de instância
produtora do discurso.

Não é aqui o momento de retomar a fundo essas distinções de
base.\footnote{Cf. o capítulo 1.}~Mas depois de feito este longo e
fundamental percurso, que por certo poderia se deixar orientar ao sabor
das análises mais particularizadas, eu gostaria de insistir na ideia de
que se esse autor (e, com ele, o mundo) \emph{não explica} o texto, isso
não significa que ele desapareça durante a leitura. Esse desaparecimento
é, na verdade, de outra ordem.~Ele está associado a um retoricismo
desinteressante, que, em nome de um rigor exagerado, tem alimentado
abordagens nominalistas e tecnicistas, que, justamente ao isolar o texto
da realidade, afastam os estudantes do prazer, e, portanto, do interesse
pela leitura.

A presença do autor, tal como eu gostaria que fosse considerada, é
abordada numa conferência ministrada em 2006, na Universidade de
Oklahoma, pelo ganhador do prêmio Nobel de 2007, o romancista turco
\index{Pamuk, Orhan}Orhan Pamuk.~Em meio a questões políticas, contas a pagar, telefones
tocando e reuniões de família, ``durante aqueles dias longos e tediosos
de politicagem'', explica o romancista, ``eu não conseguia me tornar o
autor implícito do livro maravilhoso que queria escrever''.\footnote{Orhan
  Pamuk, ``O autor implícito'', in \emph{A maleta de meu pai} (tradução
  de Sergio Flaksman), São Paulo, Companhia das Letras, 2007, pp. 89-90.}
O uso que \index{Pamuk, Orhan}Pamuk faz da expressão com a qual estamos habituados revela o
descentramento constitutivo do escritor: ``(\ldots{}) dediquei todas as
minhas forças a me tornar o autor implícito dos livros que quero
escrever''. Aquele mesmo autor, penso eu, que \index{Saramago, José}Saramago encontrou em si
em seu refúgio em Lanzarotte, e que \index{Bergman, Igmar}Bergman descobriu, serenamente, na
misteriosa ilha de Farö.

O escritor não é simplesmente um indivíduo que escreve, mas um indivíduo
capaz de \emph{se transformar} em alguém que escreve, ainda que
esse outro continue assinando ``\index{Pamuk, Orhan}Pamuk'', \index{Eco, Umberto}``Eco'' ou ``\index{Pessoa, Fernando}Pessoa'' e se apresente com as mesmas feições dos sujeitos empíricos que pagam suas contas e atendem ao
telefone. É preciso compreender que essa assinatura é já algo diferente
daquela com que esses indivíduos preenchem cheques e dão autógrafos, mas
que, ao mesmo tempo, ambas se correspondem, num jogo de empréstimos e
indeterminações. E isso se dá porque a escrita é uma prática tão
empírica quanto as dezenas de outras práticas aparentemente menos
relevantes que ela. Ela só pode ser considerada como à parte do mundo
real se a nossa concepção de realidade se resumir às nossas
necessidades primitivas e aos hábitos entediantes. O sujeito empírico,
entre os muitos sujeitos empíricos moventes que povoam o nosso dia a
dia, é também aquele que escreve. E por que ele deveria se resumir ao
homem entediado que paga as contas e atende a telefonemas é que não se
pode explicar. Se houvesse a necessidade de escolher entre eles, não
seria o autor implícito o eleito?

O direito à fantasia não é o direito de renunciar à vida, pelo
contrário: é o direito de fazer parte dela como um ser transformador.

Se não houvesse essa correspondência, quando \index{Wael, Monique De}Monique De Wael passa a
assinar como \index{Wael, Monique De}Misha Defonseca, ela não estaria contando uma mentira, mas
assumindo Misha como sua autora implícita; como algo, afinal,
absolutamente necessário.~Do mesmo modo, para \index{Seltzer, Margaret}Margaret Seltzer escrever
\emph{Love and Consequences}, antes ela teve de descobrir uma \index{Seltzer, Margaret}Margaret
Jones que pudesse dar voz àquele livro e com a qual sonhava, mas que,
sem ela, não poderia escrever. \index{Wael, Monique De}Defonseca é \index{Wael, Monique De}Wael, como Jones é \index{Seltzer, Margaret}Seltzer, dado que estas são autoras implícitas extraídas, ou descobertas, nesses
indivíduos. Mas o gênero memorialístico força a aproximação entre essas
\emph{personae}, a ponto de se poder presumi"-las como sendo uma só. A
distância entre a porção do indivíduo que não escreve e aquela que
escreve, o autor implícito, deve ser, por definição, a menor possível.
Nesse âmbito, \index{Wael, Monique De}Defonseca só pode dizer \index{Wael, Monique De}Wael na medida em que uma mentira
diz quem a contou. \index{Seltzer, Margaret}Seltzer e Jones possivelmente jamais escreveriam suas
memórias (não antes do escândalo que, agora, ironicamente, passou a
alimentá-las).

O que parece mais interessante e significativo nessas ``fraudes'' é que
essas escritoras produziram exatamente o oposto do que um escritor
costuma fazer, ou seja, não um autor implícito que guarda traços do
indivíduo gerador, mas um indivíduo que se reconstrói à imagem e
semelhança do autor implícito que ele criou.~Elas, afinal, se deixaram
penetrar pelas fábulas que produziram a ponto de assumirem para si o
mito da própria escrita. Foi dessa ousadia extraordinária que resultou
sua condenação: num livro de memórias, é sempre a arte que imita a vida,
a vida está proibida de imitar a arte.

O contrário disso, a criação de um autor implícito que guarda
inevitavelmente traços de seu indivíduo gerador, é descrito com
esclarecedora simplicidade por \index{Pamuk, Orhan}Pamuk, que, depois de constatar já ter
publicado sete romances, afirma:

\begin{quote}
Todos esses sete autores implícitos se parecem comigo, e ao longo dos
últimos trinta anos entraram em contato com a vida e o mundo como são
vistos de Istambul, como são vistos de uma janela como a minha, e, já
que conhecem esse mundo de dentro para fora e se deixaram convencer por
ele, são capazes de descrevê-lo com toda a seriedade e responsabilidade
de uma criança quando brinca.\footnote{Ibid., p. 91.}
\end{quote}

A relação entre leitor e autor implícito pode ser definida como a
relação que se estabelece entre leitor e autor exclusivamente no ato da
leitura.~E ela é, a bem da verdade, essencial: nós e os leitores de
outras \index{Wael, Monique De}Defonseca e \index{Seltzer, Margaret}Jones estamos sempre à sua procura. Mas a magia da
ficção consiste em situá-la num plano já muito diferente daquele em que
nos situaríamos se tivéssemos a felicidade de simplesmente conversar com
o autor e ouvi"-lo a respeito do romance que tanto nos agradou.~O autor
enquanto determinação biográfica ou psicológica está morto. E nós só não
voltamos a falar nessa entidade porque tememos que ela seja confundida
com uma integridade coerente, plenamente racional e esclarecida, repleta
de convicções imutáveis, e que armazena conteúdos prontos para serem
derramados sobre a folha em branco. Nós não falamos mais dessa
personagem, o autor, porque não concebemos a escrita como transparência
que permite entrever a irrupção da interioridade profunda de um
indivíduo.~O receio de que a linguagem literária seja novamente reduzida
a isso acaba por eliminar o autor de nossos discursos críticos. Mas nós
não percebemos que evitá-lo significa ainda concebê-lo sob o prisma
memorialístico ou romântico.

Resta, no entanto, um ser que habita para além da máscara narrativa e
que não pode ser reduzido a ela. Um ser que, embora fraturado e
desconhecido de si próprio, orienta a escolha desse narrador, que se
disfarça nas falas das personagens, que maneja seus encontros e
desencontros, que escolhe seus nomes, seus espaços e tempos, que
distribui e retira ênfases da fabulação, que avalia, enfim, aquilo que o
narrador nos conta e nos mostra, concordando e discordando dele a todo
momento.

Esse \emph{eu} que se entrevê no discurso está desalojado de uma
instância real, na medida em que o que há de verdadeiro nele é uma
implicação do enunciado. Mas, uma vez pronunciado, o enunciado do qual
esse \emph{eu} é objeto é também o enunciado do qual ele passa a ser o
sujeito, novamente alinhado com o mundo. E o que é necessário
compreender é que ele permanece ali, sujeito"-objeto, em sua dimensão
necessariamente equívoca, descontínua e imprevisível.

Ler não é simplesmente ouvir a voz do narrador e das personagens, mas
reparar em como estão dizendo, para que fins são conduzidos. E esse
olhar por detrás das máscaras não significa procurar o que veio antes
delas (uma curiosidade meramente personalista); é, antes, virá-las do
avesso (uma sugestão que se abre nas fendas e orifícios das próprias
máscaras).~Se o autor morto é aquele que para \index{Freud, Sigmund}Freud e \index{Sainte-Beuve, Charles Augustin}Sainte"-Beuve se traía na escrita, é chegada a hora de repensá-lo como alguém ou alguma
coisa que, a todo momento, como uma voz em surdina, \emph{se deixa trair} nela. É dessa comunicação tácita que a literatura é feita.

Estamos já num plano diferente, e é preciso voltar a enxergar a relação
outrora traumática entre o sujeito que escreve e aquele que lê não como
algo proibitivo, mas como uma implicação mútua, constitutiva do que
entendemos por \emph{literário}.~E isso é algo já muito distante da
banalidade.

%--------------------NOMES PARA REFERENCIAR NO ÍNDICE SEM Nº DE PÁGINA
\index{Malheiros, Heitor \also{Visconde de Taunay}|gobbleone}
\index{Scipione, Annibale \also{Oswald de Andrade}|gobbleone}
\index{Bananere, Juó \also{Alexandre Marcondes Machado}|gobbleone}
\index{Eliot, George \also{Mary Ann Evans}|gobbleone}
\index{Sand, George \also{Amantine Aurore Lucille Dupin}|gobbleone}
\index{Lentz, Elza \also{João Ribeiro}|gobbleone}
\index{Azevedo, Maria de \also{João Ribeiro}|gobbleone}
\index{Flag, Suzana \also{Nelson Rodrigues}|gobbleone}
\index{Cendrars, Blaise \also{Fréderic Saucer-Hall}|gobbleone}
\index{Orwell, George \also{Eric Blair}|gobbleone}
\index{Romains, Jules \also{Louis Farrigoule}|gobbleone}
\index{Henry, O. \also{William Sidney Porter}|gobbleone}
\index{Eluard, Paul@Éluard, Paul \also{Eugène Grindel}|gobbleone}
\index{Basoalto, Neftalí Ricardo Reyes \also{Pablo Neruda}|gobbleone}
\index{Defonseca, Misha \also{Monique De Wael}|gobbleone}
\index{LeRoy, J. T. \also{Laura Albert}|gobbleone}
\index{Holstein \also{Bernard Brougham}|gobbleone}
\index{Jones, Margaret B. \also{Margaret Seltzer}|gobbleone}
%----------------------------------------------------------------------

\chapter{Bibliografia}
\hedramarkboth{Bibliografia}{}

\begin{hangparas}{.35in}{1}
\textsc{aguiar e silva}, Vítor Manuel de. \emph{Teoria da literatura}.
3ª. ed. revista e aumentada. Coimbra: Livraria Almedina, 1973.

\textsc{barthes}, Roland. \emph{O rumor da língua}. Tradução de Mário
Laranjeira. São Paulo: Brasiliense, 1988.

\_\_\_\_\_\_. Introduction à l'analyse estructurale des récits.
\emph{Communications}, 8, Paris, 1996.

\textsc{\_\_\_\_\_\_}. \emph{Aula}. Tradução de Leyla Perrone-Moisés.
7\textsuperscript{ª} ed. São Paulo: Cultrix, 1997.

\textsc{benjamin, W}alter. ``Afinidades eletivas de Goethe''. In
\emph{Ensaios reunidos: escritos sobre Goethe}. Tradução de Mônica
Krausz Bornebusch, Irene Aron e Sidney Camargo. Supervisão e notas de
Marcus Vinicius Mazzari. São Paulo: Duas Cidades; Editora 34, 2009.

\textsc{berardinelli}, Alfonso. \emph{Da poesia à prosa}. Organização e
prefácio de Maria Betânia Amoroso; tradução de Maurício Santana Dias.
São Paulo: Cosac Naify, 2007.

\textsc{bloom}, Harold. \emph{A angústia da influência} -- uma teoria da
poesia. Tradução de Marcos Santarrita. Rio de Janeiro: Imago, 1991.

\textsc{\_\_\_\_\_\_}\emph{. O cânone occidental} -- os livros e a
escola do tempo. Tradução de Marcos Santarrita. Rio de Janeiro:
Objetiva, 1994.

\_\_\_\_\_\_. \emph{Poesia e repressão} -- o revisionismo de Blake a
Stevens. Tradução de Cillu Maia. Rio de Janeiro: Imago, 1994.

\_\_\_\_\_\_. \emph{Gênio} -- os 100 autores mais criativos da história
da literatura. Tradução de José Roberto O´Shea. Rio de Janeiro:
Objetiva, 2003.

\textsc{booth}, Wayne C. \emph{A Retórica da ficção}. Tradução de Maria Teresa H. Guerreiro. Lisboa: Artes e Letras/Arcádia, 1983.

\textsc{borges}, Jorge Luis. \emph{Outras inquisições}. Tradução de
Sérgio Molina. São Paulo: Globo, 1989.

\textsc{\_\_\_\_\_\_}. \emph{Ficções}.~Tradução de Davi Arrigucci Jr. São Paulo: Companhia das Letras, 2007.

\textsc{brodsky}, Joseph.~\emph{Menos que um}. Tradução de Sergio Flaksman. São Paulo: Companhia das Letras, 1994.

\textsc{chalmers}, Vera Maria. ``A crônica humorística de \emph{O Pirralho}''.
\emph{Revista de Letras}, v. 30. Unesp-Araraquara, São Paulo, 1990.

\textsc{chklovski}, Vitor. ``A arte como procedimento''. In: \emph{Teoria da
literatura: formalistas russos}. Organização de Dionísio de Oliveira
Toledo; tradução de Ana Mariza Ribeiro, Maria Aparecida Pereira, Regina
L. Zilberman e Antônio Carlos Hohlfeld. Porto Alegre: Globo, 1971.

\textsc{combe}, Dominique. ``La référence dédoublée, le sujet lyrique
entre fiction et autobiographie''. In: \emph{Teorías sobre la lírica}.
\emph{Figures du Sujet Lyrique}. Organização de Dominique Rabaté. Paris:
\textsc{puf},1996.

\_\_\_\_\_\_. ``A referência desdobrada. O sujeito lírico entre a ficção
e a autobiografia''. Tradução de Iside Mesquita e Vagner Camilo. In
\emph{Revista \versal{USP}}, n. 84, dez.-fev. 2009, 2010.

\textsc{compagnon}, Antoine. \emph{Literatura para quê?} Tradução de
Laura Taddei Brandini. Belo Horizonte: Editora \versal{UFMG}, 2009.

\_\_\_\_\_\_. \emph{O demônio da teoria}: literatura e senso comum.
Tradução de Cleonice Paes Barreto Mourão e Consuelo Fortes Santiago.
Belo Horizonte: Ed. \textsc{ufmg}, 2003.

\textsc{croce}, Benedetto. \emph{A poesia}: introdução à crítica e
história da poesia e da literatura. Tradução de Flávio Loureiro
Chaves; supervisão e revisão de Angelo Ricci. Porto Alegre: Ed.
\textsc{ufrgs}, 1965.

\textsc{culler}, Jonathan. \emph{Teoria literária} -- uma introdução.
Tradução de Sandra Vasconcelos. São Paulo: Beca, 1999.

\textsc{de man}, Paul. \emph{O ponto de vista da cegueira}: ensaios
sobre a retórica da crítica contemporânea. Tradução de Miguel Tamen.
Braga; Coimbra; Lisboa: Angelus Novus/ Cotovia, 1999.

\textsc{derrida}, Jacques. \emph{A escritura e a diferença}. 2ª ed.
Tradução de Maria Beatriz Marques Nizza da Silva. São Paulo:
Perspectiva, 1995.

\_\_\_\_\_\_. \emph{A voz e o fenómeno}. Tradução de Maria José Semião e
Carlos Aboim de Brito. Lisboa: Edições 70, 1996.

\textsc{eco}, Umberto. \emph{Interpretação e superinterpretação}.
Tradução de Monica Stahel. São Paulo: Martins Fontes, 1993.

\textsc{eliot}, T. S. ``Tradition and the individual talent''. In:
\emph{Selected prose}. Londres: Penguin Books / Faber and Faber, 1955.

\_\_\_\_\_\_. ``The Social Function of Poetry''. In: \emph{On poetry and
poets}. Londres; Boston: Faber and Faber, 1984.

\emph{Estranhar Pessoa}, n. 3. Editores Rita Patrício e Gustavo Rubim;
Lisboa: Universidade Nova de Lisboa, Outono de 2016.

\textsc{ezell}, Margaret J. M. ``Reading Pseudonyms in
Seventeenth-Century English Coterie Literature''. \emph{Essays in
Literature}, n. 21, Spring, 1994.

\textsc{fish}, Stanley. \emph{Is there a text in this class?} The
authority of interpretative communities. Cambridge/Mass.: Harvard
University Press, 1980.

\textsc{foucault}, Michel. \emph{O que é um autor}? Tradução de António
Fernando Cascais. Lisboa: Vega, 2002.

\textsc{friedrich}, Hugo. \emph{Estrutura da lírica moderna}: da metade
do século \textsc{xix} a meados do século \textsc{xx}. Tradução de
Marise M. Curioni. São Paulo: Duas Cidades, 1978.

\textsc{freud}, Sigmund. \emph{Escritos sobre literatura}. Tradução de
Saulo Krieger. Org. de Iuri Pereira. Posfácio de Noemi Moritz Kon. São
Paulo: Hedra, 2014.

\_\_\_\_\_\_. ``O delírio e os sonhos na \emph{Gradiva}''. In
\emph{Obras completas}, vol. 8. Tradução de Paulo César de Souza. São
Paulo, Companhia das Letras, 2010.

\textsc{genette}, Gerard. \emph{Figures \textsc{iii}}. Paris: Éditions
du Seuil, 1972.

\textsc{george}, A. G. \emph{T. S. Eliot: his mind and art}. Londres:
Asia Publishing House, s/d.

\textsc{gil}, José. \emph{Diferença e negação na poesia de Fernando
Pessoa}. Rio de Janeiro: Relume Dumará, 2000.

\textsc{griffin}, Robert J. ``Anonymity and Authorship''. \emph{New Literary
History}, v. 30, n. 4, The Johns Hopkins University Press, 1999.

\textsc{hamburger}, Käte. \emph{A lógica da criação literária}. 2ª ed.
Tradução de Margot Petry Malnic. São Paulo: Perspectiva, 1986.

\textsc{hamburger}, Michael. \emph{A verdade da poesia} -- tensões na
poesia modernista desde Baudelaire. Tradução de Alípio Correia de Franca
Neto. São Paulo: Cosac Naif, 2007.

\textsc{hobsbawm}, Eric. \emph{Era dos Extremos}: o breve século
\textsc{xx}, 1914-1991. 2\textsuperscript{ª} ed. Tradução de Marcos
Santarrita. São Paulo: Companhia das Letras, 1994.

\textsc{iser}, Wolfgang. \emph{The act of reading}: a theory of
aesthetic response. Baltimore: Johns Hopkins University Press, 1978.

\_\_\_\_\_\_. \emph{The implied reader}. Patterns of communication in
prose fiction from Bunyan to Beckett. Baltimore: Johns Hopkins
University Press, 1974.

\textsc{jakobson}, Roman. \emph{Linguística e comunicação}.
5\textsuperscript{ª} ed. Tradução de Izidoro Blikstein e José Paulo
Paes. São Paulo: Cultrix, 1971.

\textsc{\_\_\_\_\_\_.} \emph{Linguística; poética; cinema}. Tradução de
Haroldo de Campos. São Paulo: Perspectiva, 1970.

\_\_\_\_\_\_. \emph{Huit questions de poétique}. Paris: Seuil, 1977.

\textsc{lima}, Luiz Costa (org.). \emph{Teoria da literatura em suas
fontes}. 3ª ed. Rio de Janeiro: Civilização Brasileira, 2002. 2 v.

\textsc{lopes}, Óscar. ``Fernando Pessoa''. In: \emph{Entre Fialho e
Nemésio}: estudos de literatura portuguesa contemporânea, v. 2. Lisboa:
Imprensa Nacional-Casa da Moeda, 1987.

\textsc{lourenço}, Eduardo. \emph{Fernando Pessoa revisitado}: leitura
estruturante do drama em gente\emph{.} 2ª ed. Lisboa: Moraes Editores,
1981.

\textsc{nietzsche}. \emph{O nascimento da tragédia} ou Helenismo e
pessimismo. 2ª. ed. Tradução, notas e posfácio de Jacó Guinsburg. São
Paulo: Companhia das Letras, 1999.

\_\_\_\_\_\_. \emph{Sobre verdade e mentira.} Organização e tradução de
Fernando de Moraes Barros. São Paulo: Hedra, 2007.

\textsc{nóbrega}, Mello. \emph{Ocultação e disfarce da autoria}: do
anonimato ao nome literário. Fortaleza: Ed. \textsc{ufc}, 1981.

\textsc{mallarmé}. \emph{Poemas}. Organização e tradução de José Lino
Grünewald. Rio de Janeiro: Nova Fronteira, 1990.

\textsc{monteiro}, Adolfo Casais. \emph{Estudos sobre a poesia de
Fernando Pessoa}. Rio de Janeiro: Agir, 1958.

\emph{Orpheu 2} {[}1915{]}. 2\textsuperscript{a}. reed. Preparação de
texto e introdução de Maria Aliete Galhoz. Lisboa: Ática, 1979.

\textsc{pamuk}, Orhan. ``O autor implícito''. In: \emph{A maleta de meu
pai}. Tradução de Sergio Flaksman. São Paulo: Companhia das Letras,
2007.

\textsc{pasolini}, Pier Paolo. ``Meu `Accattone' na TV após o
genocídio''. In: \emph{Os jovens infelizes}: antologia de ensaios
corsários. Tradução de Michel Lahud e Maria Betânia Amoroso. São Paulo:
Brasiliense, 1990.

\textsc{pessoa}, Fernando. \emph{Poemas completos de Alberto Caeiro}.
Edição de Teresa Sobral Cunha. Lisboa: Presença, 1994.

\_\_\_\_\_\_. \emph{Correspondência: 1905-1922}. Organização de Manuela
Parreira da Silva. São Paulo: Companhia das Letras, 1999.

\_\_\_\_\_\_. \emph{Correspondência: 1923-1935}. Organização de Manuela
Parreira da Silva. Lisboa: Assírio \& Alvim, 1999.

\_\_\_\_\_\_. \emph{Poesia -- Ricardo Reis}. Organização de Manuela
Parreira da Silva. São Paulo: Companhia das Letras, 2000.

\_\_\_\_\_\_. \emph{Obra poética}. 3ª ed. Organização de Maria Aliete
Galhoz. Rio de Janeiro: Nova Aguilar, 2001.

\_\_\_\_\_\_. \emph{Poesia -- Alberto Caeiro}. Edição de Fernando Cabral
Martins e Richard Zenith. São Paulo: Companhia das Letras, 2001.

\_\_\_\_\_\_. \emph{Poesia -- Álvaro de Campos.} Edição de Teresa Rita
Lopes. São Paulo: Companhia das Letras, 2002.

\_\_\_\_\_\_. \emph{Poesia -- 1902-1917}. Edição de Manuela Parreira da
Silva, Ana Maria Freitas e Madalena Dine). São Paulo: Companhia das
Letras, 2006.

\_\_\_\_\_\_. \emph{Poemas de Fernando Pessoa -- Rubaiyat}, Edição
crítica de Maria Aliete Galhoz. Lisboa: \versal{INCM}, 2008.

\_\_\_\_\_\_. \emph{Poesia -- 1931-1935 e não datada}. Edição de Manuela
Parreira da Silva, Ana Maria Freitas e Madalena Dine. Lisboa: Assírio \&
Alvim, 2016.

\textsc{picchio}, Luciana Stegagno. ``Chuva Oblíqua: dall'Infinito
turbolento di F. Pessoa all'Intersezionismo portoguese''. \emph{Quaderni
Portoguesi}, n. 2, Pisa, Outono 1977.

\textsc{proust}, Marcel. ``O método de Sainte-Beuve''. In: \emph{Contre
Sainte-Beuve}: notas sobre crítica e literatura. Tradução de Haroldo
Ramanzini. São Paulo: Iluminuras, 1988.

\textsc{ricoeur}, Paul. \emph{O si mesmo como um outro}. Tradução de
Lucy Moreira Cesar. Campinas: Papirus, 1991.

\textsc{santiago}, Silviano. \emph{Uma literatura nos trópicos}. São
Paulo: Perspectiva, 1978.

\textsc{seabra}, José Augusto. \emph{Fernando Pessoa ou o poetodrama}.
São Paulo: Perspectiva, 1974.

\textsc{sena}, Jorge de. \emph{Fernando Pessoa \& Cª Heterónima:}
estudos coligidos, 1940-1978. 3ª ed. Organização de Mécia de Sena.
Lisboa: Edições 70, 2000.

\textsc{simões}, João Gaspar. \emph{Vida e obra de Fernando Pessoa}:
história duma geração. 6ª ed. Lisboa: Publicações Dom Quixote, 1991.

\textsc{staiger}, Emil. \emph{Conceitos fundamentais de poética.}
Tradução de Celeste A. Galeão. Rio de Janeiro: Tempo Brasileiro, 1972.

\textsc{todorov, T}zvetan. \emph{A literatura em perigo}. Tradução de
Caio Meira. Rio de Janeiro: Difel, 2009.

\_\_\_\_\_\_. \emph{Teoria da literatura} -- textos dos formalistas
russos. Tradução de Roberto Leal Ferreira. São Paulo: Editora Unesp,
2013.

\textsc{trilling}, Lionel. \emph{The liberal imagination} -- essays on
literature and society. New York: The Viking Press, 1950.

\textsc{wellek,} René \textsc{\& warren,} Austin. \emph{Teoria da
literatura}. 2ª. ed. Tradução de José Palla e Carmo. Lisboa: Publicações
Europa América, 1971.

\textsc{white}, Hayden. ``O momento absurdista na teoria literária
contemporânea''. In: \emph{Trópicos do discurso}: ensaios sobre a
crítica da cultura. Tradução de Alípio Correia de Franca Neto. São
Paulo: \versal{EDUSP}, 1994.

\textsc{wimsatt}, William K. \& \textsc{beardsley}, Monroe. ``The
intentional fallacy''. In \textsc{Beardsley}, Monroe. \emph{The verbal
icon}. Studies in the meaning of poetry. Lexington: University of
Kentucky Press, 1954.

\textsc{wimsatt}, William K. \& \textsc{brooks}, Cleanth. \emph{Crítica
literária} -- breve história. Tradução de Ivete Centeno e Armando de
Moraes. Lisboa: Fundação Calouste Gulbenkian, 1971.
\end{hangparas}


\pagebreak

\textsc{Artigos de jornais}\\

\hedramarkboth{Bibliografia}{}

\begin{hangparas}{.35in}{1}
\textsc{eskin}, Blake. ``Crying Wolf: Why did it take so long for a
far"-fetched holocaust memoir to be debunked?''.\\
Disponível em: \textless{}http://archive.is/iHRk6\textgreater{}. 
Acesso em: 22 nov. 2017.

\textsc{irvine}, Lindesay. ``Nazi flight memoir was fiction, author
confesses''. \\
Disponível em: \textless{}https://bit.ly/2RqYCWj\textgreater{}.
Acesso em: 22 nov. 2017.

\textsc{kakutani}, Michiko. ``However Mean the Streets, Have an Exit
Strategy''. \\
Disponível em: \textless{}https://nyti.ms/2yL6ew2\textgreater{}.
Acesso em: 22 nov. 2017.

\textsc{memmott}, Carol. ``Author´s ´Love and consequences´ memoir
untrue''. \\
Disponível em: \textless{}https://bit.ly/2yL3gaF\textgreater{}.
Acesso em: 22 nov. 2017.

\textsc{o'rourke}, Meghan. ``Lies and Consequences: Why are book editors
so bad at spotting fake memoirs?''. \\
Disponível em: \textless{}https://bit.ly/2Rt6YfZ\textgreater{}.
Acesso em: 22 nov. 2017.

\textsc{rich}, Motoko. ``Gang Memoir, Turning Page, Is Pure Fiction''.\\
Disponível em: \textless{}https://nyti.ms/2QgwLYV\textgreater{}.
Acesso em: 22 nov. 2017.
\end{hangparas}

%\afterpage{\blankpage}
