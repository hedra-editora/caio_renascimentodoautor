
\textbf{Caio Gagliardi} nasceu em São Paulo, em 1974. Pesquisa e ensina literatura portuguesa nos cursos de graduação e pós"-graduação da Universidade de São Paulo, onde coordena o grupo \emph{Estudos Pessoanos}, a respeito de Fernando Pessoa. É mestre e doutor em Teoria e História Literária pela \versal{UNICAMP} e realizou pós"-doutorados no Departamento de Teoria Literária -- \versal{USP} e no Dipartimento di Studi Europei, Americani e Interculturali da Università degli Studi di Roma ``La Sapienza''. Organizou, pela Hedra, as edições de \emph{O Ateneu}, de Raul Pompeia, \emph{Mensagem} e \emph{Teatro do Êxtase}, ambos de Fernando Pessoa, e prefaciou \emph{A cidade e as serras}, de Eça de Queirós, e \emph{Poemas completos de Alberto Caeiro}.


\textbf{O renascimento do autor} é um estudo a respeito do moderno conceito de autoria. Seus ensaios, escritos com clareza e entremeados por uma tese unificadora que pouco a pouco vai se evidenciando, incidem sobre três esferas complementares: a teoria da literatura, a obra de Fernando Pessoa e as \emph{fake memoirs}. Entre os muitos teóricos abordados no livro, Gagliardi examina sob diferentes ângulos ``A morte do autor'', de Roland Barthes. Especialista em Fernando Pessoa, o crítico convoca a obra do escritor português para repensar a autoria com base em sua experiência poética mais radical e, simultaneamente, para propor a reinterpretação da heteronímia. O desfecho do livro amalgama seus múltiplos enfoques ao conferir especial atenção às fraudes autorais na literatura contemporânea.

